%-------------------------------------------------------------------------------
%                            BAB IV
%               		HASIL DAN PEMBAHASAN
%-------------------------------------------------------------------------------
% \fancyhf{} 
% \fancyfoot[R]{\thepage}
\chapter{HASIL DAN PEMBAHASAN}
%\thispagestyle{plain} % Halaman pertama bab menggunakan gaya plain

\section{Pemodelan Spektra Emisi untuk Diagnostik Plasma}
\subsection{Penggunaan Persamaan Saha dan Rasio Intensitas untuk Karakterisasi Plasma}
Simulasi spektra emisi menggunakan intensitas $I_{ij}$ dan populasi $N_i$ dari persamaan Saha untuk menentukan suhu dan kepadatan elektron plasma. Data eksperimental digunakan untuk validasi \citep{Mason2015}.

\subsection{Aplikasi Profil Voigt dalam Dekonvolusi Fitur Spektral Kompleks}
\par Profil Voigt diterapkan secara numerik untuk mendekonvolusi garis spektral, memungkinkan analisis fitur kompleks dalam spektra plasma. Ini meningkatkan akurasi diagnostik plasma \citep{Demtroder2010}.
\section{Hasil Simulasi Spektrum}
\subsection{Tingkat Energi}
[Isi Bab 4.1.1]

\subsection{Populasi Energi}
[Isi Bab 4.1.2]

\subsection{Profil Garis}
[Isi Bab 4.1.3]

\section{Prediksi dengan Transformer}
[Isi Bab 4.2: Performa dan efisiensi.]

\section{Efek Matriks}
[Isi Bab 4.3: Simulasi dan mitigasi.]

\section{Analisis LTE}
[Isi Bab 4.4: Sensitivitas deviasi.]

\section{Perbandingan Penelitian}
[Isi Bab 4.5: Keunggulan kuantum-Transformer.]

\section{Implikasi}
[Isi Bab 4.6: Teoritis dan praktis.]


\begin{table}[H]
    \centering
    \caption{Hasil Pengujian Akurasi Menggunakan SVM Terhadap Data \textit{Training} dan \textit{Testing}}
    \label{tb_detail_akurasi_face}
    \begin{tabular}{lcccc}
    \toprule
    \textbf{Jenis Data} & \textbf{Jumlah Label} & \textbf{Jumlah Data} & {\color[HTML]{000000} \textbf{Akurasi}} \\ 
    \midrule
    {\color[HTML]{000000} Training} & {\color[HTML]{000000} 41} & {\color[HTML]{000000} 1640} & {\color[HTML]{000000} 99,51\%} \\ 
    {\color[HTML]{000000} Testing} & {\color[HTML]{000000} 41} & {\color[HTML]{000000} 410} & {\color[HTML]{000000} 96,34\%} \\ 
    \bottomrule
    \end{tabular}
    \end{table}

    \begin{table}[H]
        \centering
        \caption{Perbandingan Performa Model Performa Model Performa Model Performa Model Machine Learning pada Data LIBS}
        \label{tab:performa_ml}
        \centering
        \begin{tabular}{lcccc}
          \toprule
          Model & Akurasi (\%) & Presisi (\%) & Recall (\%) & RMSE \\
          \midrule
          Random Forest & 95.2 & 94.8 & 95.1 & 0.12 \\
          SVM & 89.7 & 88.5 & 90.2 & 0.21 \\
          Transformer & 97.1 & 96.9 & 97.0 & 0.08 \\
          CNN & 93.4 & 92.7 & 93.5 & 0.15 \\
          \bottomrule
        \end{tabular}
        
        \smallskip
        \footnotesize
        \textit{Keterangan:} Data diperoleh dari 100 sampel logam dengan 5 kelas komposisi.
      \end{table}
    

% 