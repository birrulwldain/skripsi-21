%-------------------------------------------------------------------------------
%                            BAB IV
%               		HASIL DAN PEMBAHASAN
%-------------------------------------------------------------------------------
% \fancyhf{} 
% \fancyfoot[R]{\thepage}
\chapter{HASIL DAN PEMBAHASAN}
%\thispagestyle{plain} % Halaman pertama bab menggunakan gaya plain


\begin{table}[H]
    \centering
    \caption{Hasil Pengujian Akurasi Menggunakan SVM Terhadap Data \textit{Training} dan \textit{Testing}}
    \label{tb_detail_akurasi_face}
    \begin{tabular}{|c|c|c|c|}
    \hline
    \textbf{Jenis Data} & \textbf{Jumlah Label} & \textbf{Jumlah Data} & {\color[HTML]{000000} \textbf{Akurasi}} \\ \hline
    {\color[HTML]{000000} Training} & {\color[HTML]{000000} 41} & {\color[HTML]{000000} 1640} & {\color[HTML]{000000} 99,51\%} \\ \hline
    {\color[HTML]{000000} Testing} & {\color[HTML]{000000} 41} & {\color[HTML]{000000} 410} & {\color[HTML]{000000} 96,34\%} \\ \hline
    \end{tabular}
    \end{table}

% Baris ini digunakan untuk membantu dalam melakukan sitasi
% Karena diapit dengan comment, maka baris ini akan diabaikan
% oleh compiler LaTeX.
\begin{comment}
\bibliography{daftar-pustaka}
\end{comment}