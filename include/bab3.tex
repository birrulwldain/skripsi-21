%-------------------------------------------------------------------------------
%                            BAB III
%               		METODE PENELITIAN
%-------------------------------------------------------------------------------


\chapter{METODE PENELITIAN}

Dalam pengembangan model \textit{Long Short-Term Memory} (LSTM), persiapan data yang tepat sangat memengaruhi kinerja model. Bab ini membahas teknik persiapan data numerik dan kategorikal serta penanganan urutan dengan panjang bervariasi.

\section{Waktu dan Lokasi Penelitian}
Penelitian ini dilaksanakan di Laboratorium Gelombang dan Optik, Departemen Fisika, Fakultas Matematika dan Ilmu Pengetahuan Alam, Universitas Syiah Kuala, dari September 2024 hingga Juni 2025.

\section{Jadwal Pelaksanaan Penelitian}
Penelitian ini direncanakan berlangsung selama 8 bulan, dari September 2024 hingga Juni 2025. Tabel~\ref{tab:jadwal_penelitian} menunjukkan jadwal pelaksanaan penelitian yang mencakup fase studi literatur, pengumpulan data, pengembangan model, validasi, dan penulisan laporan akhir.

\begin{table}[H] % Pastikan Anda memuat paket 'float' jika menggunakan [H]
  \centering
  \caption{Jadwal Penelitian (mulai Desember 2024)}
  \label{tab:jadwal_penelitian_des24_berwarna} % Label bisa diubah jika perlu
  \small
  \setlength{\tabcolsep}{3pt}
  % Kolom pertama 20% dari \textwidth, diikuti 6 kolom rata kiri untuk bulan
  % Tipe kolom 'l' (rata kiri) atau 'c' (rata tengah) cocok untuk sel berwarna
  \begin{tabular}{p{0.2\textwidth} *{6}{l}}
    \toprule
    \textbf{Fase Penelitian} & \textbf{Des 2024} & \textbf{Jan 2025} & \textbf{Feb 2025} & \textbf{Mar 2025} & \textbf{Apr 2025} & \textbf{Mei 2025} \\
    \midrule
    Studi literatur dan perancangan awal & & & & & & \\
    Pengumpulan dan pengolahan data spektral & \cellcolor{gray!25} & & & & & \\
    Pengembangan dan pengujian model prediksi & \cellcolor{gray!25} & \cellcolor{gray!25} & \cellcolor{gray!25} & \cellcolor{gray!25} & & \\
    Validasi model dan analisis hasil & & & & & \cellcolor{gray!25} & \cellcolor{gray!25} \\
    Penulisan laporan akhir & & & & & & \cellcolor{gray!25} \\
    \bottomrule
  \end{tabular}
  \vspace{0.2cm}
  \footnotesize
  % Catatan diubah untuk menjelaskan penggunaan sel berwarna
  \textit{Catatan:} Sel yang diarsir/berwarna (misalnya, \colorbox{gray!25}{\phantom{X}}\hspace{0.1em}) menunjukkan periode pelaksanaan masing-masing fase.
\end{table}

\section{Alat dan Bahan}
Penelitian ini memanfaatkan berbagai alat dan bahan yang mencakup perangkat keras, perangkat lunak, serta sumber data untuk mendukung proses pengumpulan, pengolahan, analisis, dan visualisasi data spektral atomik. Berikut adalah rincian alat dan bahan yang digunakan:

\subsection{Perangkat Keras}
\begin{enumerate}
  \item \textbf{Laptop \textit{Apple MacBook} Air M1 2020}: Dilengkapi dengan prosesor Apple M1, memori (RAM) sebesar \SI{8}{\giga\byte}, dan penyimpanan internal berbasis SSD. Perangkat ini digunakan untuk persiapan data, eksplorasi awal, pengembangan kode, serta penyusunan laporan penelitian.
  \item \textbf{\textit{Google Colaboratory}}: Lingkungan komputasi berbasis awan yang menyediakan akses ke unit pemrosesan grafis (GPU) NVIDIA Tesla T4 dengan memori \SIrange{15}{20}{\giga\byte}. Platform ini digunakan untuk pelatihan model \textit{machine learning}, evaluasi performa model, serta komputasi intensif lainnya.
\end{enumerate}

\subsection{Perangkat Lunak}
\begin{enumerate}
  \item \textbf{Sistem Operasi}:
  \begin{itemize}
    \item \textit{macOS Ventura} 13.6: Digunakan pada perangkat lokal untuk pengembangan dan pengujian awal.
    \item \textit{Ubuntu}: Digunakan dalam lingkungan virtual \textit{Google Colaboratory} untuk komputasi berbasis awan.
  \end{itemize}
  
  \item \textbf{Bahasa Pemrograman dan Lingkungan Kerja}:
  \begin{itemize}
    \item Python 3.8.17: Bahasa pemrograman utama untuk pengembangan algoritma dan analisis data.
    \item Jupyter Notebook: Digunakan baik secara lokal maupun pada \textit{Google Colaboratory} untuk pengembangan kode interaktif dan dokumentasi analisis.
  \end{itemize}
  
  \item \textbf{Pustaka dan Modul Python}:
  \begin{itemize}
    \item NumPy (versi 1.24.0) dan Pandas (versi 1.5.3): Untuk manipulasi, eksplorasi, dan analisis data numerik serta tabular.
    \item h5py: Untuk pengelolaan dataset dalam format HDF5, termasuk pembacaan dan penyimpanan data.
    \item scikit-learn (versi 1.2.0): Untuk prapemrosesan data, evaluasi model \textit{machine learning}, dan visualisasi dimensi rendah menggunakan algoritma t-SNE (\texttt{sklearn.manifold.TSNE}).
    \item Matplotlib (versi 3.6.2): Untuk pembuatan visualisasi grafik dan representasi hasil analisis.
    \item Joblib (versi 1.2.0): Untuk serialisasi model dan optimalisasi pemrosesan paralel.
    \item PyTorch (versi 1.13.1): Kerangka kerja pembelajaran mendalam untuk pengembangan, pelatihan, dan evaluasi model.
    \item TensorBoard (versi untuk PyTorch): Untuk pemantauan metrik pelatihan dan validasi model secara \textit{real-time}.
    \item itertools: Modul standar Python untuk operasi iterasi kompleks.
  \end{itemize}
\end{enumerate}

\subsection{Sumber Data}
\begin{enumerate}
\item \textbf{\textit{NIST Atomic Spectra Database} (ASD)}: Basis data resmi dari \textit{National Institute of Standards and Technology }(NIST) yang digunakan sebagai sumber utama parameter spektral atomik. Parameter yang diambil meliputi energi ionisasi (\(E_\text{i}\)), energi keadaan (\(E_\text{k}\)), bobot statistik (\(g_\text{i}\), \(g_\text{k}\)), dan koefisien probabilitas transisi (\(A_\text{ki}\)). Data diakses melalui antarmuka daring resmi NIST.\footnote{\url{https://physics.nist.gov/PhysRefData/ASD/}}
\end{enumerate}

\section{Prosedur Penelitian}

Prosedur penelitian ini dirancang untuk mensimulasikan spektrum emisi atom dalam plasma pada kondisi kesetimbangan termal lokal (\textit{Local Thermodynamic Equilibrium}, LTE), dengan tujuan menentukan suhu eksitasi berdasarkan intensitas garis spektral. Pendekatan ini mengintegrasikan perhitungan kuantum mekanis, statistik termodinamika, dan pemodelan \textit{machine learning} untuk analisis data spektral. Diagram alur penelitian (Gambar~\ref{fig:3-diagram}) menggambarkan langkah-langkah utama, yang diuraikan sebagai berikut:

\begin{figure}[H]
    \centering
    \includegraphics[width=1\textwidth]{images/3-Diagram.drawio.pdf}
    \caption{Diagram alur penelitian.}
    \label{fig:3-diagram}
\end{figure}

\begin{algorithm}
\small
\caption{Initialization of Atomic Spectral Simulation Parameters}
\begin{algorithmic}[1]
  \REQUIRE
    Transition dataset $\mathcal{D} = \{ (\lambda_{ij}, E_i, E_k, g_i, g_k, A_{ki}) \mid \lambda_{ij} \in [300, 800], E_i, E_k, g_i, g_k, A_{ki} > 0 \}$ \cite{Kramida2023}; \\
    Number of element-ion pairs $k = 4$; \\
    Maximum spectral samples $N \in \mathbb{N}$; \\
    Temperature step $\Delta T > 0$; \\
    Electron density step $\Delta n_e > 0$; \\
    Gaussian broadening parameter $\sigma > 0$
  \ENSURE
    Candidate atom set $\mathcal{C}$, atom subset dictionary $\mathcal{A}$, spectra set $\mathcal{S}$
  \STATE Validate inputs: Ensure $N > 0$, $\Delta T > 0$, $\Delta n_e > 0$, $\sigma > 0$
  \STATE Initialize: $\Delta T \gets 1000$, $\Delta n_e \gets 10^{0.5} \times 10^{12}$, $\sigma \gets 0.1$
  \STATE Define: $\mathcal{C} \gets \{\text{H}, \text{He}, \text{O}, \text{N}, \text{Si}, \text{Al}, \text{Fe}, \text{Ca}, \text{Mg}, \text{Na}, \text{Ti}, \text{Mn}, \text{S}, \text{Cl}, \text{Cr}, \text{Ni}, \text{Cu}\}$
  \STATE Initialize: $\mathcal{S} \gets \emptyset$, $\mathcal{A} \gets \emptyset$
  \FORALL{$T \in [5000, 15000]$ \textbf{step} $\Delta T$}
    \FORALL{$n_e \in [10^{12}, 10^{16}]$ \textbf{step} $\Delta n_e$}
      \STATE Randomly select $\mathcal{A}_{T,n_e} \subseteq \mathcal{C}$ with $|\mathcal{A}_{T,n_e}| = k$ without replacement
      \IF{no transitions exist in $\mathcal{D}$ for any species in $\mathcal{A}_{T,n_e}$}
        \STATE Log warning: ``No transitions for $\mathcal{A}_{T,n_e}$ at $T$, $n_e$'' \COMMENT{Skip}
        \STATE continue
      \ENDIF
      \STATE Store $(T, n_e, \mathcal{A}_{T,n_e})$ in $\mathcal{A}$
    \ENDFOR
  \ENDFOR
  \STATE \RETURN $\mathcal{C}$, $\mathcal{A}$, $\mathcal{S}$
\end{algorithmic}
\end{algorithm}

\begin{algorithm}
\small
\caption{Ionization and Population Calculations}
\begin{algorithmic}[1]
  \REQUIRE Transition dataset $\mathcal{D} = \{ (\lambda_{ij}, E_i, E_k, g_i, g_k, A_{ki}) \mid \lambda_{ij} \in [300, 800], E_i, E_k, g_i, g_k, A_{ki} > 0 \}$; Candidate atom set $\mathcal{C}$; Atom subset dictionary $\mathcal{A} = \{ (T, n_e, \mathcal{A}_{T,n_e}) \}$; Physical constants $m_e = 9.109 \times 10^{-31}$, $k_B = 8.617 \times 10^{-5}$, $h = 4.1357 \times 10^{-15}$
  \ENSURE Population ratio dictionary $\mathcal{R} = \{ (T, n_e, \mathbf{R}_{T,n_e}) \mid \mathbf{R}_{T,n_e} \in [0, 1]^2 \}$
  \STATE Initialize $\mathcal{R} \gets \emptyset$ \COMMENT{Ratio dictionary}
  \FORALL{$(T, n_e, \mathcal{A}_{T,n_e}) \in \mathcal{A}$}
    \STATE $\mathbf{R}_{T,n_e} \gets \emptyset$ \COMMENT{Temporary ratio}
    \FORALL{$S \in \mathcal{A}_{T,n_e}$}
      \STATE Define $(S_{\text{neutral}}, S_{\text{ion}}) \gets (S_{\text{neutral}}, S_{\text{ion}})$ \COMMENT{Species pair}
      \STATE Extract $\mathcal{T}_S \subseteq \mathcal{D}$ for $S_{\text{neutral}}$ or $S_{\text{ion}}$ \COMMENT{Transitions}
      \IF{$\mathcal{T}_S = \emptyset$}
        \STATE Log warning: ``No transitions for $S$'' \COMMENT{Skip}
        \STATE continue
      \ENDIF
      \STATE $Z_{\text{neutral}} \gets \sum_i g_i \exp\left(-\frac{E_i}{k_B T}\right)$ \COMMENT{Neutral partition}
      \IF{$Z_{\text{neutral}} \leq 0$}
        \STATE Log warning: ``Invalid partition function for $S_{\text{neutral}}$'' \COMMENT{Skip}
        \STATE continue
      \ENDIF
      \STATE $Z_{\text{ion}} \gets \sum_i g_i \exp\left(-\frac{E_i}{k_B T}\right)$ \COMMENT{Ion partition}
      \IF{$Z_{\text{ion}} \leq 0$}
        \STATE Log warning: ``Invalid partition function for $S_{\text{ion}}$'' \COMMENT{Skip}
        \STATE continue
      \ENDIF
      \STATE $\frac{N_{\text{ion}}}{N_{\text{neutral}}} \gets \frac{2 Z_{\text{ion}}}{n_e Z_{\text{neutral}}} \left( \frac{2\pi m_e k_B T}{h^2} \right)^{3/2} \exp\left(-\frac{E_{\text{ion}}}{k_B T}\right)$ \COMMENT{Ionization ratio}
      \STATE $f_{\text{neutral}} \gets \frac{1}{1 + \frac{N_{\text{ion}}}{N_{\text{neutral}}}}$ \COMMENT{Neutral fraction}
      \STATE $f_{\text{ion}} \gets \frac{\frac{N_{\text{ion}}}{N_{\text{neutral}}}}{1 + \frac{N_{\text{ion}}}{N_{\text{neutral}}}}$ \COMMENT{Ion fraction}
      \STATE Store $(f_{\text{neutral}}, f_{\text{ion}})$ in $\mathbf{R}_{T,n_e}[S_{\text{neutral}}, S_{\text{ion}}]$ \COMMENT{Store ratio}
    \ENDFOR
    \STATE Store $(T, n_e, \mathbf{R}_{T,n_e})$ in $\mathcal{R}$ \COMMENT{Add to dictionary}
  \ENDFOR
  \STATE \RETURN $\mathcal{R}$ \COMMENT{Result}
\end{algorithmic}
\end{algorithm}

\begin{algorithm}
\small
\caption{Modified Spectral Intensity Calculation}
\begin{algorithmic}[1]
  \REQUIRE 
    Transition dataset $\mathcal{D} = \{ (\lambda_{ij}, E_i, E_k, g_i, g_k, A_{ki}, m_a, w) \mid \lambda_{ij} \in [300, 800], E_i, E_k, g_i, g_k, A_{ki}, m_a, w > 0 \}$; \\
    Atom subset dictionary $\mathcal{A} = \{ (T, n_e, \mathcal{A}_{T,n_e}) \}$; \\
    Population ratio dictionary $\mathcal{R} = \{ (T, n_e, \mathbf{R}_{T,n_e}) \mid \mathbf{R}_{T,n_e} \in [0, 1]^2 \}$; \\
    Physical constants $c = 2.998 \times 10^8$, $k_B = 8.617 \times 10^{-5}$
  \ENSURE 
    Temporary intensity dictionary $\mathcal{I} = \{ (T, n_e, \mathbf{I}_{\text{temp}}) \mid \mathbf{I}_{\text{temp}} = \{ (\lambda_{ij}, I_{\text{rel}}, \alpha_G, \alpha_L) \} \}$
  \STATE Initialize $\mathcal{I} \gets \emptyset$ \COMMENT{Temporary intensity dictionary}
  \FORALL{$(T, n_e, \mathcal{A}_{T,n_e}) \in \mathcal{A}$}
    \STATE Extract $\mathbf{R}_{T,n_e}$ from $\mathcal{R}$ \COMMENT{Population ratio}
    \STATE $\mathbf{I}_{\text{temp}} \gets \emptyset$ \COMMENT{Temporary intensity list}
    \FORALL{$S \in \mathcal{A}_{T,n_e}$}
      \STATE Extract $\mathcal{T}_S \subseteq \mathcal{D}$ for $S_{\text{neutral}}$ or $S_{\text{ion}}$ \COMMENT{Transitions}
      \IF{$\mathcal{T}_S = \emptyset$}
        \STATE Log warning: ``No transitions for $S$'' \COMMENT{Skip}
        \STATE continue
      \ENDIF
      \FORALL{$(\lambda_{ij}, E_i, E_k, g_i, g_k, A_{ki}, m_a, w) \in \mathcal{T}_S$}
        \STATE $\Delta E \gets E_k - E_i$ \COMMENT{Energy}
        \STATE $n_{e,\text{min}} \gets 1.6 \times 10^{12} T^{1/2} (\Delta E)^{3/2}$ \COMMENT{Minimum density}
        \IF{$n_e \geq n_{e,\text{min}}$}
          \IF{$S$ is $S_{\text{neutral}}$}
            \STATE $Z \gets Z_{\text{neutral}}$ (Algorithm 2) \COMMENT{Neutral partition}
            \STATE $f \gets f_{\text{neutral}}$ from $\mathbf{R}_{T,n_e}$ \COMMENT{Neutral fraction}
          \ELSE
            \STATE $Z \gets Z_{\text{ion}}$ (Algorithm 2) \COMMENT{Ion partition}
            \STATE $f \gets f_{\text{ion}}$ from $\mathbf{R}_{T,n_e}$ \COMMENT{Ion fraction}
          \ENDIF
          \STATE $I_{\text{rel}} \gets \frac{g_k A_{ki} \exp\left(-\frac{E_k}{k_B T}\right)}{Z} \cdot f$ \COMMENT{Relative intensity}
          \STATE $\alpha_G \gets \frac{\lambda_{ij}}{c} \sqrt{\frac{2 k_B T \ln 2}{m_a}}$ \COMMENT{Gaussian HWHM (Doppler)}
          \STATE $\alpha_L \gets w \cdot \frac{n_e}{10^{16} \sqrt{T}}$ \COMMENT{Lorentzian HWHM}
          \STATE Append $(\lambda_{ij}, I_{\text{rel}}, \alpha_G, \alpha_L)$ to $\mathbf{I}_{\text{temp}}$ \COMMENT{Accumulate}
        \ELSE
          \STATE Log warning: ``LTE condition not satisfied'' \COMMENT{Skip}
        \ENDIF
      \ENDFOR
    \ENDFOR
    \STATE Store $(T, n_e, \mathbf{I}_{\text{temp}})$ in $\mathcal{I}$ \COMMENT{Add to dictionary}
  \ENDFOR
  \STATE \RETURN $\mathcal{I}$ \COMMENT{Result}
\end{algorithmic}
\end{algorithm}

\begin{algorithm}
\small
\caption{Line Profile Generation}
\begin{algorithmic}[1]
  \REQUIRE 
    Temporary intensity dictionary $\mathcal{I} = \{ (T, n_e, \mathbf{I}_{\text{temp}}) \mid \mathbf{I}_{\text{temp}} = \{ (\lambda_{ij}, I_{\text{rel}}, \alpha_G, \alpha_L) \} \}$; \\
    Population ratio dictionary $\mathcal{R} = \{ (T, n_e, \mathbf{R}_{T,n_e}) \mid \mathbf{R}_{T,n_e} \in [0, 1]^2 \}$; \\
    Maximum spectral samples $N \in \mathbb{N}$; \\
    Spectra set $\mathcal{S}$
  \ENSURE 
    $\mathcal{S}$ with $(T, n_e, \mathbf{I}_{T,n_e}, \mathbf{R}_{T,n_e})$
  \STATE Initialize $c \gets 0$ \COMMENT{Counter}
  \FORALL{$(T, n_e, \mathbf{I}_{\text{temp}}) \in \mathcal{I}$}
    \STATE Extract $\mathbf{R}_{T,n_e}$ from $\mathcal{R}$ \COMMENT{Population ratio}
    \STATE $\mathbf{I}_{T,n_e} \gets \emptyset$ \COMMENT{Final intensity list}
    \FORALL{$(\lambda_{ij}, I_{\text{rel}}, \alpha_G, \alpha_L) \in \mathbf{I}_{\text{temp}}$}
      \STATE $I(\lambda) \gets I_{\text{rel}} \cdot V(\lambda - \lambda_{ij}, \alpha_G, \alpha_L)$ \COMMENT{Voigt profile}
      \STATE Append $(\lambda_{ij}, I(\lambda))$ to $\mathbf{I}_{T,n_e}$ \COMMENT{Accumulate}
    \ENDFOR
    \IF{$\max(\{I \mid (\lambda_{ij}, I) \in \mathbf{I}_{T,n_e}\}) > 0$}
      \STATE Normalize $\mathbf{I}_{T,n_e}$: divide each $I$ by $\max(\{I \mid (\lambda_{ij}, I) \in \mathbf{I}_{T,n_e}\})$ \COMMENT{Normalization}
    \ENDIF
    \STATE Store $(T, n_e, \mathbf{I}_{T,n_e}, \mathbf{R}_{T,n_e})$ in $\mathcal{S}$ \COMMENT{Add spectrum}
    \STATE $c \gets c + 1$ \COMMENT{Increment}
    \IF{$c \mod 1000 = 0$} %\COMMENT{Check storage modulo}
      \STATE Save $\mathcal{S}$ \COMMENT{Save}
      \STATE Clear $\mathcal{S}$ \COMMENT{Clear}
    \ENDIF
    \IF{$c \geq N$}
      \STATE break \COMMENT{Terminate}
    \ENDIF
  \ENDFOR
  \STATE \RETURN $\mathcal{S}$ \COMMENT{Result}
\end{algorithmic}
\end{algorithm}