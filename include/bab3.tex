%-------------------------------------------------------------------------------
%                            BAB III
%               		METODE PENELITIAN
%-------------------------------------------------------------------------------


\chapter{METODE PENELITIAN}

Dalam pengembangan model \textit{Long Short-Term Memory} (LSTM), persiapan data yang tepat sangat memengaruhi kinerja model. Bab ini membahas teknik persiapan data numerik dan kategorikal serta penanganan urutan dengan panjang bervariasi.

\section{Waktu dan Lokasi Penelitian}
Penelitian ini dilaksanakan di Laboratorium Gelombang dan Optik, Departemen Fisika, Fakultas Matematika dan Ilmu Pengetahuan Alam, Universitas Syiah Kuala, dari September 2024 hingga Juni 2025.

\section{Jadwal Pelaksanaan Penelitian}
Penelitian ini direncanakan berlangsung selama 8 bulan, dari September 2024 hingga Juni 2025. Tabel~\ref{tab:jadwal_penelitian} menunjukkan jadwal pelaksanaan penelitian yang mencakup fase studi literatur, pengumpulan data, pengembangan model, validasi, dan penulisan laporan akhir.

\begin{table}[H]
  \centering
  \caption{Jadwal Penelitian}
  \label{tab:jadwal_penelitian}
  \small % Mengatur ukuran font tabel agar lebih kecil tetapi tetap terbaca
  \setlength{\tabcolsep}{3pt} % Mengurangi spasi antar kolom untuk fleksibilitas
  \begin{tabular}{p{0.35\textwidth} *{9}{c}} % Kolom pertama 35% dari \textwidth
    \toprule
    \textbf{Fase Penelitian} & \textbf{Sep 2024} & \textbf{Okt 2024} & \textbf{Nov 2024} & \textbf{Des 2024} & \textbf{Jan 2025} & \textbf{Feb 2025} & \textbf{Mar 2025} & \textbf{Apr 2025} & \textbf{Mei 2025} \\
    \midrule
    Studi literatur dan perancangan awal & $\bullet$ & $\bullet$ & $\bullet$ & & & & & & \\
    Pengumpulan dan pengolahan data spektral & & $\bullet$ & $\bullet$ & $\bullet$ & & & & & \\
    Pengembangan dan pengujian model prediksi & & & & $\bullet$ & $\bullet$ & $\bullet$ & $\bullet$ & & \\
    Validasi model dan analisis hasil & & & & & & & & $\bullet$ & $\bullet$ \\
    Penulisan laporan akhir & & & & & & & & & $\bullet$ \\
    \bottomrule
  \end{tabular}
  \vspace{0.2cm}
  \footnotesize
  \textit{Catatan:} Tanda $\bullet$ menunjukkan periode pelaksanaan masing-masing fase.
\end{table}

\section{Alat dan Bahan}
Penelitian ini memanfaatkan berbagai alat dan bahan yang mencakup perangkat keras, perangkat lunak, serta sumber data untuk mendukung proses pengumpulan, pengolahan, analisis, dan visualisasi data spektral atomik. Berikut adalah rincian alat dan bahan yang digunakan:

\subsection{Perangkat Keras}
\begin{enumerate}
  \item \textbf{Laptop \textit{Apple MacBook} Air M1 2020}: Dilengkapi dengan prosesor Apple M1, memori (RAM) sebesar \SI{8}{\giga\byte}, dan penyimpanan internal berbasis SSD. Perangkat ini digunakan untuk persiapan data, eksplorasi awal, pengembangan kode, serta penyusunan laporan penelitian.
  \item \textbf{\textit{Google Colaboratory}}: Lingkungan komputasi berbasis awan yang menyediakan akses ke unit pemrosesan grafis (GPU) NVIDIA Tesla T4 dengan memori \SIrange{15}{20}{\giga\byte}. Platform ini digunakan untuk pelatihan model \textit{machine learning}, evaluasi performa model, serta komputasi intensif lainnya.
\end{enumerate}

\subsection{Perangkat Lunak}
\begin{enumerate}
  \item \textbf{Sistem Operasi}:
  \begin{itemize}
    \item \textit{macOS Ventura} 13.6: Digunakan pada perangkat lokal untuk pengembangan dan pengujian awal.
    \item \textit{Ubuntu}: Digunakan dalam lingkungan virtual \textit{Google Colaboratory} untuk komputasi berbasis awan.
  \end{itemize}
  
  \item \textbf{Bahasa Pemrograman dan Lingkungan Kerja}:
  \begin{itemize}
    \item Python 3.8.17: Bahasa pemrograman utama untuk pengembangan algoritma dan analisis data.
    \item Jupyter Notebook: Digunakan baik secara lokal maupun pada \textit{Google Colaboratory} untuk pengembangan kode interaktif dan dokumentasi analisis.
  \end{itemize}
  
  \item \textbf{Pustaka dan Modul Python}:
  \begin{itemize}
    \item NumPy (versi 1.24.0) dan Pandas (versi 1.5.3): Untuk manipulasi, eksplorasi, dan analisis data numerik serta tabular.
    \item h5py: Untuk pengelolaan dataset dalam format HDF5, termasuk pembacaan dan penyimpanan data.
    \item scikit-learn (versi 1.2.0): Untuk prapemrosesan data, evaluasi model \textit{machine learning}, dan visualisasi dimensi rendah menggunakan algoritma t-SNE (\texttt{sklearn.manifold.TSNE}).
    \item Matplotlib (versi 3.6.2): Untuk pembuatan visualisasi grafik dan representasi hasil analisis.
    \item Joblib (versi 1.2.0): Untuk serialisasi model dan optimalisasi pemrosesan paralel.
    \item PyTorch (versi 1.13.1): Kerangka kerja pembelajaran mendalam untuk pengembangan, pelatihan, dan evaluasi model.
    \item TensorBoard (versi untuk PyTorch): Untuk pemantauan metrik pelatihan dan validasi model secara \textit{real-time}.
    \item itertools: Modul standar Python untuk operasi iterasi kompleks.
  \end{itemize}
\end{enumerate}

\subsection{Sumber Data}
\begin{enumerate}
\item \textbf{\textit{NIST Atomic Spectra Database} (ASD)}: Basis data resmi dari \textit{National Institute of Standards and Technology }(NIST) yang digunakan sebagai sumber utama parameter spektral atomik. Parameter yang diambil meliputi energi ionisasi (\(E_\text{i}\)), energi keadaan (\(E_\text{k}\)), bobot statistik (\(g_\text{i}\), \(g_\text{k}\)), dan koefisien probabilitas transisi (\(A_\text{ki}\)). Data diakses melalui antarmuka daring resmi NIST.\footnote{\url{https://physics.nist.gov/PhysRefData/ASD/}}
\end{enumerate}

\section{Prosedur Penelitian}

Prosedur penelitian ini dirancang untuk mensimulasikan spektrum emisi atom dalam plasma pada kondisi kesetimbangan termal lokal (\textit{Local Thermodynamic Equilibrium}, LTE), dengan tujuan menentukan suhu eksitasi berdasarkan intensitas garis spektral. Pendekatan ini mengintegrasikan perhitungan kuantum mekanis, statistik termodinamika, dan pemodelan \textit{machine learning} untuk analisis data spektral. Diagram alur penelitian (Gambar~\ref{fig:3-diagram}) menggambarkan langkah-langkah utama, yang diuraikan sebagai berikut:

\begin{figure}[H]
    \centering
    \includegraphics[width=1\textwidth]{images/3-Diagram.drawio.pdf}
    \caption{Diagram alur penelitian.}
    \label{fig:3-diagram}
\end{figure}

\begin{algorithm}
\small
\caption{Inisilasi Parameter Simulasi Spektral Atom}
\begin{algorithmic}[1]
  \REQUIRE
    Transition dataset $\mathcal{D} = \{ (\lambda_{ij}, E_i, E_k, g_i, g_k, A_{ki}) \mid \lambda_{ij} \in [300, 800]~\text{nm}, E_i, E_k, g_i, g_k, A_{ki} \in \mathbb{R}^+ \}$ \cite{Kramida2023}, \\
    Number of element-ion pairs $k = 4$, \\
    Maximum number of spectral samples $N \in \mathbb{N}$, \\
    Temperature step $\Delta T \in \mathbb{R}^+$, \\
    Electron density step $\Delta n_e \in \mathbb{R}^+$, \\
    Gaussian broadening parameter $\sigma \in \mathbb{R}^+$, \\
    Wavelength resolution $\Delta \lambda \in \mathbb{R}^+$
  \ENSURE
    Candidate atom set $\mathcal{C}$, wavelength vector $\boldsymbol{\lambda} \in \mathbb{R}^n$, atom subset dictionary $\mathcal{A}$, spectra set $\mathcal{S}$
  \STATE \textbf{Validate Inputs}: Ensure $N > 0$, $\Delta T > 0$, $\Delta n_e > 0$, $\sigma > 0$, $\Delta \lambda > 0$
  \STATE \textbf{Initialize}: $\Delta T \gets 1000~\text{K}$, $\Delta n_e \gets 10^{0.5} \times 10^{12}~\text{cm}^{-3}$, $\sigma \gets 0.1~\text{nm}$, $\Delta \lambda \gets 0.5~\text{nm}$
  \STATE \textbf{Define}: $\mathcal{C} \gets \{\text{H}, \text{He}, \text{O}, \text{N}, \text{Si}, \text{Al}, \text{Fe}, \text{Ca}, \text{Mg}, \text{Na}, \text{Ti}, \text{Mn}, \text{S}, \text{Cl}, \text{Cr}, \text{Ni}, \text{Cu}\}$
  \STATE \textbf{Initialize}: $\boldsymbol{\lambda} \gets$ uniform grid from $300~\text{nm}$ to $800~\text{nm}$ with step $\Delta \lambda$
  \STATE \textbf{Initialize}: $\mathcal{S} \gets \emptyset$, $\mathcal{A} \gets \emptyset$
  \FORALL{$T \in [5000, 15000]~\text{K}$ \textbf{step} $\Delta T$}
    \FORALL{$n_e \in [10^{12}, 10^{16}]~\text{cm}^{-3}$ \textbf{step} $\Delta n_e$}
      \STATE Randomly select $\mathcal{A}_{T,n_e} \subseteq \mathcal{C}$ with $|\mathcal{A}_{T,n_e}| = k$ without replacement
      \IF{no transitions exist in $\mathcal{D}$ for any species in $\mathcal{A}_{T,n_e}$}
        \STATE Log warning: ``No transitions for $\mathcal{A}_{T,n_e}$ at $T$, $n_e$'' \COMMENT{Skip if no transitions}
        \STATE continue
      \ENDIF
      \STATE Store $(T, n_e, \mathcal{A}_{T,n_e})$ in $\mathcal{A}$
    \ENDFOR
  \ENDFOR
  \STATE \RETURN $\mathcal{C}$, $\boldsymbol{\lambda}$, $\mathcal{A}$, $\mathcal{S}$
\end{algorithmic}
\end{algorithm}


\begin{algorithm}
\small
\caption{Ionization and Population Calculations}
\begin{algorithmic}[1]
  \REQUIRE
    Transition dataset $\mathcal{D} = \{ (\lambda_{ij}, E_i, E_k, g_i, g_k, A_{ki}) \mid \lambda_{ij} \in [300, 800]~\text{nm}, E_i, E_k, g_i, g_k, A_{ki} \in \mathbb{R}^+ \}$, \\
    Atom subset dictionary $\mathcal{A} = \{ (T, n_e, \mathcal{A}_{T,n_e}) \mid T \in [5000, 15000]~\text{K}, n_e \in [10^{12}, 10^{16}]~\text{cm}^{-3}, |\mathcal{A}_{T,n_e}| = 4 \}$, \\
    Physical constants: $m_e = 9.109 \times 10^{-31}~\text{kg}$, $k_B = 8.617 \times 10^{-5}~\text{eV/K}$, $h = 4.1357 \times 10^{-15}~\text{eV·s}$
  \ENSURE
    Population ratio dictionary $\mathcal{R} = \{ (T, n_e, \mathbf{R}_{T,n_e}) \mid \mathbf{R}_{T,n_e} \in [0, 1]^2 \}$
  \STATE \textbf{Initialize}: $\mathcal{R} \gets \emptyset$
  \FORALL{$(T, n_e, \mathcal{A}_{T,n_e}) \in \mathcal{A}$}
    \STATE $\mathbf{R}_{T,n_e} \gets \emptyset$
    \FORALL{$S \in \mathcal{A}_{T,n_e}$}
      \STATE Define $(S_{\text{neutral}}, S_{\text{ion}}) \gets (S \text{ I}, S \text{ II})$
      \STATE Extract $\mathcal{T}_S \subseteq \mathcal{D}$ for species $S \text{ I}$ or $S \text{ II}$
      \IF{$\mathcal{T}_S = \emptyset$}
        \STATE Log warning: ``No transitions for species $S$ at $T$, $n_e$'' \COMMENT{Skip IF no transitions}
        \STATE continue
      \ENDIF
      \STATE $Z_{\text{neutral}} \gets \sum_i g_i \exp\left(-\frac{E_i}{k_B T}\right)$
      \IF{$Z_{\text{neutral}} \leq 0$}
        \STATE Log warning: ``Invalid partition function for $S \text{ I}$ at $T$, $n_e$'' \COMMENT{Skip IF invalid}
        \STATE continue
      \ENDIF
      \STATE $Z_{\text{ion}} \gets \sum_i g_i \exp\left(-\frac{E_i}{k_B T}\right)$
      \IF{$Z_{\text{ion}} \leq 0$}
        \STATE Log warning: ``Invalid partition function for $S \text{ II}$ at $T$, $n_e$'' \COMMENT{Skip IF invalid}
        \STATE continue
      \ENDIF
      \STATE $\frac{N_{\text{ion}}}{N_{\text{neutral}}} \gets \frac{2 Z_{\text{ion}}}{n_e Z_{\text{neutral}}} \left( \frac{2\pi m_e k_B T}{h^2} \right)^{3/2} \exp\left(-\frac{E_{\text{ion}}}{k_B T}\right)$
      \STATE $f_{\text{neutral}} \gets \frac{1}{1 + \frac{N_{\text{ion}}}{N_{\text{neutral}}}}$
      \STATE $f_{\text{ion}} \gets \frac{\frac{N_{\text{ion}}}{N_{\text{neutral}}}}{1 + \frac{N_{\text{ion}}}{N_{\text{neutral}}}}$
      \STATE Store $(f_{\text{neutral}}, f_{\text{ion}})$ in $\mathbf{R}_{T,n_e}[S \text{ I}, S \text{ II}]$
    \ENDFOR
    \STATE Store $(T, n_e, \mathbf{R}_{T,n_e})$ in $\mathcal{R}$
  \ENDFOR
  \STATE \RETURN $\mathcal{R}$
\end{algorithmic}
\end{algorithm}