\begin{thebibliography}{99}


\bibitem{corsi2021}
Corsi, M., Cristoforetti, G., Hidalgo, M., Legnaioli, S., Palleschi, V., & Tognoni, E. (2021). Calibration-free laser-induced breakdown spectroscopy for quantitative analysis of alloy samples. \textit{Spectrochimica Acta Part B: Atomic Spectroscopy}, 175, 106020. \url{https://doi.org/10.1016/j.sab.2020.106020}

\bibitem{harrison2021}
Harrison, R. G., & Smith, P. J. (2021). Recent advances in laser-induced breakdown spectroscopy for environmental and industrial applications. \textit{Applied Spectroscopy Reviews}, 56(6), 437-472. \url{https://doi.org/10.1080/05704928.2020.1819834}

\bibitem{jolliffe2002}
Jolliffe, I. T. (2002). \textit{Principal component analysis} (2nd ed.). Springer.

\bibitem{jeong2020}
Jeong, S., Kim, H., & Lee, K. (2020). Enhanced analysis of metal alloys using deep learning-assisted laser-induced breakdown spectroscopy. \textit{Sensors}, 20(8), 2362. \url{https://doi.org/10.3390/s20082362}

\bibitem{lee2021}
Lee, J., & Oh, J. (2021). Application of principal component analysis for the classification and interpretation of laser-induced breakdown spectroscopy data. \textit{Journal of Analytical Atomic Spectrometry}, 36(1), 80-89. \url{https://doi.org/10.1039/D0JA00373K}

\bibitem{martin2020}
Martin, M., & Harrison, R. (2020). Advances in chemometric analysis for LIBS. \textit{Journal of Chemometrics}, 34(11), e3297. \url{https://doi.org/10.1002/cem.3297}

\bibitem{morris2020}
Morris, G. L., & Martin, A. (2020). Fundamentals of Laser-Induced Breakdown Spectroscopy. In M. A. Hughes (Ed.), \textit{Laser-induced breakdown spectroscopy: Theory and applications} (pp. 25-50). Springer. \url{https://doi.org/10.1007/978-3-030-33029-7_2}

\bibitem{smith2020}
Smith, J. A. (2020). Development of quantitative analysis methods for laser-induced breakdown spectroscopy. \textit{Analytical Chemistry}, 92(8), 5855-5862. \url{https://doi.org/10.1021/acs.analchem.9b05434}

\bibitem{voigt1929}
Voigt, W. (1929). Ueber das Gesetz der Intensitätsverteilung innerhalb der Linien eines Gasspektrums. \textit{Annalen der Physik}, 394(5), 273-330. \url{https://doi.org/10.1002/andp.19293940504}

\end{thebibliography}