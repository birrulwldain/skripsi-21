\begin{abstractind}

Spektrum emisi merupakan cahaya yang terpancar dari elektron yang bertransisi antar tingkat energi atom atau ion. Struktur tingkat energi pada masing masing atom atau ion membentuk pola unik yang memungkinkan analisis komposisi material secara presisi dalam metode \textit{Laser-Induced Breakdown Spectroscopy} (LIBS). Namun akurasi metode ini secara konvensional seringkali terhambat oleh kompleksitas spektrum emisi multi-elemen, efek matriks dan fluktuasi plasma. Penelitian ini mengembangkan pendekatan analisis berbasis \textit{Deep Learning} (DL) yang mempelajari Spektrum Sintetis tersimulasi. Spektrum sintetis disimulasikan dengan Persamaan Saha dan Distribusi Boltzmann yang kemudian di interpolasikan oleh profil garis gaussian dengan asumsi dalam keadaan \textit{Local Thermal Equilibrium} (LTE) dengan menggunakan data transisi atom dari \textit{Atomic Spectra Database} (ASD) oleh \textit{National Institute Standards and Technology} (NIST) LIBS. Model Informer merupakan arsitektur Transformer yang dioptimalkan dengan \textit{ProbSparse Self-Attention, Self-Attention Distilling,} dan \textit{Generative-style Decoder} dipilih untuk mengatasi spektrum emisi yang relatif panjang. Model dilatih dan diukur menggunakan metrik \textit{Mean Squared Error} (MSE)  untuk mencapai akurasi tinggi dengan diversitas spektrum sintetis. Pendekatan ini berpotensi meningkatkan akurasi deteksi multi-elemen dalam metode LIBS, serta berkontribusi bagi kemajuan spektroskopi modern.

\bigskip
\noindent
\textbf{Kata kunci :} LIBS, Informer, Spektrum Sintetis, Pembelajaran Mendalam, Analisis Prediktif, ProbSparse Self-Attention.
\end{abstractind}

