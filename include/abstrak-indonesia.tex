\begin{abstractind}

Artificial Neural Networks (ANN) bring important insights to the development of Laser-induced Breakdown Spectroscopy methods in qualitatively identifying spectra, especially soil spectra. Volcanic soils were highly fertile which played a major role in the development of the agrarian industry in Indonesia. Thus, the need for rapid identification techniques to analyze major elements in volcanic soil is the answer to the massive volcanic soil in Indonesia. Long Short-Term Memory (LSTM) was an enhanced variant of conventional Recurrent Neural Network (RNN). In this research, the LSTM model was trained with simulated pure element spectra. These simulated pure elemental spectra were built from various elements with plasma state variance. The model was also built to consider the matrix effect present in experiment. The model was then fine-tuned with spectra of Mount Seulawah Agam volcanic soil taken from three sub-districts: Seulimum, Cot Glie, and Lembah Seulawah with each at a depth of 20cm, 40cm, and 60cm from the ground surface. The LSTM model then successfully predicted several major elements such as Calcium, Aluminium, Iron, Magnesium, and Silica accurately and compared with the X-ray fluorescence (XRF) analysis method technique. This research makes an important contribution to the use of ANN for geochemical analysis of volcanic activity in Indonesia for further utilization.

\bigskip
\noindent
\textbf{Kata kunci :} lorem ipsum dolor sit amet
\end{abstractind}