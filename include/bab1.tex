




\chapter{PENDAHULUAN}



\section{Latar Belakang}


\par  Spektroskopi secara sederhana merupakan  interaksi antara materi dan radiasi elektromagnetik yang dapat memberi informasi bagaimana materi menyerap, memancarkan, atau menghamburkan cahaya. Kemudian interaksi ini  digunakan untuk mengidentifikasi dan mengukur komposisi, struktur, dan sifat materi tersebut \parencite{hollas2013}.  
Sejak penemuan garis spektral oleh Fraunhofer dan Kirchhoff, spektroskopi telah menjadi dasar analisis kimia modern, termasuk dalam \textit{Laser-Induced Breakdown Spectroscopy} (LIBS) \parencite{thorne1999}. LIBS menggunakan laser berenergi tinggi untuk menciptakan plasma pada permukaan sampel, menghasilkan spektrum emisi yang kaya informasi. Teknik ini menawarkan analisis cepat dan serbaguna untuk aplikasi seperti identifikasi material dan analisis lingkungan \parencite{thorne1999}.
\begin{figure}
    \centering
    \includegraphics[width=0.8\textwidth]{images/1-latar.jpg}
    \caption{Spekrum mentah (a) pra-pemrosesan (b) spektrum Hidrogen sebelum diberi \textit{noise} (c) dan sesudah diberi \textit{noise} (d) \citep{Gasior2023}}
    \label{fig:kenapa}
\end{figure}

\par \textit{Laser-Induced Breakdown Spectroscopy} (LIBS) adalah teknik analisis yang melibatkan penggunaan laser berenergi tinggi yang kemudian difokuskan ke permukaan sampel untuk menciptakan plasma. LIBS menunjukkan potensinya sebagai teknik analisis yang cepat dan serbaguna untuk berbagai aplikasi, termasuk analisis material. Namun, kompleksitas spektrum LIBS, yang dipengaruhi oleh efek matriks dan fluktuasi plasma, menyulitkan analisis kualitatif dan kuantitatif seperti gambar \ref{fig:kenapa} \parencite{gasior2023analysis}. Metode tradisional, seperti perhitungan numerik berdasarkan persamaan Saha, memerlukan komputasi intensif dan kurang efektif dalam menangani spektrum multi-elemen dengan noise tinggi. Pendekatan pembelajaran mesin, seperti jaringan saraf tiruan (JST) dan pembelajaran mendalam, menunjukkan potensi untuk mengatasi tantangan ini, tetapi memerlukan dataset besar yang mahal untuk dilatih \parencite{gasior2023analysis}. Simulasi spektrum elemen murni dengan profil \textit{Gaussian}, berdasarkan distribusi Boltzmann dan persamaan Saha, dapat menghasilkan dataset sintetis untuk melatih model secara efisien \parencite{pan-2024}. Distribusi Boltzmann memungkinkan perhitungan distribusi energi atom pada suhu tertentu, sementara persamaan Saha menghubungkan fraksi ionisasi dengan suhu dan densitas elektron, membentuk dasar simulasi spektrum LIBS \parencite{thorne1999}.

\par Algoritma Transformer, yang diperkenalkan oleh \cite{vaswani-2017}, unggul dalam memproses data sekuens secara paralel melalui mekanisme \textit{self-attention}. Transformer juga telah berhasil diterapkan dalam dekonvolusi spektrum inframerah dengan \textit{sparse self-attention} oleh \cite{gao2024}, namun aplikasinya untuk simulasi spektrum LIBS berbasis model fisika masih terbatas. Penelitian ini mengusulkan penggunaan Transformer untuk mempelajari spektrum atom yang disimulasikan, dengan memanfaatkan distribusi Boltzmann dan persamaan Saha, guna meningkatkan akurasi prediksi spektrum multi-elemen.


\section{Rumusan Masalah}
\par Distribusi Boltzmann menjelaskan distribusi energi atom dalam sistem pada suhu konstan, sementara persamaan Saha menggambarkan kesetimbangan ionisasi dalam plasma atau gas. Algoritma Transformer, yang diperkenalkan oleh \cite{vaswani-2017}, unggul dalam memproses data sekuens melalui mekanisme \textit{self-attention} dengan kompleksitas \( O(n \log n) \) pada varian \textit{sparse} \parencite{gao2024}. Namun, kompleksitas spektrum LIBS menyulitkan pemodelan komputasi yang efisien. Oleh karena itu, rumusan masalah penelitian ini adalah:
\begin{enumerate}
    \item Sejauh mana model Informer dengan mekanisme \textit{ProbSparse self-attention} dapat meningkatkan akurasi prediksi spektrum multi-elemen  dalam simulasi LIBS, diukur dengan \textit{mean squared error} (MSE)?
    \item Bagaimana efisiensi komputasi model Transformer dalam mempelajari spektrum atom yang disimulasikan pada berbagai suhu plasma dan densitas elektron, diukur dengan waktu komputasi?
    \item Apakah jumlah dan variasi spektrum sintetis memengaruhi kemampuan generalisasi model Informer untuk prediksi multi-elemen?
\end{enumerate}


\section{Tujuan Penelitian}
Berdasarkan rumusan masalah, tujuan penelitian ini adalah:
\begin{enumerate}

    \item Mengembangkan model Transformer dengan mekanisme \textit{ProbSparse self-attention} untuk memprediksi spektrum multi-elemen berdasarkan simulasi spektrum atom pada berbagai suhu plasma dan densitas elektron.
    \item Mengukur akurasi prediksi model Transformer menggunakan \textit{mean squared error} (MSE).
    \item Mengevaluasi pengaruh jumlah dan variasi spektrum sintetis terhadap kemampuan generalisasi model Transformer dalam prediksi spektrum multi-elemen.
\end{enumerate}
\section{Batasan Penelitian}
Penelitian ini berfokus pada simulasi spektrum atom untuk elemen terpilih pada varian suhu plasma dan densitas elektron, menggunakan distribusi Boltzmann dan persamaan Saha. Penelitian tidak mencakup pengukuran spektrum LIBS eksperimental atau analisis efek matriks dan fluktuasi plasma, untuk menjaga fokus pada evaluasi model Transformer.

\section{Manfaat Penelitian}
Penelitian ini diharapkan memberikan pendekatan baru dalam analisis LIBS melalui prediksi spektrum multi-elemen menggunakan algoritma Transformer. Model yang dikembangkan dapat meningkatkan akurasi deteksi elemen mayor dan minor.
