% \fancyhf{} 
% \fancyfoot[R]{\thepage}

\chapter{PENDAHULUAN}
%\thispagestyle{plain} % Halaman pertama bab menggunakan gaya plain

\section{Latar Belakang}
% Menambahkan lorem ipsum


\par  Spektroskopi secara sederhana merupakan  interaksi antara materi dan radiasi elektromagnetik yang dapat memberi informasi bagaimana materi menyerap, memancarkan, atau menghamburkan cahaya. Kemudian interaksi ini  digunakan untuk mengidentifikasi dan mengukur komposisi, struktur, dan sifat materi tersebut \citep{hollas-2013}.  Joseph von Fraunhofer  (1778 - 1826) adalah ilmuan yang menemukan garis spektral hitam pada spektrum matahari dari penelitiannya dalam mencari indeks bias dari cahaya monokromatik. Hingga sekitar 50 tahun kemudian, Gustav Robert Kirchhoff (1824-1887) dan Robert Bunsen (1811-1899) menemukan bahwa setiap zat kimia memiliki karakteristiknya masing-masing berupa garis spektrum dan menjadi dasar untuk analisis spektrokimia. Upaya berlanjut hingga pada tahun 1885, Johann Jakob Balmer (1825-1898) memprediksi panjang gelombang spektrum Hidrogen. Perkembangan terus berlanjut hingga dasar teori laser diusulkan oleh Schawlow dan Townes pada tahun 1958 dan tercatat eksperimen pertama laser menggunakan kristal ruby pada tahun 1960 oleh Maiman. Dua tahun kemudian laser ruby digunakan untuk menghasilkan plasma pada sampel dan mengamati spektrum emisi oleh Brech dan Cross \citep{thorne-1999}.
\par\textit{Laser-Induced Breakdown Spectroscopy} (LIBS) adalah teknik analisis yang melibatkan penggunaan laser berenergi tinggi yang kemudian difokuskan ke permukaan sampel untuk menciptakan plasma. LIBS menunjukkan potensinya sebagai teknik analisis yang cepat dan serbaguna untuk berbagai aplikasi, termasuk analisis material, pemantauan lingkungan, dan analisis makanan. LIBS telah menjadi teknik yang menarik dan populer di bidang analisis kimia karena keunggulannya yang unik, seperti aplikasinya pada cairan, gas, dan padatan, tidak ada perlakuan awal sampel, deteksi simultan beberapa elemen, dan deteksi jarak jauh tanpa kontak di berbagai bidang, termasuk pembersihan laser, perlindungan lingkungan, eksplorasi ruang angkasa, dan pelestarian warisan budaya. Namun kompleksitas spektrum yang dihasilkan dan pengaruh faktor-faktor seperti efek matriks dan fluktuasi plasma menjadi tantangan dalam analisis kualitatif maupun kuantitatif pada LIBS \citep{gasior2023analysis}. \textit{Artificial Neural Network} (ANN) diusulkan untuk  memberikan metode yang layak, cepat, dan kuat pada analisis LIBS. Hasil penelitian D'Andrea berhasil memprediksi konsentrasi unsur-unsur dalam sampel perunggu dengan akurasi yang baik \citep{dandrea-2014}.
\par Tanah vulkanis, yang dikenal sebagai Andosol, memiliki karakteristik unik yang menjadikannya sangat subur. Di Indonesia, tanah ini umumnya terbentuk dari material vulkanik yang kaya akan mineral dan memiliki kapasitas retensi air yang tinggi. Penelitian menunjukkan bahwa penggunaan lahan yang intensif, seperti pertanian sayuran, dapat menyebabkan penurunan pH dan kehilangan bahan organik, yang berdampak negatif pada kesuburan tanah. Namun, praktik pertanian organik dapat membantu memulihkan kesuburan tanah dengan meningkatkan kandungan karbon organik dan nutrisi lainnya \citep{fiantis2020land}. Kompleksitas dari elemen dalam tanah vulkanis menjadi tantangan untuk dianalisa. Oleh karenanya dibutuhkan algoritma ANN yang mampu memprediksi multi elemen secara \textit{real-time}. Pada penelitian ini, model \textit{Deep Learning} \textit{Long Shor-Term Memory} (LSTM) diusulkan untuk mempelajari spektrum berbagai elemen. Agar model LSTM dapat mempelajari spektrum, dibutuhkan spektrum elemen murni untuk menjadi \textit{dataset} yang cukup besar. Untuk menjawab permasalahan ini, simulasi spektrum murni dilakukan dengan menghitung intensitas relatif dari tiap elemen. Profil garis \textit{Gaussian} digunakan untuk melengkapi simulasi spektrum murni. Spektrum Elemen murni akan disimulasikan untuk berbagai keadaan suhu plasma dan memberikan rasio intensitas yang cukup untuk dipelajari model LSTM.


\section{Rumusan Masalah}
\par \cite{fiantis2020land} menyatakan bahwa karakteristik tanah vulkanis unik dan menjadikannya sangat subu. Namun, analisa multi elemen pada tanah yang memiliki kompleksitas tinggi sulit untuk di analisa dalam waktu singkat dan akurat. Oleh karena itu muncul beberapa pertanyaan terkait metode alternatif untuk mempercepat analisa LIBS secara \textit{real-ti} dan efisien; bagaimana algoritma LSTM dapat memprediksi multi elemen pada tanah vulkanis dan keakuratannya; apakah ada spektrum elemen yang bisa diidentifikasi oleh LSTM berbeda dengan identifikasi manual.
%	\item lorem ipsum dolor sit amet, consectetur adipiscing elit. Sed euismod, nisl quis lacinia ultricies,
%	\item lorem ipsum dolor sit amet, consectetur adipiscing elit. Sed euismod, nisl quis lacinia ultricies,
%	\item lorem ipsum dolor sit amet, consectetur adipiscing elit. Sed euismod, nisl quis lacinia ultricies,
%	\item lorem ipsum dolor sit amet, consectetur adipiscing elit. Sed euismod, nisl quis lacinia ultricies,
%\end{enumerate}

\section{Tujuan Penelitian}
\par Berdasarkan latar belakang dan rumusan masalah diatas, tujuan dari penelitian ini adalah untuk menguji metode LIBS berbasis LSTM dalam memprediksi multi elemen pada tanah vulkanik.

\section{Manfaat Penelitian}
\par Dengan dibuktikan nya algoritma LSTM dalam memprediksi multi elemen pada spektrum LIBS akan membawa pendekatan baru pada deteksi elemen mayor dan minor pada metode LIBS. Kemudian juga akan mendapatkan validasi dan hipotesa awal pada spektrum LIBS apa saja.



% Baris ini digunakan untuk membantu dalam melakukan sitasi
% Karena diapit dengan comment, maka baris ini akan diabaikan
% oleh compiler LaTeX.
\begin{comment}
\bibliography{daftar-pustaka}
\end{comment}
