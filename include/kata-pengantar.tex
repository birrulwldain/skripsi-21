\preface % Note: \preface JANGAN DIHAPUS!


Segala puji dan syukur kehadiran Allah SWT yang telah melimpahkan rahmat dan hidayah-Nya kepada kita semua, sehingga penulis dapat menyelesaikan penulisan Tugas Akhir yang berjudul \textbf{{\MakeUppercase{Identifikasi Multi Elemen pada Spektrum Emisi LIBS Kompleks Dari Tanah Vulkanik Seulawah Agam Menggunakan Algoritma \textit{Long Short-Term Memory} (LSTM)}}} yang telah dapat diselesaikan sesuai rencana. Penulis banyak mendapatkan berbagai pengarahan, bimbingan, dan bantuan dari berbagai pihak. Oleh karena itu, melalui tulisan ini penulis mengucapkan rasa terima kasih kepada:

\begin{enumerate} 

	\item {Bapak Dr. Saumi Syahreza, S.Si., M.Si. selaku Ketua Departemen Fisika Fakultas MIPA Universitas Syiah Kuala.}

	\item{Bapak Prof. Dr. Eng. Nasrullah, S.Si., M.T.
 selaku Dosen Pembimbing I yang telah banyak memberikan bimbingan dan arahan kepada penulis, sehingga penulis dapat menyelesaikan Tugas Akhir ini.}
	\item{Bapak Dr. Khairun Saddami, S.T.
 selaku Dosen Pembimbing II yang telah banyak memberikan bimbingan dan arahan kepada penulis, sehingga penulis dapat menyelesaikan Tugas Akhir ini.}

	\item{Bapak Dr. Kurnia Lahna, M.T. selaku Dosen Wali yang telah membimbing
dan memberikan motivasi kepada penulis selama masa perkuliahan.}
  	\item{Ayah dan Ibu sebagai kedua orang tua penulis yang senantiasa selalu mendukung aktivitas dan kegiatan yang penulis lakukan baik secara moral maupun material serta menjadi motivasi terbesar bagi penulis untuk menyelesaikan Tugas Akhir ini.}
	\item{Seluruh Dosen di Departemen Fisika Fakultas MIPA atas ilmu dan didikannya selama perkuliahan.}
	\item{Sahabat dan teman-teman seperjuangan Departemen Fisika USK 2020 lainnya.}
\end{enumerate}

%\vspace{1.5cm}

Penulis juga menyadari segala ketidaksempurnaan yang terdapat didalamnya baik dari segi materi, cara, ataupun bahasa yang disajikan. Seiring dengan ini penulis mengharapkan kritik dan saran dari pembaca yang sifatnya dapat berguna untuk kesempurnaan Tugas Akhir ini. Harapan penulis semoga tulisan ini dapat bermanfaat bagi banyak pihak dan untuk perkembangan ilmu pengetahuan.

\vspace{1cm}


\begin{tabular}{p{7.5cm}l}
	&Banda Aceh, 14 Agustus 2024\\
	&\\
	&\\
	&\multirow{1.5}{7.5cm}{\underline{Birrul Walidain}} \\ 
	&NPM. 2008102010010 \\
\end{tabular}
