\preface % Note: \preface JANGAN DIHAPUS!

Syukur Alhamdulillah dipanjatkan ke hadirat Allah SWT yang telah m membimbingkan rahmat-Nya sehingga Tugas Akhir/Tesis yang berjudul \textbf{\MakeUppercase{Analisis Prediktif Spektrum Emisi \textit{Laser-Induced Breakdown Spectroscopy} (LIBS) Multi-Elemen Berbasis Simulasi Sintetis dengan Informer}} yang telah dapat diselesaikan sesuai dapat diselesaikan. Selawat dan salam disanjungkan kepada Nabi Besar Muhammad SAW.

Tugas Akhir/Tesis ini merupakan salah satu syarat yang harus dipenuhi untuk memperoleh gelar Sarjana/Magister di Departemen Fisika, Fakultas Matematika dan Ilmu Pengetahuan Alam, Universitas Syiah Kuala. Penyelesaian Tugas Akhir/Tesis ini tidak terlepas dari bantuan dan dorongan dari berbagai pihak, baik secara moril maupun materi. Pada kesempatan ini, ucapan terima kasih diucapkan kepada:

\begin{enumerate}
    \item Bapak Dr. Saumi Syahreza, S.Si., M.Si. selaku Ketua Departemen Fisika Fakultas MIPA Universitas Syiah Kuala.
    \item Bapak Prof. Dr. Eng. Nasrullah, S.Si., M.T. selaku Dosen Pembimbing I yang telah banyak memberikan bimbingan dan arahan kepada penulis, sehingga penulis dapat menyelesaikan Tugas Akhir ini.
    \item Bapak Dr. Khairun Saddami, S.T. selaku Dosen Pembimbing II yang telah banyak memberikan bimbingan dan arahan kepada penulis, sehingga penulis dapat menyelesaikan Tugas Akhir ini.
    \item Bapak Dr. Kurnia Lahna, M.T. selaku Dosen Wali yang telah membimbing dan memberikan motivasi kepada penulis selama masa perkuliahan.
    \item Ibu Syamsiah Rasyid yang telah mendukung penulis dalam penyelesaian studi di Fakultas Matematika dan Ilmu Pengetahuan Alam, Universitas Syiah Kuala.
\end{enumerate}
%\vspace{1.5cm}

Penulis juga menyadari segala ketidaksempurnaan yang terdapat didalamnya baik dari segi materi, cara, ataupun bahasa yang disajikan. Seiring dengan ini penulis mengharapkan kritik dan saran dari pembaca yang sifatnya dapat berguna untuk kesempurnaan Tugas Akhir ini. Harapan penulis semoga tulisan ini dapat bermanfaat bagi banyak pihak dan untuk perkembangan ilmu pengetahuan.

\vspace{1cm}


\begin{tabular}{p{7.5cm}l}
	&Banda Aceh, 27 Mei 2025\\
	&\\
	&\\
	&\multirow{1.5}{7.5cm}{\underline{Birrul Walidain}} \\ 
	&NPM. 2008102010010 \\
\end{tabular}
