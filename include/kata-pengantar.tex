\preface % Note: \preface JANGAN DIHAPUS!
\makeatletter
Syukur Alhamdulillah dipanjatkan ke hadirat Allah SWT yang telah membimbingkan rahmat-Nya sehingga Tugas Akhir yang berjudul "\textit{\@judul}" yang telah dapat diselesaikan sesuai dapat diselesaikan. Selawat dan salam disanjungkan kepada Nabi Besar Muhammad SAW.

Tugas Akhir ini merupakan salah satu syarat yang harus dipenuhi untuk memperoleh gelar Sarjana/Magister di Departemen Fisika, Fakultas Matematika dan Ilmu Pengetahuan Alam, Universitas Syiah Kuala. Penyelesaian Tugas Akhir ini tidak terlepas dari bantuan dan dorongan dari berbagai pihak, baik secara moril maupun materi. Pada kesempatan ini, ucapan terima kasih diucapkan kepada:

\begin{enumerate}
    \item Bapak \@dekan \textit{ }selaku Dekan Fakultas Matematika dan Ilmu Pengetahuan Alam Universitas Syiah Kuala
    \item Bapak \@kaprodi \textit{ }selaku Ketua Departemen Fisika Fakultas MIPA Universitas Syiah Kuala.
    \item Bapak \@firstsupervisor \textit{ }selaku  Pembimbing II yang telah banyak memberikan bimbingan dan arahan kepada penulis, sehingga penulis dapat menyelesaikan Tugas Akhir ini.
    \item Bapak \@secondsupervisor   \textit{ }selaku  Pembimbing II yang telah banyak memberikan bimbingan dan arahan kepada penulis, sehingga penulis dapat menyelesaikan Tugas Akhir ini.
    \item Ibu Dr. Gunawati, S.Si., M.Si. dan Ibu Dr. Rara Mitaphonna, S.Si. selaku Tim Penguji yang telah memberikan masukan dan saran kepada penulis dalam penyelesaian Tugas Akhir/Tesis ini.
    \item Bapak Dr. Kurnia Lahna, M.T. selaku Dosen Wali yang telah membimbing dan memberikan motivasi kepada penulis selama masa perkuliahan.
    \item Ibu Syamsiah Rasyid yang telah mendukung penulis dalam penyelesaian studi di Fakultas Matematika dan Ilmu Pengetahuan Alam, Universitas Syiah Kuala.
    \item Seluruh dosen, laboran, dan staff administrasi Departemen Fisika Fakultas Matematika dan Ilmu Pengetahuan Alam Universitas Syiah Kuala yang telah memberikan ilmu, waktu untuk diskusi, arahan, saran, dan bantuan dalam permasalahan akademik sejak awal dari perkuliahan dimulai.
    \item Seluruh teman-teman mahasiswa Program Studi Sarjana Fisika Fakultas Fisika dan Ilmu Pengetahuan Alam angkatan 2020 yang senantiasa menyemangati dan menemani penulis untuk menyelesaikan Tugas Akhir ini.
\end{enumerate}
%\vspace{1.5cm}

Penulis juga menyadari segala ketidaksempurnaan yang terdapat didalamnya baik dari segi materi, cara, ataupun bahasa yang disajikan. Seiring dengan ini penulis mengharapkan kritik dan saran dari pembaca yang sifatnya dapat berguna untuk kesempurnaan Tugas Akhir ini. Harapan penulis semoga tulisan ini dapat bermanfaat bagi banyak pihak dan untuk perkembangan ilmu pengetahuan.

\vspace{1cm}


\begin{tabular}{p{7.5cm}l}
	&Banda Aceh, 27 Mei 2025\\
	&\\
	&\\
	&\multirow{1.5}{7.5cm}{\underline{\@fullname}} \\ 
	&NPM. \@idnum \\
\end{tabular}
\makeatother