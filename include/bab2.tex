%-------------------------------------------------------------------------------
%                            BAB II
%               TINJAUAN PUSTAKA DAN DASAR TEORI
%-------------------------------------------------------------------------------

\chapter{TINJAUAN PUSTAKA}

\section{Transisi Elektronik dalam Spektra Atom}
\subsection{\textit{Planck's Law} dalam Emisi Foton dari Transisi Atom}
Hukum Planck menghubungkan energi foton dengan frekuensinya melalui persamaan 
\begin{equation}
    E = h\nu ,\label{planck}
\end{equation}
dimana $h = 4.1357 \times 10^{-15} \, \text{eV·s}$ adalah konstanta Planck dan $\nu$ adalah frekuensi \citep{Beiser1992}. Hukum ini menjadi dasar emisi foton dalam transisi elektronik atom, yang diukur melalui spektroskopi untuk memverifikasi tingkat energi.

\subsection{Tingkat Energi Elektron dalam Spektra Atom Hidrogen}
Dalam atom hidrogen, energi elektron pada tingkat kuantum $n$ diberikan oleh:
\begin{equation}
E_n = -\frac{13.6 \, \text{eV}}{n^2} \quad (n \in \mathbb{Z}^+), \label{energy_hydrogen}
\end{equation}
dengan selisih energi antar tingkat 

\begin{equation}
\Delta E = E_{n_f} - E_{n_i} = -13.6 \left( \frac{1}{n_f^2} - \frac{1}{n_i^2} \right) \, \text{eV} \quad (n_f < n_i), \label{deltaE}
\end{equation} 
Energi ini menentukan panjang gelombang foton yang diukur dalam eksperimen spektroskopi \citep{Griffiths2005}.

\subsection{Diagram Grotrian untuk Identifikasi Garis Spektral Hidrogen dan Helium}
Diagram Grotrian memvisualisasikan tingkat energi dan transisi yang diizinkan, seperti deret Lyman ($n_f = 1$) dan Balmer ($n_f = 2$) pada hidrogen. Diagram ini digunakan untuk mengidentifikasi garis spektral dalam eksperimen, membantu analisis spektra hidrogen dan helium \citep{Mason2015}.

\subsection{Aturan Seleksi dan Efek Spin dalam Spektra Atom Alkali}
Transisi elektronik terjadi ketika elektron berpindah antar tingkat energi, dengan probabilitas ditentukan oleh elemen matriks dipol $P \propto |\langle \psi_f | \hat{\mu} | \psi_i \rangle|^2$ \citep{Demtroder2010}. Aturan seleksi meliputi $\Delta l = \pm 1$ dan $\Delta m_l = 0, \pm 1$, dengan aturan spin $\Delta S = 0$ mencegah transisi singlet-triplet kecuali ada kopling spin-orbit \citep{Griffiths2005}. Ini diuji pada atom alkali dalam spektroskopi emisi.

\section{Statistik Populasi Elektron dalam Keseimbangan Termal}

\subsection{Distribusi Boltzmann untuk Menentukan Suhu Eksitasi di Plasma}

Populasi relatif pada berbagai tingkat energi atom bergantung pada proses kompleks populasi dan depopulasi. Namun, dalam kesetimbangan termal, populasi hanya ditentukan oleh suhu \( T \) sistem. Populasi atom pada tingkat energi ke-\( i \), yaitu \( N_i \), sebanding dengan:
\begin{equation}
N_i \propto g_i e^{-\frac{E_i}{k_B T}}, \label{eq:boltzmann_propto}
\end{equation}
di mana \( g_i \) adalah faktor degenerasi, \( E_i \) adalah energi tingkat ke-\( i \), dan \( k_B = 8.617 \times 10^{-5} \, \text{eV/K} \) adalah konstanta Boltzmann. Distribusi Boltzmann secara lengkap diberikan oleh:
\begin{equation}
N_i = \frac{N g_i e^{-\frac{E_i}{k_B T}}}{Z}, \label{eq:boltzmann}
\end{equation}
di mana \( N \) adalah total populasi atom, dan \( Z \) adalah fungsi partisi.

Fungsi partisi \( Z \) didefinisikan sebagai jumlah kontribusi semua tingkat energi atom yang mungkin:
\begin{equation}
Z = \sum_i g_i e^{-\frac{E_i}{k_B T}}, \label{eq:partition}
\end{equation}
dengan indeks \( i \) merujuk pada semua tingkat energi relevan, seperti keadaan dengan bilangan kuantum utama \( n \) untuk atom hidrogen. Fungsi partisi memastikan normalisasi sehingga \( \sum_i N_i = N \). Fraksi populasi relatif pada tingkat ke-\( i \) adalah:
\begin{equation}
\frac{N_i}{N} = \frac{g_i e^{-\frac{E_i}{k_B T}}}{Z}. \label{eq:rel_population}
\end{equation}
Persamaan ini bergantung pada suhu \( T \) dan sifat atom, seperti energi \( E_i \) dan degenerasi \( g_i \), yang ditentukan oleh struktur kuantum atom \citep{Pathria2011,rybicki-1985}.

\subsection{Rasio Intensitas Garis Spektral sebagai Indikator Suhu}

Intensitas garis spektral \( I_{ij} \) dihasilkan dari emisi spontan saat atom bertransisi dari tingkat energi atas \( E_i \) ke tingkat energi bawah \( E_j \) (dengan \( E_i > E_j \)). Probabilitas emisi spontan per satuan waktu diberikan oleh koefisien Einstein \( A_{ij} \). Daya total yang dihasilkan adalah:
\begin{equation}
P_{ij} = N_i A_{ij} h \nu_{ij}, \label{eq:power}
\end{equation}
di mana \( N_i \) adalah populasi atom pada tingkat \( E_i \), dan \( h \nu_{ij} = E_i - E_j \) adalah energi foton yang dipancarkan. Intensitas \( I_{ij} \), yaitu daya per satuan sudut ruang, diberikan oleh:
\begin{equation}
I_{ij} = \frac{P_{ij}}{4\pi} = \frac{N_i A_{ij} h \nu_{ij}}{4\pi}. \label{eq:intensity}
\end{equation}

Dalam plasma pada kesetimbangan termal, populasi \( N_i \) dihitung menggunakan distribusi Boltzmann pada Persamaan~\eqref{eq:boltzmann}. Substitusi \( N_i \) ke dalam Persamaan~\eqref{eq:intensity} menghasilkan:
\begin{equation}
I_{ij} = \frac{N g_i A_{ij} h \nu_{ij} e^{-\frac{E_i}{k_B T}}}{4\pi Z}. \label{eq:intensity_full}
\end{equation}
Dengan \( h \nu_{ij} = \frac{h c}{\lambda_{ij}} \) dan \( h c \approx 1239.84 \, \text{eV·nm} \), intensitas relatif sering dinormalisasi untuk menghilangkan konstanta seperti \( N \) dan \( h c \), sehingga:
\begin{equation}
I_{ij} \propto \frac{g_i A_{ij} e^{-\frac{E_i}{k_B T}}}{\lambda_{ij} Z}. \label{eq:intensity_relative}
\end{equation}

Rasio intensitas dua garis spektral, misalnya \( I_{ij} \) (transisi \( i \to j \)) dan \( I_{mn} \) (transisi \( m \to n \)), digunakan untuk menentukan suhu plasma \citep{Mason2015}. Intensitas untuk transisi \( m \to n \) adalah:
\begin{equation}
I_{mn} = \frac{N g_m A_{mn} h \nu_{mn} e^{-\frac{E_m}{k_B T}}}{4\pi Z}, \label{eq:intensity_mn}
\end{equation}
dengan \( h \nu_{mn} = \frac{h c}{\lambda_{mn}} \). Rasio intensitas diberikan oleh:
\begin{equation}
\frac{I_{ij}}{I_{mn}} = \frac{g_i A_{ij} (E_i - E_j)}{g_m A_{mn} (E_m - E_n)} e^{-\frac{E_i - E_m}{k_B T}}. \label{eq:intensity_ratio}
\end{equation}
Substitusi \( E_i - E_j = \frac{h c}{\lambda_{ij}} \) dan \( E_m - E_n = \frac{h c}{\lambda_{mn}} \) menghasilkan:
\begin{equation}
\frac{I_{ij}}{I_{mn}} = \frac{g_i A_{ij} \lambda_{mn}}{g_m A_{mn} \lambda_{ij}} e^{-\frac{E_i - E_m}{k_B T}}. \label{eq:intensity_ratio_lambda}
\end{equation}
Untuk menghitung suhu \( T \), ambil logaritma natural:
\begin{equation}
\ln \left( \frac{I_{ij} \lambda_{ij}}{I_{mn} \lambda_{mn}} \right) = \ln \left( \frac{g_i A_{ij}}{g_m A_{mn}} \right) - \frac{E_i - E_m}{k_B T}. \label{eq:ln_ratio}
\end{equation}
Dari sini, suhu \( T \) dapat ditentukan sebagai:
\begin{equation}
T = \frac{E_i - E_m}{k_B \left[ \ln \left( \frac{g_i A_{ij}}{g_m A_{mn}} \right) - \ln \left( \frac{I_{ij} \lambda_{ij}}{I_{mn} \lambda_{mn}} \right) \right]}. \label{eq:temperature}
\end{equation}
Persamaan~\eqref{eq:temperature} memungkinkan penentuan suhu eksitasi plasma berdasarkan intensitas relatif garis spektral, yang konsisten dengan simulasi dan analisis spektrum atom \citep{rybicki-1985,Draine2011,Mason2015}.
\subsection{Persamaan Saha untuk Keseimbangan Ionisasi di Atmosfer Bintang}
Untuk plasma terionisasi, distribusi populasi diatur oleh persamaan Saha:
\begin{equation}
\frac{N_{i+1} N_e}{N_i} = \frac{2 Z_{i+1}}{Z_i} \left( \frac{2\pi m_e k_B T}{h^2} \right)^{3/2} e^{-\frac{\chi_i}{k_B T}},
\end{equation}
di mana $N_e$ adalah kepadatan elektron dan $\chi_i$ adalah energi ionisasi \citep{Pathria2011}. Persamaan ini digunakan untuk analisis keseimbangan ionisasi dalam atmosfer bintang.

\section{Pelebaran Garis Spektral dan Profil Voigt}
\subsection{Pelebaran Doppler dan Profil Gaussian dalam Lampu Decharge Gas}
Efek Doppler akibat gerakan termal atom menghasilkan profil Gaussian:
\begin{equation}
G(\lambda) = \frac{1}{\sqrt{2\pi \sigma^2}} \exp\left(-\frac{(\lambda - \lambda_0)^2}{2\sigma^2}\right),
\end{equation}
dengan $\sigma = \frac{\lambda_0}{c} \sqrt{\frac{k_B T}{m}}$ \citep{Demtroder2010}. Ini diamati dalam lampu decharge gas untuk analisis suhu.

\subsection{Pelebaran Tekanan dan Profil Lorentzian dalam Plasma Padat}
Pelebaran tekanan menghasilkan profil Lorentzian:
\begin{equation}
L(\lambda) = \frac{\Gamma / 2\pi}{(\lambda - \lambda_0)^2 + (\Gamma / 2)^2},
\end{equation}
di mana $\Gamma$ dipengaruhi oleh kepadatan plasma \citep{Demtroder2010}. Profil ini signifikan dalam plasma padat.

\subsection{Analisis Profil Voigt untuk Pelebaran Doppler dan Tekanan}
Profil Voigt, konvolusi Gaussian dan Lorentzian, diberikan oleh:
\begin{equation}
V(\lambda; \Gamma, \sigma) = \int_{-\infty}^{\infty} G(\lambda - \lambda') L(\lambda') \, d\lambda'.
\end{equation}
Profil ini digunakan untuk memodelkan garis spektral dalam analisis plasma \citep{Mason2015}.


