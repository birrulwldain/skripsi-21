%-------------------------------------------------------------------------------
%                            BAB II
%               TINJAUAN PUSTAKA DAN DASAR TEORI
%-------------------------------------------------------------------------------



\chapter{TINJAUAN PUSTAKA}
\par Dalam bab ini, berbagai teori yang relevan dan literatur sebelumnya yang menunjang penelitian ini dibahas secara komprehensif. Pembahasan mencakup teknik spektroskopi yang diaplikasikan, metode analisis spektrum, serta penerapan teknik statistik dalam pemrosesan data spektroskopi dengan penekanan pada penerapan \textit{Deep Learning} di dalamnya. Arsitektur \textit{Long Short-Term Memory} (LSTM) diterapkan untuk memprediksi elemen dalam sampel tersebut.

\section{Dasar Teoretis Spektroskopi Atom}
\subsection{Perkembangan Mekanika Kuantum}
[Isi Bab 2.1.1: Planck, Einstein, Bohr, Schrödinger.]

\subsection{Transisi Elektronik dan Spektrum Emisi}
Spektroskopi emisi atom bertumpu pada teori atom Bohr yang menjelaskan kuantisasi energi elektron melalui persamaan:
\begin{equation}
E_n = -\frac{13.6 \, \text{eV}}{n^2} \quad (n \in \mathbb{Z}^+),
\end{equation}
di mana transisi elektron antar tingkat energi memenuhi $\Delta E = h\nu$ \cite{Beiser1992}. Fenomena spektrum garis hidrogen mengikuti persamaan umum Rydberg:
\begin{equation}
\frac{1}{\lambda} = R \left( \frac{1}{n_f^2} - \frac{1}{n_i^2} \right) \quad (n_i > n_f),
\end{equation}
dengan $R = 1.097 \times 10^7 \, \text{m}^{-1}$ sebagai konstanta Rydberg \cite{Beiser1992}. Contoh spesifik seperti deret Balmer ($n_f = 2$) dan Lyman ($n_f = 1$) menunjukkan konsistensi persamaan ini \cite{Griffiths2005}.

Mekanisme eksitasi terjadi ketika atom menyerap energi dari sumber eksternal (e.g., plasma), menyebabkan elektron berpindah ke tingkat energi lebih tinggi. De-eksitasi menghasilkan emisi foton dengan panjang gelombang:
\begin{equation}
\lambda = \frac{hc}{\Delta E},
\end{equation}
di mana $h$ adalah konstanta Planck dan $c$ kecepatan cahaya. Aturan seleksi $\Delta l = \pm 1$ \cite{Liboff2003} membatasi transisi yang diperbolehkan, menghasilkan pola spektrum unik untuk setiap unsur.

Intensitas garis spektrum bergantung pada probabilitas transisi dan populasi elektron yang dapat dimodelkan dengan distribusi Boltzmann \cite{Demtroder2010}.

\subsection{Mekanika Statistik Kuantum}
[Isi Bab 2.1.3: Ansambel kanonik, Boltzmann, Fermi-Dirac.]

\subsection{Persamaan Saha}
[Isi Bab 2.1.4: Keseimbangan ionisasi.]

\subsection{Teori Plasma dan LTE}
[Isi Bab 2.1.5: Kriteria McWhirter, deviasi LTE.]

\subsection{Interaksi Radiasi-Materi}
[Isi Bab 2.1.6: Emisi, absorpsi, self-absorption.]

\subsection{Profil Garis Spektral}
Profil Voigt adalah gabungan dari dua fungsi profil spektral: profil Lorentzian dan Gaussian. Fungsi Voigt sering digunakan untuk mendeskripsikan bentuk garis spektrum yang diukur dalam berbagai teknik spektroskopi, termasuk Laser Induced Breakdown Spectroscopy (LIBS). Fungsi Voigt dapat didefinisikan sebagai konvolusi dari fungsi Lorentzian dan Gaussian.

Fungsi Lorentzian, \( L(\lambda) \), menggambarkan lebar garis spektrum yang disebabkan oleh efek tekanan atau damping. Dalam konteks panjang gelombang \( \lambda \), fungsi Lorentzian dapat dinyatakan sebagai:
\begin{equation}
L(\lambda) = \frac{\Gamma / 2\pi}{(\lambda - \lambda_0)^2 + (\Gamma / 2)^2},
\end{equation}
di mana \( \lambda \) adalah panjang gelombang yang diukur, \( \lambda_0 \) adalah panjang gelombang pusat dari garis spektrum, dan \( \Gamma \) adalah lebar garis Lorentzian yang berhubungan dengan lebar dari garis spektrum akibat efek tekanan. Dalam hal ini, \( \Gamma \) adalah Full Width at Half Maximum (FWHM) dari profil Lorentzian.

Fungsi Gaussian, \( G(\lambda) \), menggambarkan lebar garis spektrum yang disebabkan oleh efek Doppler, yang terkait dengan pergerakan relatif antara sumber spektrum dan detektor. Fungsi Gaussian dalam konteks panjang gelombang \( \lambda \) dinyatakan sebagai:
\begin{equation}
G(\lambda) = \frac{1}{\sqrt{2\pi \sigma^2}} \exp\left(-\frac{(\lambda - \lambda_0)^2}{2\sigma^2}\right),
\end{equation}
di mana \( \sigma \) adalah deviasi standar dari distribusi Gaussian, yang berkaitan dengan lebar dari garis spektrum dalam konteks efek Doppler. Deviasi standar \( \sigma \) berhubungan dengan Half Width at Half Maximum (HWHM) Gaussian, yang dinyatakan sebagai \( b = \sigma \sqrt{2 \ln 2} \).

Fungsi Voigt, \( V(\lambda; \Gamma, \sigma) \), adalah hasil konvolusi dari fungsi Lorentzian dan Gaussian. Konvolusi ini menghasilkan profil spektrum yang menggabungkan kedua efek. Fungsi Voigt didefinisikan sebagai:
\begin{equation}
V(\lambda; \Gamma, \sigma) = \int_{-\infty}^{\infty} G(\lambda - \lambda') L(\lambda') \, d\lambda'.
\end{equation}
Fungsi Voigt dapat disederhanakan dengan menggunakan fungsi \textit{W} (fungsi Voigt) yang merupakan integral konvolusi dari fungsi Lorentzian dan Gaussian. Fungsi Voigt dapat dinyatakan sebagai:
\begin{equation}
V(\lambda; a, b) = \text{Re} \left[ W\left(\frac{\lambda - \lambda_0 + i a}{b}\right) \right],
\end{equation}
di mana \( a \) adalah lebar Lorentzian (FWHM) dan \( b \) adalah deviasi standar Gaussian (HWHM). Fungsi Voigt kompleks, \( W(z) \), didefinisikan sebagai:
\begin{equation}
W(z) = e^{-z^2} \left( 1 + \text{erfi}(z) \right),
\end{equation}
dengan \text{erfi}(z) sebagai fungsi kesalahan kompleks.

\section{Aplikasi dalam LIBS}
\subsection{Simulasi Spektrum Atom}
Simulasi spektrum adalah alat penting dalam analisis spektroskopi, yang memungkinkan peneliti untuk memprediksi intensitas cahaya yang dipancarkan atau diserap oleh suatu substansi pada berbagai panjang gelombang. Penjelasan tentang bagaimana spektrum emisi dihasilkan dari plasma menggunakan persamaan intensitas emisi:
\begin{equation}
I_{ij} = A_{ij} \cdot N_i \cdot h \cdot \nu_{ij},
\end{equation}
di mana $I_{ij}$ adalah intensitas emisi, $A_{ij}$ adalah probabilitas transisi, $N_i$ adalah populasi atom pada tingkat energi i, $h$ adalah konstanta Planck, dan $\nu_{ij}$ adalah frekuensi transisi.

Fungsi partisi dihitung menggunakan:
\begin{equation}
Z = \sum_{i} g_i e^{-\frac{E_i}{k_B T}},
\end{equation}
di mana $g_i$ adalah degenerasi, $E_i$ adalah tingkat energi, $k_B$ adalah konstanta Boltzmann, dan $T$ adalah suhu \cite{pathria2011}.

Intensitas spektrum dihitung dengan rumus:
\begin{equation}
I = \frac{g \cdot e^{-\frac{E}{k_B T}} \cdot A}{Z},
\end{equation}
di mana $I$ adalah intensitas, $g$ adalah degenerasi, $E$ adalah energi, $A$ adalah koefisien Einstein, dan $Z$ adalah fungsi partisi \cite{mason2015}.

\subsection{Efek Matriks}
[Isi Bab 2.2.2: Pengaruh komposisi sampel.]

\subsection{Keterbatasan Pendekatan Tradisional}
[Isi Bab 2.2.3: PCA/PLS, ANN.]

\section{Kemajuan Pembelajaran Mesin}
\subsection{Peran ANN dan PCA/PLS}
[Isi Bab 2.3.1]

\subsection{Transformer dan Self-Attention}
[Isi Bab 2.3.2: Prinsip dan keunggulan.]

\subsection{Studi Terkini}
Long Short-Term Memory (LSTM) adalah jenis jaringan saraf yang dirancang untuk mengatasi tantangan dalam memproses dan memprediksi urutan data. LSTM sangat efektif dalam menangkap dependensi jangka panjang dalam data sekuensial, yang membuatnya berguna dalam berbagai aplikasi, termasuk analisis deret waktu, pemrosesan bahasa alami, dan pengenalan suara. Dengan kemampuannya untuk mengingat informasi dalam jangka waktu yang lebih lama, LSTM telah menjadi alat penting dalam bidang pembelajaran mesin \cite{hochreiter1997}.

LSTM telah terbukti menjadi alat yang sangat efektif dalam berbagai aplikasi pembelajaran mesin. Dengan kemampuan untuk menangkap dependensi jangka panjang dalam data sekuensial, LSTM telah membuka banyak peluang baru dalam analisis data dan pengembangan model prediktif \cite{graves2013}.

Struktur LSTM terdiri dari beberapa komponen kunci, termasuk sel memori dan tiga pintu: pintu input, pintu lupa, dan pintu output. Pintu input mengontrol informasi baru yang masuk ke dalam sel memori, pintu lupa menentukan informasi mana yang harus dihapus, dan pintu output mengatur informasi yang akan dikeluarkan dari sel memori. Mekanisme ini memungkinkan LSTM untuk mempertahankan informasi yang relevan dan mengabaikan informasi yang tidak penting, sehingga meningkatkan kinerja model dalam memprediksi urutan data \cite{graves2013}.

LSTM juga banyak digunakan dalam pemrosesan bahasa alami (NLP), di mana urutan kata dalam kalimat sangat penting. Dalam tugas-tugas seperti penerjemahan bahasa dan analisis sentimen, LSTM dapat digunakan untuk memahami konteks dan makna dari urutan kata. Dengan memanfaatkan kemampuan LSTM untuk mengingat informasi dari kata-kata sebelumnya, model dapat menghasilkan terjemahan yang lebih akurat dan analisis sentimen yang lebih tepat \cite{zhang2019}.

Dalam pengenalan suara, LSTM digunakan untuk mengubah sinyal audio menjadi teks. Model ini dilatih menggunakan data audio yang telah dilabeli untuk mengenali pola dalam suara dan menghasilkan transkripsi yang akurat. Dengan kemampuan LSTM untuk menangkap informasi temporal, model dapat mengenali kata-kata dalam konteks yang lebih luas, meningkatkan akurasi pengenalan suara.

Pelatihan model LSTM melibatkan penggunaan algoritma optimasi untuk meminimalkan fungsi kerugian. Proses ini biasanya dilakukan dengan menggunakan teknik backpropagation melalui waktu (BPTT), yang memungkinkan model untuk memperbarui bobot berdasarkan kesalahan prediksi. Pemilihan hyperparameter, seperti jumlah unit LSTM dan tingkat dropout, juga sangat penting untuk mencapai kinerja optimal \cite{bengio2012}.

Evaluasi model LSTM dilakukan dengan menggunakan metrik seperti akurasi, presisi, dan recall, tergantung pada jenis masalah yang dihadapi. Untuk masalah regresi, metrik seperti Mean Squared Error (MSE) sering digunakan. Evaluasi yang tepat sangat penting untuk memastikan bahwa model dapat diandalkan dan memberikan hasil yang akurat pada data baru.

\section{Celah Penelitian}
[Isi Bab 2.4: Validasi eksperimental, LTE, efek matriks.]




% \begin{thebibliography}{9}
% \bibitem{Beiser1992}
% Beiser, A., \emph{Concepts of Modern Physics}, McGraw-Hill, 1992.

% \bibitem{Griffiths2005}
% Griffiths, D. J., \emph{Introduction to Quantum Mechanics}, 2nd ed., Pearson Education, 2005.

% \bibitem{Liboff2003}
% Liboff, R. L., \emph{Introductory Quantum Mechanics}, 4th ed., Addison-Wesley, 2003.

% \bibitem{Demtroder2010}
% Demtröder, W., \emph{Laser Spectroscopy: Basic Concepts and Instrumentation}, 3rd ed., Springer, 2010.

% \bibitem{pathria2011}
% Pathria, R. K., \emph{Statistical Mechanics}, 3rd ed., Elsevier, 2011.

% \bibitem{mason2015}
% Mason, N. J., \emph{Atomic-Molecular Spectroscopy}, Cambridge University Press, 2015.

% \bibitem{hochreiter1997}
% Hochreiter, S., Schmidhuber, J., \emph{Long Short-Term Memory}, Neural Computation, vol. 9, no. 8, pp. 1735--1780, 1997.

% \bibitem{graves2013}
% Graves, A., \emph{Generating Sequences With Recurrent Neural Networks}, arXiv preprint arXiv:1308.0850, 2013.

% \bibitem{zhang2019}
% Zhang, Y., \emph{LSTM for Natural Language Processing}, Journal of Machine Learning Research, 2019.

% \bibitem{bengio2012}
% Bengio, Y., \emph{Practical Recommendations for Gradient-Based Training of Deep Architectures}, Neural Networks: Tricks of the Trade, Springer, 2012.
% \end{thebibliography}

% \end{document}