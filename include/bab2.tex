%-------------------------------------------------------------------------------
%                            BAB II
%               TINJAUAN PUSTAKA DAN DASAR TEORI
%-------------------------------------------------------------------------------

\chapter{TINJAUAN PUSTAKA DAN DASAR TEORI}

\section{Dasar Kuantum Emisi Spektral}

\subsection{Hukum Planck untuk Emisi Foton}
Hukum Planck menghubungkan energi foton dengan frekuensinya melalui persamaan:
\begin{equation}
E = h\nu, \label{planck}
\end{equation}
dimana \( h = 4.1357 \times 10^{-15} \, \text{eV·s} \) adalah konstanta Planck dan \( \nu \) adalah frekuensi. Hukum ini menjadi dasar emisi foton dalam transisi elektronik atom, yang diukur melalui spektroskopi dalam simulasi \textit{Laser-Induced Breakdown Spectroscopy} (LIBS) untuk memverifikasi tingkat energi \citep{Beiser1992}.

\subsection{Tingkat Energi Atom Hidrogen}
Dalam atom hidrogen, energi elektron pada tingkat kuantum \( n \) diberikan oleh:
\begin{equation}
E_n = -\frac{13.6 \, \text{eV}}{n^2} \quad (n \in \mathbb{Z}^+), \label{energy_hydrogen}
\end{equation}
Untuk transisi emisi, di mana elektron berpindah dari tingkat awal \( n_i \) (energi lebih tinggi) ke tingkat akhir \( n_f \) (energi lebih rendah, \( n_f < n_i \)), selisih energi adalah:
\begin{equation}
\Delta E = E_{n_f} - E_{n_i} = -13.6 \left( \frac{1}{n_f^2} - \frac{1}{n_i^2} \right) \, \text{eV}, \label{deltaE}
\end{equation}
Dimana \( n_i \) dan \( n_f \) menunjukkan indeks tingkat energi awal (\( i \)) dan akhir (\( k \)) dalam konteks transisi umum. Energi ini menentukan panjang gelombang foton yang diukur dalam simulasi spektrum LIBS \citep{Griffiths2005}.

\subsection{Diagram Grotrian untuk Transisi Spektral}
Diagram Grotrian memvisualisasikan tingkat energi dan transisi yang diizinkan, seperti deret Lyman (\( n_f = 1 \)) dan Balmer (\( n_f = 2 \)) pada hidrogen. Dalam simulasi LIBS, diagram ini membantu mengidentifikasi garis spektral hidrogen dan helium, memungkinkan analisis transisi elektronik yang relevan \citep{Mason2015}.

\subsection{Aturan Seleksi untuk Spektra Atom Alkali}
Transisi elektronik terjadi ketika elektron berpindah antar tingkat energi, dengan probabilitas ditentukan oleh elemen matriks dipol \( P \propto |\langle \psi_k | \hat{\mu} | \psi_i \rangle|^2 \), dimana \( \psi_i \) adalah keadaan awal dan \( \psi_k \) adalah keadaan akhir \citep{Demtroder2010}. Aturan seleksi meliputi \( \Delta l = \pm 1 \) dan \( \Delta m_l = 0, \pm 1 \), dengan aturan spin \( \Delta S = 0 \) mencegah transisi singlet-triplet kecuali ada kopling spin-orbit. Aturan ini diuji pada atom alkali dalam simulasi spektrum LIBS untuk memodelkan emisi spektral \citep{Griffiths2005}.

\section{Statistik Populasi dalam Plasma LIBS}

\subsection{Distribusi Boltzmann untuk Suhu Eksitasi}
Dalam plasma LIBS, populasi atom pada tingkat energi tertentu bergantung pada suhu plasma. Dalam kesetimbangan termal lokal, distribusi Boltzmann menentukan probabilitas pada tingkat energi ke-\( i \), yaitu \( P_i \), sebagai:
\begin{equation}
P_i \propto g_i e^{-\frac{E_i}{k_B T}}, \label{eq:boltzmann_propto}
\end{equation}
dimana \( g_i \) adalah faktor degenerasi, \( E_i \) adalah energi tingkat ke-\( i \), dan \( k_B = 8.617 \times 10^{-5} \, \text{eV / K} \) adalah konstanta Boltzmann. Kemudian, jika \( N \) adalah populasi atom, maka fraksi populasi relatif untuk tingkat energi ke-\( i \) adalah:
\begin{equation}
N_i = \frac{N g_i e^{-\frac{E_i}{k_B T}}}{Z}, \label{eq:boltzmann1}
\end{equation}
dan fungsi partisi \( Z \) didefinisikan sebagai:
\begin{equation}
Z = \sum_i g_i e^{-\frac{E_i}{k_B T}}, \label{eq:partition}
\end{equation}
Fungsi partisi memastikan normalisasi sehingga \( \sum_i N_i = N \). Persamaan ini bergantung pada suhu \( T \) dan sifat atom seperti energi \( E_i \) dan degenerasi \( g_i \) yang penting untuk simulasi spektrum LIBS menggunakan data dari \textit{National Institute of Standards and Technology} (NIST) \citep{Pathria2011,Rybicki1985}.

\subsection{Intensitas Garis Spektral dan Suhu Plasma}
Intensitas garis spektral \( I_{ik} \) dihasilkan dari emisi spontan saat atom bertransisi dari tingkat energi atas \( E_k \) ke tingkat energi bawah \( E_i \) (\( E_i < E_k \)). Probabilitas emisi spontan per satuan waktu diberikan oleh \( A_{ki} \). Daya total yang dihasilkan adalah:
\begin{equation}
P_{k} = N_k A_{ki} h \nu_{ki}, \label{eq:power}
\end{equation}
Intensitas garis spektral \( I_{ik} \) dihasilkan dari emisi spontan saat atom bertransisi dari tingkat energi atas \( E_k \) ke tingkat energi bawah \( E_i \) (\( E_k > E_i \)). Energi foton yang dipancarkan diberikan oleh:
\begin{equation}
h \nu_{ik} = E_k - E_i,
\end{equation}
dimana \( \nu_{ik} = \frac{E_k - E_i}{h} \) adalah frekuensi foton, dan \( \lambda_{ik} = \frac{c}{\nu_{ik}} = \frac{h c}{E_k - E_i} \) adalah panjang gelombang terkait, dengan \( h \nu_{ik} = \frac{h c}{\lambda_{ik}} \) dan \( h c \approx 1239.84 \, \text{eV·nm} \). Probabilitas emisi spontan per satuan waktu ditentukan oleh koefisien Einstein \( A_{ik} \) \citep{rybicki-1985}. Daya total yang dihasilkan dari transisi ini adalah:
\begin{equation}
P_k = N_k A_{ik} h \nu_{ik}, \label{eq:power}
\end{equation}
dimana \( N_k \) adalah populasi atom pada tingkat energi \( E_k \), dihitung menggunakan distribusi Boltzmann [Persamaan~\eqref{eq:boltzmann1}]. Intensitas \( I_{ik} \), yaitu daya per satuan sudut ruang dalam sudut solider \( 4\pi \) steradian, diberikan oleh:
\begin{equation}
I_{ik} = \frac{P_k}{4\pi} = \frac{N_k A_{ik} h \nu_{ik}}{4\pi}. \label{eq:intensity}
\end{equation}

Dengan mensubstitusi \( N_k \) dari Persamaan~\eqref{eq:boltzmann1}, dimana \( g_k \) adalah faktor degenerasi tingkat \( E_k \), \( k_B = \SI{8.617e-5}{\electronvolt\per\kelvin} \) adalah konstanta Boltzmann, \( T \) adalah suhu plasma, dan \( Z \) adalah fungsi partisi [Persamaan~\eqref{eq:partition}], intensitas menjadi:
\begin{equation}
I_{ik} = \frac{N g_k A_{ik} h \nu_{ik} e^{-\frac{E_k}{k_B T}}}{4\pi Z}. \label{eq:intensity_full}
\end{equation}

Dimana \( h \nu_{ik} \) dan \( 4\pi \) dapat diabaikan untuk perhitungan intensitas relatif yang dinormalisasi. Persamaan ini akan memfokuskan pada ketergantungan intensitas terhadap suhu plasma dan parameter spesifik transisi (\( g_k \), \( A_{ik} \), \( E_k \)). Dengan demikian, intensitas relatif dinyatakan sebagai:
\begin{equation}
I_{ik} \propto \frac{N g_k A_{ik} e^{-\frac{E_k}{k_B T}}}{Z}. \label{eq:intensity_relative}
\end{equation}

dimana \( Z \) dihitung untuk setiap tingkat energi transisi. Persamaan ini digunakan untuk menghitung intensitas relatif pada setiap panjang gelombang \( \lambda_{ik} \), yang penting untuk memodelkan spektrum \citep{rybicki-1985,draine-2011,Mason2015}.

\subsection{Persamaan Saha untuk Populasi Ion}

Persamaan Saha digunakan untuk menghitung rasio ionisasi \( \frac{n_{i+1} n_e}{n_i} \) berdasarkan suhu plasma \( T \), densitas elektron \( n_e \), dan energi ionisasi atom, yang mendukung pemodelan populasi ion untuk spektrum atom dalam simulasi \textit{Laser-Induced Breakdown Spectroscopy} (LIBS). Persamaan ini menggambarkan keseimbangan ionisasi:
\[
A_i \leftrightarrow A_i^+ + e^-.
\]
Untuk menurunkannya, kita mulai dengan fungsi partisi, yang merepresentasikan jumlah keadaan kuantum yang dapat diakses oleh partikel dalam kesetimbangan termal \citep{Pathria2011}.

Fungsi partisi total untuk setiap partikel (atom netral \( A_i \), ion \( A_i^+ \), atau elektron \( e^- \)) terdiri dari komponen translasional dan internal. Fungsi partisi translasional dihitung secara umum untuk partikel bebas dengan massa \( m \) dalam volume \( V \):
\begin{equation}
Z_{\text{trans}} = \frac{1}{h^3} \int e^{-p^2/(2m k_B T)} \, d^3q \, d^3p, \label{eq:partition_trans}
\end{equation}
dimana \( h \) adalah konstanta Planck, \( k_B = \SI{8.617e-5}{\electronvolt\per\kelvin} \) adalah konstanta Boltzmann, dan \( T \) adalah suhu plasma. Faktor \( 1/h^3 \) menormalkan ruang fase untuk setiap keadaan kuantum. Integral over posisi menghasilkan:
\begin{equation}
\int d^3q = V. \label{eq:integral_position}
\end{equation}
Integral over momentum, menggunakan koordinat bola dan substitusi \( u = p^2/(2m k_B T) \), adalah:
\begin{equation}
\int e^{-p^2/(2m k_B T)} \, d^3p = \left( 2\pi m k_B T \right)^{3/2}. \label{eq:integral_momentum}
\end{equation}
Substitusi ke dalam Persamaan~\eqref{eq:partition_trans} menghasilkan:
\begin{equation}
Z_{\text{trans}} = V \left( \frac{2\pi m k_B T}{h^2} \right)^{3/2}. \label{eq:partition_trans_final}
\end{equation}

Untuk atom netral (\( A_i \)) dengan massa \( m_i \):
\begin{equation}
Z_{i,\text{trans}} = V \left( \frac{2\pi m_i k_B T}{h^2} \right)^{3/2}. \label{eq:partition_trans_neutral}
\end{equation}
Untuk ion (\( A_i^+ \)) dengan massa \( m_{i+1} \approx m_i \):
\begin{equation}
Z_{i+1,\text{trans}} = V \left( \frac{2\pi m_{i+1} k_B T}{h^2} \right)^{3/2}. \label{eq:partition_trans_ion}
\end{equation}
Fungsi partisi internal mencakup keadaan energi elektronik:
\begin{equation}
Z_{\text{int}} = \sum_m g_m e^{-E_m/(k_B T)}, \label{eq:partition_internal}
\end{equation}
dimana \( g_m \) adalah faktor degenerasi keadaan dengan energi \( E_m \). Pada suhu LIBS (10.000–30.000 K), hanya keadaan dasar yang dominan, sehingga fungsi partisi internal untuk atom netral dan ion masing-masing adalah:
\[
Z_{i,\text{int}} \approx g_i, \quad Z_{i+1,\text{int}} \approx g_{i+1},
\]
dimana \( g_i \) dan \( g_{i+1} \) adalah degenerasi keadaan dasar. Fungsi partisi total untuk atom netral dan ion adalah:
\begin{equation}
Z_i = Z_{i,\text{int}} Z_{i,\text{trans}} = g_i \left( \frac{2\pi m_i k_B T}{h^2} \right)^{3/2} V, \quad Z_{i+1} = g_{i+1} \left( \frac{2\pi m_{i+1} k_B T}{h^2} \right)^{3/2} V. \label{eq:partition_total}
\end{equation}

Untuk elektron, yang merupakan fermion dengan spin-1/2, fungsi partisi harus mempertimbangkan statistik Fermi-Dirac:
\begin{equation}
f(E) = \frac{1}{e^{(E - \mu)/(k_B T)} + 1}, \label{eq:fermi_dirac}
\end{equation}
dimana \( \mu \) adalah potensial kimia. Dalam plasma LIBS pada suhu 10.000–30.000 K, densitas elektron rendah, sehingga plasma bersifat non-degenerasi (\( e^{(E - \mu)/(k_B T)} \gg 1 \)). Pada batas ini, distribusi Fermi-Dirac mendekati distribusi Maxwell-Boltzmann:
\begin{equation}
f(E) \approx e^{-(E - \mu)/(k_B T)}, \label{eq:fermi_dirac_approx}
\end{equation}
memungkinkan pendekatan klasik untuk menghitung fungsi partisi translasional elektron \citep{Pathria2011}. Dengan massa elektron \( m_e \), fungsi partisi translasional elektron adalah:
\begin{equation}
Z_{e,\text{trans}} = V \left( \frac{2\pi m_e k_B T}{h^2} \right)^{3/2}, \label{eq:partition_trans_electron}
\end{equation}
sesuai dengan Persamaan~\eqref{eq:partition_trans_final}. Elektron memiliki dua keadaan spin (\( m_s = \pm 1/2 \)), memberikan faktor degenerasi spin \( g_e = 2 \). Oleh karena itu, fungsi partisi total untuk elektron adalah:
\begin{equation}
Z_e = g_e Z_{e,\text{trans}} = 2 \left( \frac{2\pi m_e k_B T}{h^2} \right)^{3/2} V. \label{eq:partition_electron}
\end{equation}

Untuk menurunkan persamaan Saha, kita mempertimbangkan keseimbangan termal reaksi ionisasi \( A_i \leftrightarrow A_i^+ + e^- \). Rasio densitas dalam kesetimbangan termal diberikan oleh:
\begin{equation}
\frac{n_{i+1} n_e}{n_i} = \frac{Z_{i+1} Z_e}{Z_i} e^{-\frac{E_{xi}}{k_B T}}, \label{eq:saha_initial}
\end{equation}
dimana \( E_{xi} \) adalah energi ionisasi atom, diperoleh dari database NIST. Substitusi fungsi partisi dari Persamaan~\eqref{eq:partition_total} dan~\eqref{eq:partition_electron}:
\begin{equation}
\frac{n_{i+1} n_e}{n_i} = \frac{g_{i+1} \left( \frac{2\pi m_{i+1} k_B T}{h^2} \right)^{3/2} V \cdot 2 \left( \frac{2\pi m_e k_B T}{h^2} \right)^{3/2} V}{g_i \left( \frac{2\pi m_i k_B T}{h^2} \right)^{3/2} V} e^{-\frac{E_{xi}}{k_B T}}. \label{eq:saha_substitution}
\end{equation}
Karena \( m_i \approx m_{i+1} \), faktor massa dan volume membatalkan, sehingga:
\begin{equation}
\frac{n_{i+1} n_e}{n_i} = \frac{2 g_{i+1}}{g_i} \left( \frac{2\pi m_e k_B T}{h^2} \right)^{3/2} e^{-\frac{E_{xi}}{k_B T}}. \label{eq:saha_final}
\end{equation}
Dengan \( Z_{i,\text{int}} = g_i \) dan \( Z_{i+1,\text{int}} = g_{i+1} \), persamaan Saha dapat ditulis sebagai:
\begin{equation}
\frac{n_{i+1}}{n_i} = \frac{\frac{N_{i+1}}{V}} {\frac{N_{i}}{V}}= \frac{N_{i+1}} {N_{i}} = \frac{2Z_{i+1,\text{int}}}{N_eZ_{i,\text{int}}} \left( \frac{2\pi m_e k_B T}{h^2} \right)^{3/2} e^{-\frac{E_{xi}}{k_B T}}. \label{eq:saha_final_int}
\end{equation}

Persamaan ini digunakan dalam simulasi LIBS untuk menghitung rasio ionisasi dengan memasukkan suhu plasma \( T \), densitas elektron \( n_e \), dan energi ionisasi \( E_{xi} \). Konstanta-konstanta memiliki asal-usul fisik: faktor 2 dari degenerasi spin elektron, \( Z_{i+1,\text{int}}/Z_{i,\text{int}} \) dari rasio degenerasi keadaan dasar, \( \left( \frac{2\pi m_e k_B T}{h^2} \right)^{3/2} \) dari densitas keadaan translasional elektron, dan \( e^{-\frac{E_{xi}}{k_B T}} \) dari probabilitas ionisasi. Persamaan Saha berlaku untuk plasma non-degenerasi dalam LIBS, tetapi memerlukan modifikasi untuk plasma padat, seperti dalam kondisi ekstrem \citep{Chandrasekhar1939}.

Untuk memastikan plasma berada dalam kondisi \textit{Local Thermodynamic Equilibrium} (selanjutnya disebut LTE), kriteria McWhirter menetapkan batas minimum densitas elektron untuk LTE sebagai \citep{McWhirter1965}:
\begin{equation}
n_e \geq 1.6 \times 10^{12} T^{1/2} \Delta E^3 \, \text{cm}^{-3}, \label{eq:mcwhirter}
\end{equation}
dimana \( T \) adalah suhu dalam eV dan \( \Delta E \) adalah perbedaan energi antar level dalam eV. Kondisi ini memvalidasi asumsi distribusi Boltzmann, yang mendasari interpretasi data spektral temporal.

\section{Profil Garis Spektral dalam LIBS}

Dalam \textit{Laser-Induced Breakdown Spectroscopy} (LIBS), profil garis spektral merupakan alat penting untuk menentukan parameter plasma, seperti suhu (\( T \)), densitas elektron (\( n_e \)), dan komposisi elemen. Profil garis dihasilkan dari transisi elektronik atom atau ion dalam plasma, yang dipengaruhi oleh mekanisme pelebaran seperti efek Doppler, tekanan (termasuk pelebaran Stark), dan efek instrumental. Mekanisme ini menghasilkan profil Gaussian, Lorentzian, atau kombinasi keduanya (Voigt), yang memengaruhi intensitas garis, resolusi spektral, dan akurasi pengukuran parameter plasma \citep{Demtroder2010,Griem1997}. Bagian berikut menjelaskan profil garis ini, hubungan fisiknya dengan kondisi plasma, dan aplikasinya dalam simulasi LIBS.

\subsection{Pelebaran Doppler dan Profil Gaussian}
Efek Doppler, yang disebabkan oleh gerakan termal atom atau ion dalam plasma LIBS, menghasilkan pelebaran garis spektral dengan profil Gaussian:
\begin{equation}
G(\lambda) = \frac{1}{\sqrt{2\pi \sigma^2}} \exp\left(-\frac{(\lambda - \lambda_0)^2}{2\sigma^2}\right), \label{eq:gaussian}
\end{equation}
dimana \( \lambda_0 \) adalah panjang gelombang pusat transisi, dan \( \sigma \) adalah lebar setengah maksimum (HWHM) yang diberikan oleh:
\begin{equation}
\sigma = \frac{\lambda_0}{c} \sqrt{\frac{k_B T}{m}}, \label{eq:sigma_doppler}
\end{equation}
dengan \( c = \SI{2.998e8}{\meter\per\second} \) sebagai kecepatan cahaya, \( k_B = \SI{8.617e-5}{\electronvolt\per\kelvin} \) sebagai konstanta Boltzmann, \( T \) sebagai suhu plasma (dalam kelvin), dan \( m \) sebagai massa atom atau ion \citep{Demtroder2010}. Lebar Doppler (\( \sigma \)) sebanding dengan \( \sqrt{T/m} \), mencerminkan distribusi kecepatan Maxwell-Boltzmann dari partikel dalam plasma. Dalam LIBS, efek Doppler dominan pada tekanan rendah (< 1 mbar) dan suhu tinggi (10.000–30.000 K), memungkinkan estimasi suhu plasma melalui pemasangan profil Gaussian \citep{Miziolek2006}.

\subsection{Pelebaran Tekanan dan Profil Lorentzian}
Pelebaran tekanan, yang diakibatkan oleh tumbukan antara atom/ion pemancar dengan partikel lain (elektron, ion, atau atom netral) dalam plasma LIBS, menghasilkan profil Lorentzian:
\begin{equation}
L(\lambda) = \frac{\Gamma / 2\pi}{(\lambda - \lambda_0)^2 + (\Gamma / 2)^2}, \label{eq:lorentzian}
\end{equation}
dimana \( \Gamma \) adalah lebar penuh setengah maksimum (FWHM) yang bergantung pada densitas plasma. Dalam LIBS, pelebaran Stark, akibat interaksi dengan elektron dan ion, sering kali mendominasi \( \Gamma \). Lebar Stark dapat diaproksimasi sebagai:
\begin{equation}
\Gamma \approx 2w \left( \frac{n_e}{10^{16}} \right), \label{eq:stark_broadening}
\end{equation}
dimana \( w \) adalah parameter pelebaran Stark (dalam nm) untuk transisi spesifik, dan \( n_e \) adalah densitas elektron (dalam \( \text{cm}^{-3} \)) \citep{Griem1997,Konjevic1999}. Untuk transisi hidrogen seperti H\(\alpha\) (garis spektral hidrogen pada 656.3 nm), \( w \approx 0.1 \, \text{nm} \) pada \( T \approx 10.000 \, \text{K} \) dan \( n_e \approx 10^{16} \, \text{cm}^{-3} \). Pelebaran tekanan signifikan pada densitas elektron tinggi (\( n_e > 10^{15} \, \text{cm}^{-3} \)), memengaruhi intensitas sayap garis dan akurasi pengukuran distribusi Boltzmann \citep{Aragon2008}.

\subsection{Profil Voigt untuk Analisis Spektrum}
Profil Voigt adalah konvolusi dari profil Gaussian (efek Doppler dan instrumental) dan Lorentzian (efek tekanan/Stark):
\begin{equation}
V(\lambda; \Gamma, \sigma) = \int_{-\infty}^{\infty} G(\lambda - \lambda') L(\lambda') \, d\lambda', \label{eq:voigt}
\end{equation}
dimana \( \sigma \) dan \( \Gamma \) masing-masing adalah lebar Gaussian dan Lorentzian. Profil Voigt menggabungkan efek Doppler, yang dominan di pusat garis, dengan efek tekanan/Stark, yang dominan di sayap garis, menjadikannya model akurat untuk garis spektral dalam plasma LIBS \citep{Griem1997}. Dalam simulasi LIBS, profil Voigt digunakan untuk memodelkan bentuk garis spektral, memungkinkan ekstraksi parameter plasma seperti suhu (\( T \)) dari komponen Gaussian dan densitas elektron (\( n_e \)) dari komponen Lorentzian. Selain itu, profil Voigt kritis untuk menghitung intensitas garis yang digunakan dalam analisis distribusi Boltzmann untuk menentukan populasi tingkat energi \citep{Miziolek2006,Aragon2008}.

\subsection{Efek Fisik Profil Garis pada Analisis Spektral}
Mekanisme pelebaran garis memengaruhi analisis spektral dalam LIBS melalui hubungan kuantitatif dengan parameter plasma. Tabel~\ref{tab:broadening_effects} merangkum mekanisme pelebaran, profil garis terkait, parameter fisik yang memengaruhinya, dan dampaknya pada pengukuran spektral.

\begin{table}[H]
\centering
\caption{Mekanisme Pelebaran Garis Spektral dalam LIBS}
\label{tab:broadening_effects}
\begin{tabularx}{\textwidth}{XXXX}
\toprule
\textbf{Mekanisme Pelebaran} & \textbf{Profil Garis} & \textbf{Parameter Fisik} & \textbf{Dampak pada Analisis Spektral} \\
\midrule
Doppler & Gaussian & Suhu (\( T \)), massa atom (\( m \)) & Estimasi \( T \) melalui lebar \( \sigma \); memengaruhi resolusi spektral \citep{Demtroder2010}. \\
Stark & Lorentzian & Densitas elektron (\( n_e \)), suhu & Pengukuran \( n_e \) melalui lebar \( \Gamma \); dominan pada \( n_e > 10^{15} \, \si{\per\cubic\centi\meter} \) \citep{Griem1997,Konjevic1999}. \\
Tekanan (van der Waals) & Lorentzian & Densitas atom netral, tekanan & Memengaruhi sayap garis pada tekanan tinggi \citep{Konjevic1999}. \\
Instrumental & Gaussian & Resolusi spektrometer & Mengurangi akurasi lebar garis dan resolusi spektral \citep{Miziolek2006}. \\
Voigt (Konvolusi) & Voigt & \( T \), \( n_e \), resolusi & Model akurat untuk \( T \) dan \( n_e \); kunci untuk distribusi Boltzmann \citep{Aragon2008}. \\
\bottomrule
\end{tabularx}
\end{table}

Efek fisik dari pelebaran garis dalam LIBS meliputi:
\begin{itemize}
  \item \textbf{Suhu Plasma}: Lebar Doppler (\( \sigma \)) sebanding dengan \( \sqrt{T/m} \), memungkinkan pengukuran suhu plasma melalui pemasangan profil Gaussian. Untuk atom hidrogen pada \( T = 10.000 \, \text{K} \), \( \sigma \approx 0.01 \, \text{nm} \) untuk H\(\alpha\) \citep{Demtroder2010}.
  \item \textbf{Densitas Elektron}: Lebar Stark (\( \Gamma \)) berkorelasi dengan \( n_e \), memungkinkan diagnosis densitas elektron melalui pemasangan profil Lorentzian atau Voigt \citep{Griem1997,Konjevic1999}.
  \item \textbf{Intensitas Garis dan Distribusi Boltzmann}: Luas puncak spektral, yang dipengaruhi oleh profil Voigt, menentukan intensitas garis yang digunakan untuk menghitung rasio populasi tingkat energi dalam distribusi Boltzmann [Persamaan~\eqref{eq:boltzmann1}] \citep{Miziolek2006}.
  \item \textbf{Resolusi Spektral}: Pelebaran instrumental dan Doppler dapat mengaburkan garis spektral yang berdekatan, mengurangi kemampuan untuk membedakan transisi \citep{Aragon2008}.
\end{itemize}

Dalam simulasi LIBS, profil Voigt dihitung menggunakan algoritma numerik, seperti aproksimasi Humlíček, untuk memodelkan garis spektral dengan akurasi tinggi sambil meminimalkan biaya komputasi \citep{Griem1997}. Pendekatan ini memungkinkan analisis kuantitatif parameter plasma dan komposisi elemen dengan memisahkan kontribusi Gaussian dan Lorentzian melalui dek convolusi \citep{Aragon2008}.

\section{Pembelajaran Mendalam: Pemodelan Deret Waktu Berbasis \textit{Transformer}}
\label{sec:transformer_informer}

Pemodelan deret waktu adalah inti dari analisis data sekuensial, seperti prediksi cuaca, analisis data spektral, dan berbagai aplikasi berbasis deret waktu. Tantangan utamanya adalah menangkap ketergantungan jarak jauh (\textit{long-range dependencies}) dalam urutan panjang dengan efisiensi komputasi tinggi. Pemodelan deret waktu telah berkembang dari \textit{Recurrent Neural Networks} (RNN) \citep{Bengio1994}, yang terhambat oleh masalah \textit{vanishing gradient} dan pemrosesan sekuensial \citep{Hochreiter1997}. Model seperti \textit{Long Short-Term Memory} (LSTM) dan \textit{Gated Recurrent Unit} (GRU) meningkatkan penanganan ketergantungan jarak jauh, tetapi tetap kurang efisien untuk deret waktu sangat panjang \citep{Hochreiter1997}. \textit{Transformer}, diperkenalkan oleh \citet{Vaswani2017}, merevolusi pemrosesan deret waktu dengan mekanisme \textit{self-attention} yang memungkinkan pemrosesan paralel dan panjang jalur maksimum \( O(1) \). Namun, kompleksitas kuadratiknya (\( O(n^2) \)) membatasi skalabilitas untuk urutan panjang.

\begin{table}[H]
\centering
\caption{Perbandingan Karakteristik RNN dan \textit{Transformer}}
\label{tab:rnn_vs_transformer}
\begin{tabularx}{\textwidth}{>{\raggedright\arraybackslash}X>{\raggedright\arraybackslash}X>{\raggedright\arraybackslash}X}
\toprule
\textbf{Aspek} & \textbf{RNN} & \textbf{\textit{Transformer}} \\
\midrule
Proses & Rekursif, \( O(n) \) operasi sekuensial & Paralel, \( O(1) \) operasi sekuensial \\
Ketergantungan Jarak Jauh & Sulit (\textit{vanishing gradient}) & Efektif (\textit{self-attention}) \\
Kompleksitas per Lapisan & \( O(n \cdot d^2) \) & \( O(n^2 \cdot d) \) \\
Panjang Jalur Maksimum & \( O(n) \) & \( O(1) \) \\
\bottomrule
\end{tabularx}
\end{table}

\subsection{Arsitektur \textit{Transformer}}
Arsitektur \textit{Transformer} terdiri dari \textit{encoder} dan \textit{decoder}, masing-masing dengan \( N=6 \) lapisan identik, yang dirancang untuk memproses deret waktu secara paralel \citep{Vaswani2017}. Setiap lapisan \textit{encoder} memiliki dua sub-lapisan: \textit{multi-head self-attention} untuk menangkap ketergantungan temporal dan jaringan \textit{feed-forward} posisi-\textit{wise} untuk transformasi non-linear. Operasi penjumlahan (\textit{residual connection}) dan normalisasi lapisan (\textit{layer normalization}) diterapkan setelah setiap sub-lapisan:
\begin{equation}
y = x + \text{Sublayer}(x), \quad \text{LayerNorm}(y) = \gamma \cdot \frac{y - \mu}{\sqrt{\sigma^2 + \epsilon}} + \beta
\end{equation}
di mana \( x \in \mathbb{R}^{n \times 512} \) adalah input sub-lapisan (dimensi \( d_{\text{model}} = 512 \)), \(\text{Sublayer}(x)\) adalah output sub-lapisan, \( \mu \) dan \( \sigma^2 \) adalah rata-rata dan varians fitur per token, dan \( \epsilon = 10^{-5} \) mencegah pembagian nol. Parameter \( \gamma \) dan \( \beta \) memungkinkan penskalaan fleksibel. Operasi ini memastikan stabilitas pelatihan dengan mengurangi perubahan distribusi fitur, yang sangat penting untuk deret waktu multivariat seperti prediksi konsumsi energi, di mana pola temporal bervariasi. \textit{Decoder} menambahkan sub-lapisan \textit{multi-head attention} atas output \textit{encoder} dan \textit{masking} untuk sifat auto-regresif, memungkinkan peramalan deret waktu secara iteratif.

\begin{figure}[H]
    \centering
    \includegraphics[width=0.6\textwidth]{images/ModalNet-21.png}
    \caption{Arsitektur \textit{Transformer} \autocite{Vaswani2017}}
    \label{fig:transformer_architecture}
\end{figure}

\subsection{Mekanisme \textit{Attention}}
\label{sec:attention}

\begin{figure}[H]
    \centering
    \begin{subfigure}[t]{0.3\textwidth}
        \centering
        \includegraphics[width=\textwidth]{images/ModalNet-19.png}
        \caption{Mekanisme \textit{self-attention} dalam \textit{Transformer}}
        \label{fig:self_attention_a}
    \end{subfigure}
    \hfill
    \begin{subfigure}[t]{0.3\textwidth}
        \centering
        \includegraphics[width=\textwidth]{images/ModalNet-20.png}
        \caption{\textit{Scaled Dot-Product Attention}}
        \label{fig:self_attention_b}
    \end{subfigure}
    \caption{Ilustrasi mekanisme \textit{self-attention} dan \textit{scaled dot-product attention} dalam \textit{Transformer} \autocite{Vaswani2017}}
    \label{fig:self_attention_comparison}
\end{figure}

Mekanisme \textit{self-attention} adalah inti dari \textit{Transformer}, memungkinkan model menghitung ketergantungan antar token dalam urutan secara paralel \citep{Vaswani2017}. Menggunakan \textit{scaled dot-product attention}, setiap token direpresentasikan sebagai matriks \textit{query} (\( Q \)), \textit{key} (\( K \)), dan \textit{value} (\( V \)) dengan dimensi \( \mathbb{R}^{n \times d_k} \), di mana \( n \) adalah panjang urutan dan \( d_k = 64 \) (untuk \( h=8 \)). Skor perhatian dihitung sebagai:
\begin{equation}
\text{Score}_{ij} = \frac{(Q_i \cdot K_j)}{\sqrt{d_k}}
\end{equation}
Skor ini diskalakan oleh \( \sqrt{d_k} \) untuk mencegah nilai besar yang mengganggu \textit{softmax}, lalu dikonversi menjadi bobot perhatian:
\begin{equation}
\text{Attention}(Q, K, V) = \text{softmax}\left(\frac{QK^T}{\sqrt{d_k}}\right)V
\end{equation}
Operasi ini memiliki kompleksitas \( O(n^2 \cdot d_k) \), tetapi memungkinkan model menangkap hubungan temporal jarak jauh, seperti pola musiman dalam prediksi cuaca atau tren jangka panjang dalam konsumsi energi, tanpa ketergantungan pada pemrosesan sekuensial seperti RNN.

\subsection{\textit{Multi-Head Attention}}
\label{sec:multi_head_attention}

\textit{Multi-head attention} meningkatkan fleksibilitas \textit{self-attention} dengan memproyeksikan matriks \( Q \), \( K \), dan \( V \) ke \( h=8 \) ruang paralel, masing-masing dengan dimensi \( d_k = d_v = d_{\text{model}} / h = 64 \), untuk \( d_{\text{model}} = 512 \) \citep{Vaswani2017}. Setiap kepala menghitung:
\begin{equation}
\text{head}_i = \text{Attention}(Q W_i^Q, K W_i^K, V W_i^V)
\end{equation}
di mana \( W_i^Q, W_i^K, W_i^V \in \mathbb{R}^{512 \times 64} \) adalah matriks proyeksi, dan hasilnya digabungkan:
\begin{equation}
\text{MultiHead}(Q, K, V) = \text{Concat}(\text{head}_1, \ldots, \text{head}_8) W^O
\end{equation}
dengan \( W^O \in \mathbb{R}^{512 \times 512} \). Kompleksitas tetap \( O(n^2 \cdot d_{\text{model}}) \), tetapi setiap kepala menangkap pola temporal berbeda, seperti korelasi jangka pendek (misalnya, harian) dan jangka panjang (misalnya, mingguan) dalam deret waktu multivariat. Misalnya, dalam prediksi konsumsi energi, \textit{multi-head attention} dapat menghubungkan suhu harian dengan pola penggunaan listrik tahunan.

\subsection{\textit{Position-wise Feed-Forward Networks}}
\label{sec:ffn}

Setiap lapisan \textit{encoder} dan \textit{decoder} memiliki jaringan \textit{feed-forward} posisi-\textit{wise} (FFN) yang diterapkan pada setiap token secara independen (lihat Gambar~\ref{fig:transformer_architecture}) \citep{Vaswani2017}. FFN didefinisikan sebagai:
\begin{equation}
\text{FFN}(x) = \max(0, x W_1 + b_1) W_2 + b_2
\end{equation}
di mana \( x \in \mathbb{R}^{512} \), \( W_1 \in \mathbb{R}^{512 \times 2048} \), \( b_1 \in \mathbb{R}^{2048} \), \( W_2 \in \mathbb{R}^{2048 \times 512} \), dan \( b_2 \in \mathbb{R}^{512} \). Fungsi aktivasi ReLU (\(\max(0, \cdot)\)) memperkenalkan non-linearitas, memungkinkan model menangkap pola kompleks seperti anomali (misalnya, lonjakan konsumsi energi) atau perubahan tren. Kompleksitas FFN adalah \( O(n \cdot d_{\text{model}} \cdot d_{\text{ff}}) \), linier terhadap panjang urutan \( n \). Dalam deret waktu, FFN meningkatkan kapasitas representasi dengan memproses setiap titik waktu secara independen, mendukung analisis spektral atau prediksi cuaca.

\subsection{\textit{Positional Encoding}}
\label{sec:positional_encoding}

Karena \textit{Transformer} tidak memiliki rekurensi, informasi posisi token disuntikkan melalui \textit{positional encoding} (PE) \citep{Vaswani2017}. Untuk urutan dengan panjang \( n \), PE adalah matriks \( \mathbb{R}^{n \times 512} \), dihitung sebagai:
\begin{equation}
PE_{(pos, 2i)} = \sin\left(\frac{pos}{10000^{2i/512}}\right), \quad PE_{(pos, 2i+1)} = \cos\left(\frac{pos}{10000^{2i/512}}\right)
\end{equation}
di mana \( pos \) adalah indeks posisi (0 hingga \( n-1 \)), dan \( i \) adalah indeks dimensi (0 hingga 255). PE ditambahkan ke vektor input:
\begin{equation}
x_{\text{input}} = x + PE
\end{equation}
dengan \( x \in \mathbb{R}^{n \times 512} \) sebagai vektor representasi token. Fungsi sinusoid memastikan pola periodik yang mendukung generalisasi ke urutan panjang, krusial untuk peramalan deret waktu panjang seperti prediksi cuaca bulanan. Sifat ini memungkinkan model mempertahankan urutan temporal tanpa ketergantungan sekuensial.

\subsection{Kelebihan dan Keterbatasan}
\label{sec:transformer_limits}

\textit{Transformer} menawarkan keunggulan signifikan untuk pemodelan deret waktu \citep{Vaswani2017}:
\begin{enumerate}
    \item \textbf{Ketergantungan Jarak Jauh}: Panjang jalur maksimum \( O(1) \), memungkinkan model menangkap hubungan temporal seperti pola musiman tanpa batasan RNN.
    \item \textbf{Paralelisme}: Pemrosesan paralel mengurangi waktu pelatihan (12 jam pada 8 GPU P100 untuk WMT 2014).
    \item \textbf{Fleksibilitas}: Digeneralisasi ke tugas seperti analisis spektral (F1 92.7 untuk \textit{parsing}).
\end{enumerate}

Namun, keterbatasan meliputi:
\begin{enumerate}
    \item \textbf{Kompleksitas Kuadratik}: Kompleksitas waktu per lapisan adalah \( O(n^2 \cdot d) \) untuk \textit{self-attention} dan \( O(n \cdot d \cdot d_{\text{ff}}) \) untuk FFN, menghasilkan total:
    \begin{equation}
    \text{Waktu} = O(N \cdot (n^2 \cdot d + n \cdot d \cdot d_{\text{ff}}))
    \end{equation}
    Memori juga kuadratik: \( O(N \cdot n^2) \), membatasi skalabilitas untuk deret waktu panjang.
    \item \textbf{Kebutuhan Memori}: Penumpukan \( N=6 \) lapisan meningkatkan kebutuhan memori, terutama untuk \( n \) besar.
    \item \textbf{Decoding Lambat}: \textit{Dynamic decoding} auto-regresif memperlambat inferensi \citep{Zhou2021}.
\end{enumerate}
Keterbatasan ini mendorong pengembangan varian seperti \textit{Informer}, yang mengurangi kompleksitas menjadi \( O(n \cdot \log n) \) menggunakan \textit{ProbSparse self-attention}.

\section{\textit{Informer}: Model untuk Penelitian Empiris}
\label{sec:informer}

Pemodelan deret waktu panjang (\textit{long sequence time-series forecasting}, LSTF) menantang karena sulit menangkap hubungan jarak jauh dalam urutan data yang sangat panjang, seperti prediksi konsumsi energi atau lalu lintas \citep{Wu2021}. \textit{Transformer} \citep{Vaswani2017} efektif untuk pola musiman dan tren jangka panjang, tetapi membutuhkan memori dan waktu komputasi yang besar. \textit{Informer}, yang dikembangkan oleh \citet{Zhou2021}, mengatasi keterbatasan ini dengan mekanisme perhatian yang efisien, pengurangan dimensi data, dan inferensi cepat. \textit{Informer} menawarkan solusi yang efisien dan akurat, ideal untuk penelitian empiris pada LSTF.

\subsection{Arsitektur \textit{Informer}}
\label{sec:informer_architecture}

\textit{Informer} menggunakan struktur \textit{encoder-decoder} yang diadaptasi dari \textit{Transformer}, tetapi dirancang untuk deret waktu panjang \citep{Zhou2021}. \textit{Encoder} mengolah urutan masukan menjadi representasi yang lebih ringkas dengan mekanisme perhatian hemat memori dan pengurangan dimensi. \textit{Decoder} menghasilkan prediksi dalam satu langkah cepat, berbeda dari pendekatan berulang \textit{Transformer}. Desain ini mengurangi kebutuhan komputasi, mendukung aplikasi seperti prediksi konsumsi energi.

\subsection{\textit{ProbSparse Self-Attention}}
\label{sec:probsparse}

\textit{ProbSparse self-attention} meningkatkan efisiensi \textit{self-attention} tradisional dengan hanya memproses hubungan data yang paling relevan \citep{Zhou2021}. Berbeda dari \textit{Transformer} yang memeriksa semua hubungan, mekanisme ini memilih sebagian kecil koneksi penting, mengurangi beban komputasi. Pendekatan ini memungkinkan \textit{Informer} menangkap pola temporal yang signifikan, seperti dalam prediksi lalu lintas, dengan sumber daya minimal.

\subsection{\textit{Self-Attention Distilling}}
\label{sec:distilling}

\textit{Self-attention distilling} memungkinkan \textit{encoder} menyaring informasi penting dengan mengurangi ukuran data secara bertahap \citep{Zhou2021}. Proses ini menggunakan teknik seperti konvolusi untuk menghasilkan representasi yang lebih kecil, mengatasi keterbatasan memori \textit{Transformer}. \textit{Informer} juga menggunakan beberapa salinan \textit{encoder} untuk ketahanan, memungkinkan pemrosesan urutan panjang dengan efisien, cocok untuk analisis spektral.

\subsection{\textit{Generative-Style Decoder}}
\label{sec:decoder}

\textit{Decoder} \textit{Informer} menghasilkan urutan prediksi dalam satu langkah, lebih cepat dibandingkan \textit{Transformer} yang memerlukan proses berulang \citep{Zhou2021}. \textit{Decoder} memproses kombinasi data awal dan tempat kosong untuk menghasilkan keluaran yang mempertahankan urutan waktu. Pendekatan ini mendukung prediksi cepat untuk aplikasi seperti pemantauan jaringan listrik.

\subsection{\textit{Input Representation}}
\label{sec:input_representation}

Representasi masukan \textit{Informer} mengintegrasikan informasi temporal, seperti posisi dalam urutan dan konteks waktu (misalnya, jam atau hari) \citep{Zhou2021}. Ini memungkinkan model menangkap pola waktu yang kompleks, mendukung analisis data multivariat dalam LSTF.

\subsection{\textit{Metrik Evaluasi}}
\label{sec:evaluation_metrics}

\textit{Informer} dievaluasi menggunakan metrik standar seperti Mean Squared Error (MSE), Mean Absolute Error (MAE), dan Root Mean Squared Error (RMSE). MSE mengukur rata-rata kuadrat selisih antara prediksi dan nilai sebenarnya, menilai akurasi model. MAE menghitung rata-rata kesalahan absolut, memberikan gambaran sederhana tentang kesalahan. RMSE menawarkan interpretasi dalam satuan data asli. Metrik ini memungkinkan perbandingan \textit{Informer} dengan model lain, seperti \textit{Transformer} atau LSTM, dalam penelitian empiris.

\subsection{Aplikasi dan Relevansi Empiris}
\label{sec:informer_application}

\textit{Informer} mendukung LSTF untuk aplikasi seperti prediksi konsumsi listrik atau lalu lintas. Mekanisme perhatiannya menangkap hubungan temporal, pengurangan dimensi menghemat memori, dan \textit{decoder} cepat memungkinkan inferensi real-time (Tabel~\ref{tab:computational_efficiency}). Studi menunjukkan kemampuan \textit{distilling} menangani urutan panjang dengan efisien. Penelitian empiris dapat membandingkan \textit{Informer} dengan model lain menggunakan metrik seperti MSE atau MAE.

\begin{table}[H]
    \centering
    \caption{Perbandingan Efisiensi Komputasi per Lapisan \citep{Zhou2021}}
    \label{tab:computational_efficiency}
    \begin{tabular}{lccc}
        \toprule
        \textbf{Model} & \textbf{Waktu Pelatihan} & \textbf{Memori} & \textbf{Langkah Inferensi} \\
        \midrule
        \textit{Informer} & \( O(n \log n) \) & \( O(n \log n) \) & 1 \\
        \textit{Transformer} & \( O(n^2) \) & \( O(n^2) \) & \( n \) \\
        \textit{LogTrans} & \( O(n \log n) \) & \( O(n^2) \) & 1* \\
        \textit{Reformer} & \( O(n \log n) \) & \( O(n \log n) \) & \( n \) \\
        LSTM & \( O(n) \) & \( O(n) \) & \( n \) \\
        \bottomrule
        \multicolumn{4}{l}{*Menggunakan \textit{Decoder} generatif \textit{Informer}.}
    \end{tabular}
\end{table}

\subsection{Analisis Kinerja Empiris}
\label{sec:informer_performance}

\textit{Informer} menawarkan keunggulan dalam menangani urutan panjang dibandingkan \textit{Transformer} dan LSTM, dengan inferensi cepat untuk aplikasi real-time seperti prediksi lalu lintas. Perbandingan efisiensi (Tabel~\ref{tab:computational_efficiency}) menunjukkan \textit{Informer} lebih hemat dalam waktu pelatihan dan memori.

\subsection{Potensi Ekstensi Penelitian}
\label{sec:informer_extensions}

\textit{Informer} mendukung penelitian empiris melalui:
\begin{enumerate}
    \item \textbf{Optimasi Parameter}: Menyesuaikan pengaturan mekanisme perhatian.
    \item \textbf{Integrasi Multimodal}: Menggabungkan data tambahan, seperti cuaca, untuk prediksi energi.
    \item \textbf{Adaptasi Domain}: Menerapkan pada peramalan keuangan.
    \item \textbf{Perbandingan Model}: Dengan model seperti \textit{Performer} \citep{Choromanski2021}.
    \item \textbf{Skalabilitas}: Menguji pada data sangat panjang dengan perangkat keras modern.
\end{enumerate}
Metrik seperti MSE, MAE, dan waktu inferensi dapat digunakan untuk evaluasi.