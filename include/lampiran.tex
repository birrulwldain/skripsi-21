% File: include/lampiran.tex
% Lampiran untuk Tugas Akhir Jurusan Informatika Unsyiah
% Sesuai Panduan Tugas Akhir dan Tesis 2024 FMIPA Universitas Syiah Kuala
% Struktur: Hanya Lampiran 1, 2, dst., tanpa subbab atau section

% \begin{onehalfspace} % Spasi 1,5 sesuai dokumen utama

\lampiran{Algoritma Inisialisasi Data Spektral Atom}
\label{app:algo1}
\begin{algoritma}[H]
\small
\caption{Inisialisasi Data Spektral Atom}
\begin{algorithmic}[1]
  \REQUIRE Transition dataset $\mathcal{D} = \{ (\lambda_{ij}, E_i, E_k, g_i, g_k, A_{ki}) \mid \lambda_{ij} \in [200, 900], E_i, E_k, g_i, g_k, A_{ki} > 0 \}$ ; Number of element-ion pairs $k = 4$; Maximum spectral samples $N \in \mathbb{N}$; Temperature step $\Delta T > 0$; Electron density step $\Delta n_e > 0$
  \ENSURE Candidate atom set $\mathcal{C}$, atom subset dictionary $\mathcal{A}$, spectra set $\mathcal{S}$
  \STATE Validate inputs: Ensure $N > 0$, $\Delta T > 0$, $\Delta n_e > 0$
  \STATE Initialize: $\Delta T \gets 1000$, $\Delta n_e \gets 10^{0.5} \times 10^{12}$
  \STATE Define: $\mathcal{C} \gets \{\text{H}, \text{He}, \text{O}, \text{N}, \text{Si}, \text{Al}, \text{Fe}, \text{Ca}, \text{Mg}, \text{Na}, \text{Ti}, \text{Mn}, \text{S}, \text{Cl}, \text{Cr}, \text{Ni}, \text{Cu}\}$
  \STATE Initialize: $\mathcal{S} \gets \emptyset$, $\mathcal{A} \gets \emptyset$
  \FORALL{$T \in [5000, 15000]$ \textbf{step} $\Delta T$}
    \FORALL{$n_e \in [10^{12}, 10^{16}]$ \textbf{step} $\Delta n_e$}
      \STATE Randomly select $\mathcal{A}_{T,n_e} \subseteq \mathcal{C}$ with $|\mathcal{A}_{T,n_e}| = k$ without replacement
      \IF{no transitions exist in $\mathcal{D}$ for any species in $\mathcal{A}_{T,n_e}$}
        \STATE Log warning: ``No transitions for $\mathcal{A}_{T,n_e}$ at $T$, $n_e$'' \COMMENT{Skip}
        \STATE continue
      \ENDIF
      \STATE Store $(T, n_e, \mathcal{A}_{T,n_e})$ in $\mathcal{A}$
    \ENDFOR
  \ENDFOR
  \RETURN $\mathcal{C}$, $\mathcal{A}$, $\mathcal{S}$
\end{algorithmic}
\end{algoritma}
% Algoritma ini mengumpulkan dan memvalidasi data transisi atom dari basis data NIST, menginisialisasi set kandidat atom (\(\mathcal{C}\)), kamus subset atom (\(\mathcal{A}\)), dan set spektrum (\(\mathcal{S}\)). Subset atom dipilih secara acak untuk setiap kombinasi \(T\) dan \(n_e\), memastikan representasi yang beragam.

\lampiran{Algoritma Kalkulasi Rasio Populasi Ionisasi}
\label{app:algo2}
\begin{algoritma}[h]
\small
\caption{Kalkulasi Rasio Populasi Ionisasi}
\begin{algorithmic}[1]
  \REQUIRE Transition dataset $\mathcal{D} = \{ (\lambda_{ij}, E_i, E_k, g_i, g_k, A_{ki}) \mid \lambda_{ij} \in [200, 900], E_i, E_k, g_i, g_k, A_{ki} > 0 \}$; \\ Candidate atom set $\mathcal{C}$; \\ Atom subset dictionary $\mathcal{A} = \{ (T, n_e, \mathcal{A}_{T,n_e}) \}$; \\ Physical constants $m_e = 9.109 \times 10^{-31}$, $k_B = 8.617 \times 10^{-5}$, $h = 4.1357 \times 10^{-15}$
  \ENSURE Population ratio dictionary $\mathcal{R} = \{ (T, n_e, \mathbf{R}_{T,n_e}) \mid \mathbf{R}_{T,n_e} \in [0, 1]^2 \}$
  \STATE Initialize $\mathcal{R} \gets \emptyset$ \COMMENT{Ratio dictionary}
  \FORALL{$(T, n_e, \mathcal{A}_{T,n_e}) \in \mathcal{A}$}
    \STATE $\mathbf{R}_{T,n_e} \gets \emptyset$ \COMMENT{Temporary ratio}
    \FORALL{$S \in \mathcal{A}_{T,n_e}$}
      \STATE Define $(S_{\text{neutral}}, S_{\text{ion}}) \gets (S_{\text{neutral}}, S_{\text{ion}})$ \COMMENT{Species pair}
      \STATE Extract $\mathcal{T}_S \subseteq \mathcal{D}$ for $S_{\text{neutral}}$ or $S_{\text{ion}}$ \COMMENT{Transitions}
      \IF{$\mathcal{T}_S = \emptyset$}
        \STATE Log warning: ``No transitions for $S$'' \COMMENT{Skip}
        \STATE continue
      \ENDIF
      \STATE $Z_{\text{neutral}} \gets \sum_i g_i \exp\left(-\frac{E_i}{k_B T}\right)$ \COMMENT{Neutral partition}
      \IF{$Z_{\text{neutral}} \leq 0$}
        \STATE Log warning: ``Invalid partition function for $S_{\text{neutral}}$'' \COMMENT{Skip}
        \STATE continue
      \ENDIF
      \STATE $Z_{\text{ion}} \gets \sum_i g_i \exp\left(-\frac{E_i}{k_B T}\right)$ \COMMENT{Ion partition}
      \IF{$Z_{\text{ion}} \leq 0$}
        \STATE Log warning: ``Invalid partition function for $S_{\text{ion}}$'' \COMMENT{Skip}
        \STATE continue
      \ENDIF
      \STATE $\frac{N_{\text{ion}}}{N_{\text{neutral}}} \gets \frac{2 Z_{\text{ion}}}{n_e Z_{\text{neutral}}} \left( \frac{2\pi m_e k_B T}{h^2} \right)^{3/2} \exp\left(-\frac{E_{\text{ion}}}{k_B T}\right)$ \COMMENT{Ionization ratio}
      \STATE $f_{\text{neutral}} \gets \frac{1}{1 + \frac{N_{\text{ion}}}{N_{\text{neutral}}}}$ \COMMENT{Neutral fraction}
      \STATE $f_{\text{ion}} \gets \frac{\frac{N_{\text{ion}}}{N_{\text{neutral}}}}{1 + \frac{N_{\text{ion}}}{N_{\text{neutral}}}}$ \COMMENT{Ion fraction}
      \STATE Store $(f_{\text{neutral}}, f_{\text{ion}})$ in $\mathbf{R}_{T,n_e}[S_{\text{neutral}}, S_{\text{ion}}]$ \COMMENT{Store ratio}
    \ENDFOR
    \STATE Store $(T, n_e, \mathbf{R}_{T,n_e})$ in $\mathcal{R}$ \COMMENT{Add to dictionary}
  \ENDFOR
  \RETURN $\mathcal{R}$ \COMMENT{Result}
\end{algorithmic}
\end{algoritma}
% Algoritma ini menghitung rasio populasi ionisasi untuk setiap pasangan elemen-ion, menghasilkan kamus rasio populasi (\(\mathcal{R}\)). Rasio ini penting untuk menentukan fraksi atom netral dan terionisasi dalam plasma.

\lampiran{Algoritma Kalkulasi Intensitas Garis Spektral}
\label{app:algo3}
\begin{algoritma}[H]
\small
\caption{Kalkulasi Intensitas Garis Spektral}
\begin{algorithmic}[1]
  \REQUIRE Transition dataset $\mathcal{D} = \{ (\lambda_{ij}, E_i, E_k, g_i, g_k, A_{ki}, m_a, w) \mid \lambda_{ij} \in [200, 900], E_i, E_k, g_i, g_k, A_{ki}, m_a, w > 0 \}$; \\ Atom subset dictionary $\mathcal{A} = \{ (T, n_e, \mathcal{A}_{T,n_e}) \}$; \\ Population ratio dictionary $\mathcal{R} = \{ (T, n_e, \mathbf{R}_{T,n_e}) \mid \mathbf{R}_{T,n_e} \in [0, 1]^2 \}$; \\ Physical constants $c = 2.998 \times 10^8$, $k_B = 8.617 \times 10^{-5}$
  \ENSURE Temporary intensity dictionary $\mathcal{I} = \{ (T, n_e, \mathbf{I}_{\text{temp}}) \mid \mathbf{I}_{\text{temp}} = \{ (\lambda_{ij}, I_{\text{rel}}) \} \}$
  \STATE Initialize $\mathcal{I} \gets \emptyset$ \COMMENT{Temporary intensity dictionary}
  \FORALL{$(T, n_e, \mathcal{A}_{T,n_e}) \in \mathcal{A}$}
    \STATE Extract $\mathbf{R}_{T,n_e}$ from $\mathcal{R}$ \COMMENT{Population ratio}
    \STATE $\mathbf{I}_{\text{temp}} \gets \emptyset$ \COMMENT{Temporary intensity list}
    \FORALL{$S \in \mathcal{A}_{T,n_e}$}
      \STATE Extract $\mathcal{T}_S \subseteq \mathcal{D}$ for $S_{\text{neutral}}$ or $S_{\text{ion}}$ \COMMENT{Transitions}
      \IF{$\mathcal{T}_S = \emptyset$}
        \STATE Log warning: ``No transitions for $S$'' \COMMENT{Skip}
        \STATE continue
      \ENDIF
      \FORALL{$(\lambda_{ij}, E_i, E_k, g_i, g_k, A_{ki}, m_a, w) \in \mathcal{T}_S$}
        \STATE $\Delta E \gets E_k - E_i$ \COMMENT{Energy}
        \STATE $n_{e,\text{min}} \gets 1.6 \times 10^{12} T^{1/2} (\Delta E)^{3/2}$ \COMMENT{Minimum density}
        \IF{$n_e \geq n_{e,\text{min}}$}
          \IF{$S$ is $S_{\text{neutral}}$}
            \STATE $Z \gets Z_{\text{neutral}}$ (Algoritma 2) \COMMENT{Neutral partition}
            \STATE $f \gets f_{\text{neutral}}$ from $\mathbf{R}_{T,n_e}$ \COMMENT{Neutral fraction}
          \ELSE
            \STATE $Z \gets Z_{\text{ion}}$ (Algoritma 2) \COMMENT{Ion partition}
            \STATE $f \gets f_{\text{ion}}$ from $\mathbf{R}_{T,n_e}$ \COMMENT{Ion fraction}
          \ENDIF
          \STATE $I_{\text{rel}} \gets \frac{g_k A_{ki} \exp\left(-\frac{E_k}{k_B T}\right)}{Z} \cdot f$ \COMMENT{Relative intensity}
          \STATE Append $(\lambda_{ij}, I_{\text{rel}})$ to $\mathbf{I}_{\text{temp}}$ \COMMENT{Accumulate}
        \ELSE
          \STATE Log warning: ``LTE condition not satisfied'' \COMMENT{Skip}
        \ENDIF
      \ENDFOR
    \ENDFOR
    \STATE Store $(T, n_e, \mathbf{I}_{\text{temp}})$ in $\mathcal{I}$ \COMMENT{Add to dictionary}
  \ENDFOR
  \RETURN $\mathcal{I}$ \COMMENT{Result}
\end{algorithmic}
\end{algoritma}
% \par Algoritma ini menghitung intensitas relatif garis spektral (\(I_{\text{rel}}\)) untuk setiap transisi atom, menghasilkan kamus intensitas sementara (\(\mathcal{I}\)). Intensitas ini mencerminkan probabilitas emisi foton.

\lampiran{Algoritma Kalkulasi Spektrum Emisi dengan Profil Voigt}
\label{app:algo4}
\begin{algoritma}[H]
\small
\caption{Kalkulasi Spektrum Emisi Atom dengan Profil Garis \textit{Voigt}}
\begin{algorithmic}[1]
  \REQUIRE Temporary intensity dictionary $\mathcal{I} = \{ (T, n_e, \mathbf{I}_{\text{temp}}) \mid \mathbf{I}_{\text{temp}} = \{ (\lambda_{ij}, I_{\text{rel}}) \} \}$; \\ Transition dataset $\mathcal{D} = \{ (\lambda_{ij}, E_i, E_k, g_i, g_k, A_{ki}, m_a, w) \mid \lambda_{ij} \in [200, 900], E_i, E_k, g_i, g_k, A_{ki}, m_a, w > 0 \}$; \\ Atom subset dictionary $\mathcal{A} = \{ (T, n_e, \mathcal{A}_{T,n_e}) \}$; \\ Population ratio dictionary $\mathcal{R} = \{ (T, n_e, \mathbf{R}_{T,n_e}) \mid \mathbf{R}_{T,n_e} \in [0, 1]^2 \}$; \\ Physical constants $c = 2.998 \times 10^8$, $k_B = 8.617 \times 10^{-5}$; \\ Maximum spectral samples $N \in \mathbb{N}$; \\ Spectra set $\mathcal{S}$
  \ENSURE $\mathcal{S}$ with $(T, n_e, \mathbf{I}_{T,n_e}, \mathbf{R}_{T,n_e})$
  \STATE Initialize $c \gets 0$ \COMMENT{Counter}
  \FORALL{$(T, n_e, \mathbf{I}_{\text{temp}}) \in \mathcal{I}$}
    \STATE Extract $\mathbf{R}_{T,n_e}$ from $\mathcal{R}$ \COMMENT{Population ratio}
    \STATE $\mathbf{I}_{T,n_e} \gets \emptyset$ \COMMENT{Final intensity list}
    \FORALL{$(\lambda_{ij}, I_{\text{rel}}) \in \mathbf{I}_{\text{temp}}$}
      \STATE Extract $m_a, w$ from $\mathcal{D}$ for the transition at $\lambda_{ij}$ \COMMENT{Atom mass and Lorentzian width}
      \STATE $\alpha_G \gets \frac{\lambda_{ij}}{c} \sqrt{\frac{2 k_B T \ln 2}{m_a}}$ \COMMENT{Gaussian HWHM (Doppler)}
      \STATE $\alpha_L \gets w \cdot \frac{n_e}{10^{16} \sqrt{T}}$ \COMMENT{Lorentzian HWHM}
      \STATE $I(\lambda) \gets I_{\text{rel}} \cdot V(\lambda - \lambda_{ij}, \alpha_G, \alpha_L)$ \COMMENT{Voigt profile}
      \STATE Append $(\lambda_{ij}, I(\lambda))$ to $\mathbf{I}_{T,n_e}$ \COMMENT{Accumulate}
    \ENDFOR
    \IF{$\max(\{I \mid (\lambda_{ij}, I) \in \mathbf{I}_{T,n_e}\}) > 0$}
      \STATE Normalize $\mathbf{I}_{T,n_e}$: divide each $I$ by $\max(\{I \mid (\lambda_{ij}, I) \in \mathbf{I}_{T,n_e}\})$ \COMMENT{Normalization}
    \ENDIF
    \STATE Store $(T, n_e, \mathbf{I}_{T,n_e}, \mathbf{R}_{T,n_e})$ in $\mathcal{S}$ \COMMENT{Add spectrum}
    \STATE $c \gets c + 1$ \COMMENT{Increment}
    \IF{$c \mod 1000 = 0$} %\COMMENT{Check storage modulo}
      \STATE Save $\mathcal{S}$ \COMMENT{Save}
      \STATE Clear $\mathcal{S}$ \COMMENT{Clear}
    \ENDIF
    \IF{$c \geq N$}
      \STATE break \COMMENT{Terminate}
    \ENDIF
  \ENDFOR
  \RETURN $\mathcal{S}$ \COMMENT{Result}
\end{algorithmic}
\end{algoritma}
% Algoritma ini menghitung parameter pelebaran Gaussian (\(\alpha_G\)) dan Lorentzian (\(\alpha_L\)), lalu mengaplikasikan profil Voigt untuk menghasilkan spektrum emisi akhir, menyimpannya dalam set spektrum (\(\mathcal{S}\)) dengan normalisasi intensitas.

% % \end{onehalfspace}