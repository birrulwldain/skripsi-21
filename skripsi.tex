%
% Template Laporan Tugas Akhir Jurusan Informatika Unsyiah 
%
% @author Abdul Hafidh
% @version 1.1
% @since 08.09.2023
%
% Template ini telah disesuaikan dengan aturan penulisan tugas akhir 
% yang terdapat pada dokumen Panduan Tugas Akhir FMIPA Unsyiah tahun 2016.
%

% karena jifhasiltheme.cls ada di folder lib, maka kita harus menambahkan path lib/ ke dalam path pencarian file
\makeatletter
\def\input@path{{lib/}}
\makeatother
% Template pembuatan naskah tugas akhir.
\documentclass[dvipsnames]{jifhasiltheme-final}

\tolerance=1
\emergencystretch=\maxdimen
\hyphenpenalty=10000
\hbadness=10000

% Karena file hype.indonesia.tex ada di folder language, tambah path pencarian file
\makeatletter
\def\input@path{{language/}}
\makeatother
\include{hype.indonesia}

% Untuk prefiks pada daftar gambar dan tabel
\usepackage[titles]{tocloft}

\usepackage{etoolbox}% http://ctan.org/pkg/etoolbox
\makeatletter
\patchcmd{\@chapter}{\addtocontents{lof}{\protect\addvspace{10\p@}}}{}{}{} % LoF
\patchcmd{\@chapter}{\addtocontents{lot}{\protect\addvspace{10\p@}}}{}{}{} % LoT
\makeatother

\usepackage[justification=centering]{caption} % atau misalnya [format=hang]
\usepackage{amssymb}
% Tambahan dari Budi
\newcommand*{\enableboldchapterintoc}{%
  \addtocontents{toc}{\string\renewcommand{\protect\cftchapfont}{\protect\normalfont\protect\bfseries}}%
  \addtocontents{toc}{\string\renewcommand{\protect\cftchappagefont}{\protect\normalfont\protect}}%
  \addtocontents{toc}{\protect\setlength{\cftbeforechapskip}{12pt}}%
}
\newcommand*{\disableboldchapterintoc}{%
  \addtocontents{toc}{\string\renewcommand{\protect\cftchappagefont}{\protect\normalfont}}%
  \addtocontents{toc}{\string\renewcommand{\protect\cftchapfont}{\protect\normalfont}}%
  \addtocontents{toc}{\protect\setlength{\cftbeforechapskip}{0pt}}%
}
% End tambahan dari Budi

\renewcommand{\cftdotsep}{0.5}
\renewcommand{\cftchapleader}{\cftdotfill{\cftdotsep}}

\renewcommand\cftfigpresnum{Gambar\  }
\renewcommand\cfttabpresnum{Tabel\   }

\newcommand{\listappendicesname}{DAFTAR LAMPIRAN}
\newlistof{appendices}{apc}{\listappendicesname}
\newcommand{\appendices}[1]{\addcontentsline{apc}{appendices}{#1}}
\newcommand{\newappendix}[1]{\section*{#1}\appendices{#1}}

% Untuk hyperlink dan table of content
\usepackage[hidelinks]{hyperref}
\renewcommand\UrlFont{\rmfamily\itshape}
\newlength{\mylenf}
\settowidth{\mylenf}{\cftfigpresnum}
\setlength{\cftfignumwidth}{\dimexpr\mylenf+2em}
\setlength{\cfttabnumwidth}{\dimexpr\mylenf+2em}

% Agar ada tulisan BAB pada TOC
\renewcommand\cftchappresnum{BAB } 
\cftsetindents{chapter}{0em}{4.5em}
\cftsetindents{section}{4.5em}{2em}
\cftsetindents{subsection}{6.5em}{3em}

\renewcommand{\cftsecaftersnum}{.}
\renewcommand{\cftsubsecaftersnum}{.}

\addtocontents{toc}{~\hfill \textit{Halaman}\par}
\addtocontents{lof}{~\hfill \textit{Halaman}\par}
\addtocontents{lot}{~\hfill \textit{Halaman}\par}
\addtocontents{apc}{~\hfill \textit{Halaman}\par}

% Untuk judul section
\usepackage{titlesec}
\titlelabel{\thetitle.\quad}

% Untuk caption dengan titik sebagai pemisah
\usepackage[labelsep=period]{caption}
\usepackage[labelfont=bf]{caption}
\usepackage{subcaption}

\usepackage{color} % untuk warna
\usepackage{longtable}
\usepackage{pdflscape}
\usepackage{lscape}
\usepackage{listings, lstautogobble}
\usepackage{adjustbox}
\usepackage{fancybox, graphicx} % shadow gambar
\usepackage{url}
\usepackage{microtype}
\usepackage{siunitx}
\usepackage{xcolor}
\usepackage{multirow}
\usepackage[normalem]{ulem}
\useunder{\uline}{\ul}{}
\usepackage{array}
\newcolumntype{P}[1]{>{\centering\arraybackslash}p{#1}}
\newcolumntype{M}[1]{>{\centering\arraybackslash}m{#1}}

\makeatletter
\def\input@path{{include/}}
\makeatother

% Sampul Depan
%-----------------------------------------------------------------
% Sampul Depan
%-----------------------------------------------------------------
\judulcover{Identifikasi Multi Elemen pada Spektrum Emisi LIBS Kompleks Dari Tanah Vulkanik Seulawah Agam Menggunakan Algoritma \textit{Long Short-Term Memory} (LSTM)}

\judul{Identifikasi Multi Elemen pada Spektrum Emisi LIBS Kompleks Dari Tanah Vulkanik Seulawah Agam Menggunakan Algoritma \textit{Long Short-Term Memory} (LSTM)}
\judulinggris{Identification of Multi-Elements in Complex LIBS Emission Spectra from Seulawah Agam Volcanic Soil Using Long Short-Term Memory (LSTM) Algorithm)}

% nama lengkap
\fullname{Birrul Walidain }

% NPM (Nomor Pokok Mahasiswa)
\idnum{2008102010010}

\degree{Sarjana Sains}

\yearsubmit{Desember, 2024}

\dept{Fisika}

\dept{Fisika}

% Pembimbing Pertama
\firstsupervisor{Prof. Dr. Eng. Nasrullah, S.Si., M.T.}
\firstnip{197607031995121001}

% Pembimbing Kedua
\secondsupervisor{Dr. Khairun Saddami, S.T.}
\secondnip{199103182022031008}

% Ketua Jurusan
\kajur{Dr. Saumi Syahreza, S.Si., M.Si.}
\kajurnip{197609172005011002}

% Dekan Fakultas
\dekan{Prof. Dr. Taufik Fuadi Abidin, S.Si., M.Tech.}
\dekannip{197010081994031002}

%kaprodi
\kaprodi{Dr. Saumi Syahreza, S.Si., M.Si.}
\kaprodinip{197609172005011002}

% tangal lulus proposal, seminar hasil atau sidang
\approvaldate{Kamis, 14 Mei 2024}

%-----------------------------------------------------------------
% End of Sampul Depan
%-----------------------------------------------------------------


% Awal dokumen
\usepackage{fancyhdr}
\usepackage{rotating}
% Untuk daftar program
\makeatletter
\begingroup\let\newcounter\@gobble\let\setcounter\@gobbletwo
\globaldefs\@ne \let\c@loldepth\@ne
\newlistof{listings}{lol}{\lstlistlistingname}
\endgroup
\let\l@lstlisting\l@listings
\AtBeginDocument{\addtocontents{lol}{\protect\addvspace{10\p@}}}
\makeatother
\renewcommand{\lstlistoflistings}{\listoflistings}
\renewcommand\cftlistingspresnum{Program~}
\cftsetindents{listings}{1.5em}{7em}

% Tab untuk daftar pustaka
\setlength{\bibhang}{30pt}
\usepackage{pdfpages}
\usepackage[table,xcdraw]{xcolor} % Jika diperlukan, pastikan xcdraw hanya dimuat sekali
\usepackage{colortbl}

\begin{document}
\sloppy
\fancyhf{} 
\fancyfoot[C]{\thepage}

\cover
\approvalpage

\bplagiasi % Note: \preface JANGAN DIHAPUS!
\onehalfspacing
\noindent
Saya yang bertanda tangan di bawah ini,

\vspace{0.2cm}
\noindent
\begin{center}
	\begin{tabularx}{0.9\textwidth}{@{} l X @{}} % Gunakan X untuk kolom yang menyesuaikan lebar
		Nama lengkap          & : Birrul Walidain \\
		Tempat/tanggal lahir  & : Lhokseumawe/11 November 2002 \\
		NPM                   & : 2008102010010 \\
		Program Studi         & : Fisika \\
		Fakultas              & : MIPA \\
		Judul Tugas Akhir     & : 
		\begin{tabular}[t]{@{}p{\linewidth}@{}}
			IDENTIFIKASI MULTI ELEMEN PADA SPEKTRUM EMISI LIBS KOMPLEKS DARI 
			TANAH VULKANIK SEULAWAH AGAM MENGGUNAKAN ALGORITMA 
			\textit{LONG SHORT-TERM MEMORY} (LSTM)
		\end{tabular} \\
	\end{tabularx}
\end{center}



\vspace{0.2cm}
\noindent
Menyatakan dengan sesungguhnya bahwa Laporan Tugas Akhir saya dengan judul di atas adalah \textbf{hasil karya saya sendiri} bersama dosen pembimbing dan \textbf{bebas plagiasi}.

\vspace{0.2cm}
\noindent
Jika ternyata di kemudian hari terbukti bahwa Laporan Tugas Akhir merupakan hasil plagiasi, saya bersedia menerima sanksi yang berlaku di Universitas Syiah Kuala.

\vspace{1cm}
\begin{onehalfspace}
	\begin{flushright} % Rata kanan
		\begin{tabular}{l@{}}
			Banda Aceh, 17 Maret 2025 \\[-0.15cm]
			Yang menyatakan, \\
			\\[1.5cm] % Jarak untuk tanda tangan
			Birrul Walidain \\[-0.15cm]
			NPM. 2008102010010 \\
		\end{tabular}
	\end{flushright}
\end{onehalfspace}


\spernyataan % Note: \preface JANGAN DIHAPUS!
\makeatletter
\noindent
Saya yang bertanda tangan di bawah ini,
\vspace{-0.1cm}
\begin{table}[H]
{\renewcommand{\arraystretch}{0.7}
\begin{tabular}{M{0.6cm}ll}
	& Nama   		&: \@fullname \\
	& NPM       	&: \@idnum \\
	& Judul 		&: \@judul \\
	& Fakultas		&: MIPA \\
	& Departemen 	&: \@dept \\
	& Program Studi &: \@prodi \\

\end{tabular}
}
\end{table}
\onehalfspacing
\vspace{-0.4cm}
\noindent
menyetujui: \\ 
( ) untuk mengunggah softcopy Tugas Akhir/Tesis saya diatas secara \textbf{penuh} atau \textit{\textbf{full-text}} di Repository Perpustakaan USK dan diakses atau terbaca lewat mesin pencari internet secara publik \\
( ) untuk mengunggah \textit{softcopy} Tugas Akhir/Tesis kami diatas secara \textbf{parsial} atau \textbf{hanya bagian Tertentu} (misalnya \textbf{Cover, Lembar Pengesahan, Abstrak, Daftar Isi, Pendahuluan dan Kesimpulan}) ke Repository Perpustakaan Unsyiah dan diakses atau terbaca lewat mesin pencari internet secara publik \\
( ) untuk \textbf{tidak membolehkan} sama sekali bagian Tugas Akhir/Tesis diakses online sampai dengan batas waktu 5 tahun dari tanggal kelulusan dengan alasan:.....
\vspace{0.4cm}
\noindent
dan setuju memberikan atau mengunggah softcopy Tugas Akhir/Tesis kepada Perpustakaan USK secara penuh dalam bentuk PDF untuk disimpan secara private ke dalam Repository yang tidak bisa dijangkau oleh mesin pencari (mis. Google).

\vspace{0.4cm}
\noindent
Demikian pernyataan ini dibuat untuk dipergunakan sebagaimana mestinya

\vspace{0.4cm}
{\renewcommand{\arraystretch}{0.8}
\begin{tabular}{p{7.5cm}l}
	&Banda Aceh, 15 Juli 2025 \\
	[-0.1cm]
	&Yang menyatakan \\
	[1.5cm]
	&\underline{\@fullname}\\
	&NPM. \@idnum 
\end{tabular}

\centering
\begin{tabular}{lll}
	Mengetahui, && \\
	[-0.1cm]
	Pembimbing &&Koordinator\\
	&&Program Studi \\
	[1.5cm]
	\underline{\@firstsupervisor}&&\underline{\@kaprodi}\\
	NIP. \@firstnip&&NIP. \@kaprodinip
\end{tabular}
}



\makeatother

\begin{abstractind}

Spektrum emisi merupakan cahaya yang terpancar dari elektron yang bertransisi antar tingkat energi atom atau ion. Struktur tingkat energi pada masing masing atom atau ion membentuk pola unik yang memungkinkan analisis komposisi material secara presisi dalam metode \textit{Laser-Induced Breakdown Spectroscopy} (LIBS). Namun akurasi metode ini secara konvensional seringkali terhambat oleh kompleksitas spektrum emisi multi-elemen, efek matriks dan fluktuasi plasma. Penelitian ini mengembangkan pendekatan analisis berbasis \textit{Deep Learning} (DL) yang mempelajari Spektrum Sintetis tersimulasi. Spektrum sintetis disimulasikan dengan Persamaan Saha dan Distribusi Boltzmann yang kemudian di interpolasikan oleh profil garis gaussian dengan asumsi dalam keadaan \textit{Local Thermal Equilibrium} (LTE) dengan menggunakan data transisi atom dari \textit{Atomic Spectra Database} (ASD) oleh \textit{National Institute Standards and Technology} (NIST) LIBS. Model Informer merupakan arsitektur Transformer yang dioptimalkan dengan \textit{ProbSparse Self-Attention, Self-Attention Distilling,} dan \textit{Generative-style Decoder} dipilih untuk mengatasi spektrum emisi yang relatif panjang. Model dilatih dan diukur menggunakan metrik \textit{Mean Squared Error} (MSE)  untuk mencapai akurasi tinggi dengan diversitas spektrum sintetis. Pendekatan ini berpotensi meningkatkan akurasi deteksi multi-elemen dalam metode LIBS, serta berkontribusi bagi kemajuan spektroskopi modern.

\bigskip
\noindent
\textbf{Kata kunci :} LIBS, Informer, Spektrum Sintetis, Pembelajaran Mendalam, Analisis Prediktif, ProbSparse Self-Attention.
\end{abstractind}

 
\begin{abstracteng}
\textit{Artificial Neural Networks (ANN) bring important insights to the development of Laser-induced Breakdown Spectroscopy methods in qualitatively identifying spectra, especially soil spectra. Volcanic soils were highly fertile which played a major role in the development of the agrarian industry in Indonesia. Thus, the need for rapid identification techniques to analyze major elements in volcanic soil is the answer to the massive volcanic soil in Indonesia. Long Short-Term Memory (LSTM) was an enhanced variant of conventional Recurrent Neural Network (RNN). In this research, the LSTM model was trained with simulated pure element spectra. These simulated pure elemental spectra were built from various elements with plasma state variance. The model was also built to consider the matrix effect present in experiment. The model was then fine-tuned with spectra of Mount Seulawah Agam volcanic soil taken from three sub-districts: Seulimum, Cot Glie, and Lembah Seulawah with each at a depth of 20cm, 40cm, and 60cm from the ground surface. The LSTM model then successfully predicted several major elements such as Calcium, Aluminium, Iron, Magnesium, and Silica accurately and compared with the X-ray fluorescence (XRF) analysis method technique. This research makes an important contribution to the use of ANN for geochemical analysis of volcanic activity in Indonesia for further utilization.
}

\bigskip
\noindent
\textbf{Kata kunci :} lorem, ipsum, dolor, sit, amet
\end{abstracteng} 

% Kata pengantar
\preface % Note: \preface JANGAN DIHAPUS!


Segala puji dan syukur kehadiran Allah SWT yang telah melimpahkan rahmat dan hidayah-Nya kepada kita semua, sehingga penulis dapat menyelesaikan penulisan Tugas Akhir yang berjudul \textbf{{\MakeUppercase{Identifikasi Multi Elemen pada Spektrum Emisi LIBS Kompleks Dari Tanah Vulkanik Seulawah Agam Menggunakan Algoritma \textit{Long Short-Term Memory} (LSTM)}}} yang telah dapat diselesaikan sesuai rencana. Penulis banyak mendapatkan berbagai pengarahan, bimbingan, dan bantuan dari berbagai pihak. Oleh karena itu, melalui tulisan ini penulis mengucapkan rasa terima kasih kepada:

\begin{enumerate} 

	\item {Bapak Dr. Saumi Syahreza, S.Si., M.Si. selaku Ketua Departemen Fisika Fakultas MIPA Universitas Syiah Kuala.}

	\item{Bapak Prof. Dr. Eng. Nasrullah, S.Si., M.T.
 selaku Dosen Pembimbing I yang telah banyak memberikan bimbingan dan arahan kepada penulis, sehingga penulis dapat menyelesaikan Tugas Akhir ini.}
	\item{Bapak Dr. Khairun Saddami, S.T.
 selaku Dosen Pembimbing II yang telah banyak memberikan bimbingan dan arahan kepada penulis, sehingga penulis dapat menyelesaikan Tugas Akhir ini.}

	\item{Bapak Dr. Kurnia Lahna, M.T. selaku Dosen Wali yang telah membimbing
dan memberikan motivasi kepada penulis selama masa perkuliahan.}
  	\item{Ayah dan Ibu sebagai kedua orang tua penulis yang senantiasa selalu mendukung aktivitas dan kegiatan yang penulis lakukan baik secara moral maupun material serta menjadi motivasi terbesar bagi penulis untuk menyelesaikan Tugas Akhir ini.}
	\item{Seluruh Dosen di Departemen Fisika Fakultas MIPA atas ilmu dan didikannya selama perkuliahan.}
	\item{Sahabat dan teman-teman seperjuangan Departemen Fisika USK 2020 lainnya.}
\end{enumerate}

%\vspace{1.5cm}

Penulis juga menyadari segala ketidaksempurnaan yang terdapat didalamnya baik dari segi materi, cara, ataupun bahasa yang disajikan. Seiring dengan ini penulis mengharapkan kritik dan saran dari pembaca yang sifatnya dapat berguna untuk kesempurnaan Tugas Akhir ini. Harapan penulis semoga tulisan ini dapat bermanfaat bagi banyak pihak dan untuk perkembangan ilmu pengetahuan.

\vspace{1cm}


\begin{tabular}{p{7.5cm}l}
	&Banda Aceh, 14 Agustus 2024\\
	&\\
	&\\
	&\multirow{1.5}{7.5cm}{\underline{Birrul Walidain}} \\ 
	&NPM. 2008102010010 \\
\end{tabular}


\titleformat{\section}{\normalfont\bfseries\uppercase}{\thesection}{1.7em}{}
\titleformat{\subsection}{\normalfont\bfseries}{\thesubsection}{0.9em}{}

\addcontentsline{toc}{chapter}{Daftar Isi}
\begin{singlespace}
  \tableofcontents
\end{singlespace}
\listoftables
\addcontentsline{toc}{chapter}{Daftar Tabel}
\listoffigures
\addcontentsline{toc}{chapter}{Daftar Gambar}
\listofappendices
\addcontentsline{toc}{chapter}{Daftar Lampiran}

\enableboldchapterintoc

\singkatan
\begin{itemize}[itemsep=0pt, parsep=0pt]
    \item[$A_{ki}$] : Probabilitas transisi spontan (Koefisien Einstein A).
    \item[$\mathcal{A}$] : Kamus (dictionary) himpunan bagian atom.
    \item[$\alpha_G$] : Lebar profil garis Gaussian (Doppler).
    \item[$\alpha_L$] : Lebar profil garis Lorentzian (Stark/Pressure).
    \item[$\approx$] : Simbol kira-kira sama dengan.
    \item[$c$] : Kecepatan cahaya dalam vakum.
    \item[$\mathcal{C}$] : Himpunan atom kandidat.
    \item[$\mathcal{D}$] : Dataset transisi atom.
    \item[$\Delta$] : Simbol selisih atau perbedaan (Delta).
    \item[$\partial$] : Operator turunan parsial.
    \item[$E$] : Energi, khususnya tingkat energi atom ($E_i, E_k$).
    \item[$\exp()$] : Fungsi eksponensial.
    \item[$f$] : Fraksi populasi ($f_{\text{ion}}, f_{\text{neutral}}$).
    \item[$g$] : Berat statistik atau degenerasi tingkat energi.
    \item[$h$] : Konstanta Planck.
    \item[$I$] : Intensitas garis spektral.
    \item[$\mathcal{I}$] : Kamus intensitas sementara.
    \item[$\int$] : Operator integral.
    \item[$k_B$] : Konstanta Boltzmann.
    \item[$\lambda$] : Panjang gelombang transisi spektral ($\lambda_{ij}$).
    \item[$\log()$] : Fungsi logaritma.
    \item[$m_e$] : Massa elektron.
    \item[$n_e$] : Kerapatan elektron (electron density).
    \item[$N$] : Jumlah atau kerapatan partikel ($N_{\text{ion}}, N_{\text{neutral}}$).
    \item[$\nabla$] : Operator Nabla (del).
    \item[$\propto$] : Simbol sebanding dengan.
    \item[$\mathcal{R}$] : Kamus rasio populasi.
    \item[$\rho$] : Kerapatan (densitas).
    \item[$\mathcal{S}$] : Himpunan data spektrum final.
    \item[$\sigma$] : Tampang lintang (cross-section) atau deviasi standar.
    \item[$\sqrt{}$] : Akar kuadrat.
    \item[$\sum$] : Operator penjumlahan (Sigma).
    \item[$T$] : Temperatur plasma atau gas.
    \item[$Z$] : Fungsi partisi.
\end{itemize}


\renewcommand{\thelstlisting}{\arabic{chapter}.\arabic{lstlisting}}
\renewcommand*\lstlistingname{Program}

\begin{onehalfspace}
\fancyhf{} 
\fancyfoot[C]{\thepage}
\pagenumbering{arabic}

\captionsetup[figure]{labelfont={normalfont}, textfont={normalfont}}
\captionsetup[table]{labelfont={normalfont}, textfont={normalfont}}

\fancyhf{} 
\fancyfoot[R]{\thepage}

\chapter{PENDAHULUAN}

\section{Latar Belakang}

Spektroskopi Emisi Atomik Plasma Hasil Laser (\textit{Laser-Induced Breakdown Spectroscopy} - LIBS) merupakan teknik analisis komposisi unsur yang telah mapan, dikenal karena kemampuannya yang cepat, bersifat non-destruktif, dan memerlukan preparasi sampel yang minimal \parencite{Thorne1999, Cremers2013}. Metode ini bekerja dengan memfokuskan pulsa laser berenergi tinggi ke permukaan sampel untuk mengablasi sejumlah kecil material dan menghasilkan plasma. Plasma yang mendingin memancarkan cahaya pada panjang gelombang diskrit yang sesuai dengan transisi tingkat energi elektronik dari atom dan ion yang ada. Spektrum emisi yang dihasilkan berfungsi sebagai ``sidik jari'' spektral, yang memungkinkan identifikasi dan kuantifikasi kualitatif maupun kuantitatif unsur-unsur dalam sampel \parencite{Harmon2021}.

Meskipun memiliki keunggulan praktis yang signifikan, akurasi dan penerapan LIBS dihadapkan pada beberapa tantangan fundamental. Spektrum LIBS memiliki kompleksitas yang sangat tinggi, sering kali mengandung ribuan garis emisi dari berbagai unsur yang dapat tumpang-tindih dan menimbulkan interferensi spektral yang parah. Sinyal LIBS menunjukkan variabilitas yang signifikan akibat fluktuasi stokastik dalam kondisi plasma (misalnya, suhu dan densitas elektron) dari satu tembakan laser ke tembakan berikutnya. Kemudian efek matriks, di mana sifat fisik dan kimia keseluruhan sampel (matriks) memengaruhi emisi dari unsur target, sehingga menyulitkan analisis kuantitatif yang akurat \parencite{Gaudiuso2023}. Akibatnya, metode LIBS konvensional sangat bergantung pada proses kalibrasi yang ekstensif, yang memerlukan penggunaan sampel standar bersertifikat (\textit{Certified Reference Materials} - CRMs). Proses ini tidak hanya memakan waktu tetapi juga sangat mahal dan sering kali tidak praktis, terutama untuk matriks sampel yang kompleks atau langka \parencite{Porizka2018}.

Dalam beberapa tahun terakhir, metode berbasis \textit{Deep Learning} (DL) telah muncul sebagai pendekatan yang menjanjikan untuk mengatasi kompleksitas data spektral LIBS. Sebagai contoh, penelitian oleh \textcite{Yang2025} menunjukkan bahwa arsitektur \textit{Transformer} dapat secara efektif memodelkan hubungan kompleks dalam spektrum LIBS untuk kuantifikasi Lantanum (La), Cerium (Ce), dan Neodymium (Nd) dalam bijih tanah jarang. Namun, pendekatan-pendekatan ini umumnya masih bergantung pada dataset eksperimental yang besar untuk melatih model, yang tidak sepenuhnya menghilangkan beban pengumpulan data dan ketergantungan pada sampel standar.

Untuk mengatasi limitasi tersebut, penelitian ini mengusulkan sebuah pendekatan inovatif yang bertujuan untuk menciptakan metode analisis LIBS yang sepenuhnya \textbf{bebas kalibrasi} (\textit{calibration-free}). Inovasi inti terletak pada pelatihan eksklusif model \textit{Deep Learning} pada set data spektrum emisi sintetis yang besar dan beragam. Dengan cara ini, model belajar untuk mengenali ``sidik jari'' spektral fundamental dari setiap elemen, terlepas dari variasi yang disebabkan oleh efek matriks dan fluktuasi plasma, sehingga menghilangkan kebutuhan akan kalibrasi eksperimental.

Pendekatan kami memanfaatkan dua komponen utama: simulasi spektrum berbasis fisika dan arsitektur \textit{Deep Learning} yang efisien. Spektrum sintetis dihasilkan berdasarkan prinsip-prinsip pertama fisika plasma. Dengan asumsi kondisi kesetimbangan termodinamika lokal (\textit{Local Thermodynamic Equilibrium} - LTE), populasi tingkat ionisasi dihitung menggunakan \textbf{Persamaan Saha}, dan distribusi populasi tingkat energi dihitung menggunakan \textbf{Distribusi Boltzmann} \parencite{Chandrasekhar1939, Panne2024}. Data transisi atom, termasuk panjang gelombang, probabilitas transisi, dan tingkat energi, diperoleh dari basis data spektrum atomik yang dikelola oleh \textit{National Institute of Standards and Technology} (NIST ASD). Profil garis emisi dimodelkan menggunakan fungsi pelebaran (misalnya, Gaussian atau Voigt) untuk mereplikasi spektrum eksperimental secara realistis \parencite{Miziolek2006}.

Sebagai arsitektur model, kami mengadopsi model \textbf{Informer}, sebuah varian \textit{Transformer} yang sangat efisien dan dirancang untuk memproses sekuens data yang panjang \parencite{Zhou2021}. Pemilihan ini didasarkan pada kemampuannya untuk menangani data spektral LIBS beresolusi tinggi secara efisien. \textit{Informer} memperkenalkan mekanisme \textit{ProbSparse Self-Attention}, yang mengurangi kompleksitas komputasi dari $O(L^2)$ menjadi $O(L \log L)$, di mana $L$ adalah panjang sekuens. Fitur tambahan seperti \textit{self-attention distilling} secara progresif mengurangi panjang sekuens di lapisan yang lebih dalam, menghemat memori dan komputasi, sementara \textit{decoder} bergaya generatif memungkinkan prediksi output dalam satu langkah maju, mempercepat proses inferensi secara signifikan \parencite{Zhou2021}.

Dengan melatih arsitektur \textit{Informer} pada set data sintetis yang komprehensif, penelitian ini bertujuan untuk mengembangkan sistem analisis LIBS yang tidak hanya akurat dan tangguh (\textit{robust}) terhadap variabilitas sinyal, tetapi juga sepenuhnya independen dari kalibrasi eksperimental. Pendekatan ini berpotensi merevolusi aplikasi LIBS di berbagai bidang—mulai dari geologi dan metalurgi hingga pemantauan lingkungan—dengan menyediakan solusi analisis unsur yang cepat, hemat biaya, dan dapat diskalakan secara luas.



\section{Rumusan Masalah}
\par Ketergantungan pada kalibrasi eksperimental menjadi penghalang utama dalam penerapan LIBS. Sebuah metode yang memanfaatkan spektrum sintetis dan model \textit{Deep Learning} yang efisien seperti \textit{Informer} dengan mekanisme \textit{ProbSparse Self-Attention} diajukan untuk mengatasi masalah ini. Oleh karena itu, rumusan masalah penelitian ini adalah:
\begin{enumerate}
    \item Bagaimana sebuah metode yang menggunakan model \textit{Informer} dan dilatih sepenuhnya pada spektrum sintetis dapat mencapai akurasi prediksi multi-elemen yang tinggi tanpa melalui proses kalibrasi eksperimental?
    \item Seberapa efisien model \textit{Informer} dalam memproses spektrum LIBS resolusi tinggi, diukur dari segi waktu komputasi, untuk memastikan kepraktisan metode yang diusulkan?
    \item Bagaimana pengaruh jumlah dan variasi data spektrum sintetis terhadap kemampuan generalisasi dan ketahanan (\textit{robustness}) metode yang diusulkan dalam menghadapi spektrum uji dengan kondisi yang beragam?
\end{enumerate}

\section{Tujuan Penelitian}
Berdasarkan rumusan masalah, tujuan penelitian ini adalah:
\begin{enumerate}
    \item Mengembangkan sebuah metode analisis LIBS yang mengintegrasikan model \textit{Informer} dengan mekanisme \textit{ProbSparse Self-Attention}, yang dilatih secara eksklusif menggunakan spektrum emisi sintetis yang disimulasikan dari parameter fisis fundamental.
    \item Memvalidasi akurasi metode yang dikembangkan untuk prediksi multi-elemen pada data spektrum uji untuk membuktikan kemampuannya beroperasi secara akurat tanpa kalibrasi eksperimental.
    \item Mengevaluasi efisiensi komputasi dan menganalisis pengaruh variasi data latih sintetis terhadap kemampuan generalisasi metode untuk memastikan ketahanannya.
\end{enumerate}

\section{Batasan Penelitian}
Penelitian ini berfokus pada simulasi spektrum atom untuk elemen terpilih pada varian suhu plasma dan densitas elektron, menggunakan distribusi Boltzmann dan persamaan Saha. Penelitian tidak mencakup pengukuran spektrum LIBS eksperimental atau analisis efek matriks dan fluktuasi plasma, untuk menjaga fokus pada pengembangan dan evaluasi metode berbasis model \textit{Informer} yang dilatih secara sintetis.

\section{Manfaat Penelitian}
Penelitian ini diharapkan dapat menghasilkan sebuah metode analisis LIBS baru yang inovatif. Manfaat utamanya adalah menghilangkan ketergantungan pada proses kalibrasi yang mahal dan memakan waktu, sehingga menyediakan metode analisis material yang lebih praktis, efisien, dan dapat diakses secara luas. Model yang dikembangkan berpotensi meningkatkan akurasi dan kecepatan deteksi elemen, baik mayor maupun minor.


%-------------------------------------------------------------------------------
%                            BAB II
%               TINJAUAN PUSTAKA DAN DASAR TEORI
%-------------------------------------------------------------------------------

\chapter{TINJAUAN PUSTAKA}
\par Dalam bab ini, berbagai teori yang relevan dan literatur sebelumnya yang menunjang penelitian ini dibahas secara komprehensif. Pembahasan mencakup teknik spektroskopi yang diaplikasikan, metode analisis spektrum, serta penerapan teknik statistik dalam pemrosesan data spektroskopi dengan penekanan pada penerapan \textit{Deep Learning} di dalamnya. Arsitektur \textit{Long Short-Term Memory} (LSTM) diterapkan untuk memprediksi elemen dalam sampel tersebut.

\section{Spektroskopi Emisi Atom}
\subsection{Prinsip Dasar}
\label{subsec:prinsip-dasar}

Spektroskopi emisi atom bertumpu pada teori atom Bohr yang menjelaskan kuantisasi energi elektron melalui persamaan:
\begin{equation}
\label{eq:energi-bohr}
E_n = -\frac{13.6 \, \text{eV}}{n^2} \quad (n \in \mathbb{Z}^+),
\end{equation}
di mana transisi elektron antar tingkat energi memenuhi $\Delta E = h\nu$ \citep{Beiser1992}. Fenomena spektrum garis hidrogen mengikuti persamaan umum Rydberg:
\begin{equation}
\label{eq:rydberg-umum}
\frac{1}{\lambda} = R \left( \frac{1}{n_f^2} - \frac{1}{n_i^2} \right) \quad (n_i > n_f),
\end{equation}
dengan $R = 1.097 \times 10^7 \, \text{m}^{-1}$ sebagai konstanta Rydberg \citep{Beiser1992}. Contoh spesifik seperti deret Balmer ($n_f = 2$) dan Lyman ($n_f = 1$) menunjukkan konsistensi persamaan ini \citep{Griffiths2005}.

Mekanisme eksitasi terjadi ketika atom menyerap energi dari sumber eksternal (e.g., plasma), menyebabkan elektron berpindah ke tingkat energi lebih tinggi. De-eksitasi menghasilkan emisi foton dengan panjang gelombang:
\[
\lambda = \frac{hc}{\Delta E},
\]
di mana $h$ adalah konstanta Planck dan $c$ kecepatan cahaya. Aturan seleksi $\Delta l = \pm 1$ \citep{Liboff2003} membatasi transisi yang diperbolehkan, menghasilkan pola spektrum unik untuk setiap unsur.

Aplikasi praktis meliputi:
\begin{itemize}
\item Pemantauan polutan udara via spektroskopi laser resolusi tinggi \citep{Demtroder2010},
\item Analisis komposisi bintang menggunakan spektrograf astronomi \citep{Kaufmann2020},
\item Deteksi logam berat dengan teknik LIBS (\textit{Laser-Induced Breakdown Spectroscopy}).
\end{itemize}

Intensitas garis spektrum bergantung pada probabilitas transisi dan populasi elektron yang dapat dimodelkan dengan distribusi Boltzmann \citep{Demtroder2010}.
\subsubsection{Mekanisme eksitasi dan de-eksitasi atom.}
\subsubsection{Panjang gelombang emisi dan hubungannya dengan identifikasi unsur.}


\subsection{Hukum Beer-Lambert}

\subsubsection{Hukum Beer-Lambert dan aplikasinya dalam analisis kuantitatif.}
\subsubsection{Absorbansi, transmitansi, dan konsentrasi analit.}
\subsubsection{Keterbatasan Hukum Beer-Lambert.}


\subsection{Instrumentasi Spektroskopi Emisi}

\subsubsection{Sumber eksitasi (misalnya, plasma, api, arc).}
\subsubsection{Sistem dispersi cahaya (misalnya, prisma, kisi difraksi).}
\subsubsection{Detektor (misalnya, \textit{photomultiplier tube} (PMT), \textit{charge-coupled device} (CCD)).}
\subsubsection{Jenis-jenis spektrometer (misalnya, \textit{Czerny-Turner}, \textit{Echelle}) dan prinsip kerjanya.}


\section{Laser-Induced Breakdown Spectroscopy (LIBS)}

\subsection{Prinsip Dasar LIBS}

\subsubsection{Skema LIBS dan tahapan-tahapan dalam analisis LIBS.}
\subsubsection{Interaksi laser-materi: proses absorpsi energi dan peningkatan suhu.}
\subsubsection{Mekanisme ablasi laser: \textit{thermal vaporization}, \textit{photochemical decomposition}, \textit{photophysical sputtering}.}
\subsubsection{Formasi plasma: ionisasi atom dan molekul, pembentukan spesies tereksitasi.}
\subsubsection{Emisi atomik: de-eksitasi radiatif dan emisi foton pada panjang gelombang karakteristik.}

\subsubsection{Simulasi Spektrum Emisi:}

\par Penjelasan tentang bagaimana spektrum emisi dihasilkan dari plasma.
\par Persamaan intensitas emisi:  $I_{ij} = A_{ij} \cdot N_i \cdot h \cdot \nu_{ij}$,  dimana $I_{ij}$ adalah intensitas emisi, $A_{ij}$ adalah probabilitas transisi, $N_i$ adalah populasi atom pada tingkat energi i, $h$ adalah konstanta Planck, dan  $\nu_{ij}$ adalah frekuensi transisi.
\par Contoh simulasi spektrum emisi dengan Manim:


\subsubsection{Akuisisi dan analisis spektrum: identifikasi unsur dan kuantifikasi konsentrasi.}


\subsection{Karakteristik Plasma LIBS}

\subsubsection{Parameter plasma: suhu, densitas elektron, komposisi spesies.}
\subsubsection{Kesetimbangan Termodinamika Lokal (LTE): definisi dan kriteria pemenuhan.}
\subsubsection{Efek Stark: pergeseran dan pelebaran garis spektral akibat medan listrik dalam plasma.}
\subsubsection{Efek matriks: pengaruh komposisi sampel terhadap proses ablasi dan emisi.}


\subsection{Laser Nd:YAG dalam LIBS}

\subsubsection{Prinsip kerja laser Nd:YAG: media aktif, rongga resonator, proses \textit{pumping}.}
\subsubsection{Karakteristik laser Nd:YAG: panjang gelombang (1064 nm, 532 nm, 355 nm, 266 nm), energi pulsa, durasi pulsa (nanosecond, picosecond, femtosecond), mode operasi (\textit{Q-switched}, \textit{mode-locked}).}

\subsubsection{Diagram Grotrian Nd:YAG:}


\par Tingkat energi ion Nd$^{3+}$ dalam kristal YAG (4F$_{3/2}$, 4I$_{11/2}$, dll.).
\par Transisi elektronik yang menghasilkan emisi laser pada 1064 nm (4F$_{3/2}$ $\rightarrow$ 4I$_{11/2}$).
\par Mekanisme \textit{pumping} dan proses relaksasi non-radiatif.

\subsubsection{Keuntungan menggunakan laser Nd:YAG dalam LIBS: ketersediaan, keandalan, fleksibilitas panjang gelombang.}
\subsubsection{Pertimbangan dalam menggunakan laser Nd:YAG untuk CF-LIBS: validitas asumsi LTE.}


\subsection{Calibration-Free LIBS (CF-LIBS)}

\subsubsection{Motivasi CF-LIBS: mengatasi keterbatasan kalibrasi standar (efek matriks, kesulitan mendapatkan standar yang sesuai).}
\subsubsection{Prinsip dasar CF-LIBS: penggunaan persamaan Boltzmann dan Saha untuk menghubungkan intensitas emisi dengan konsentrasi unsur.}
\subsubsection{Metode CF-LIBS: \textit{one-line}, \textit{multi-line}, \textit{iterative}.}
\subsubsection{Asumsi LTE dalam CF-LIBS dan validitasnya pada plasma yang dihasilkan oleh laser Nd:YAG.}
\subsubsection{Keunggulan CF-LIBS: analisis kuantitatif tanpa standar, kecepatan, fleksibilitas.}
\subsubsection{Tantangan CF-LIBS: akurasi pengukuran parameter plasma, efek \textit{self-absorption}.}


\subsection{Aplikasi LIBS}

\subsubsection{Contoh aplikasi LIBS di berbagai bidang:}

\par Industri: kontrol kualitas, analisis logam dan paduan, identifikasi material.
\par Lingkungan: pemantauan polusi, analisis tanah dan air, karakterisasi limbah.
\par Kedokteran: analisis jaringan biologis, deteksi kanker, diagnosis penyakit.
\par Arkeologi: analisis artefak, identifikasi pigmen, penentuan asal-usul material.
\par Forensik: analisis material bukti, identifikasi bahan peledak, analisis \textit{gunshot residue}.
\section{Model Long Short-Term Memory (LSTM)}
\par Long Short-Term Memory (LSTM) adalah jenis jaringan saraf yang dirancang untuk mengatasi tantangan dalam memproses dan memprediksi urutan data. LSTM sangat efektif dalam menangkap dependensi jangka panjang dalam data sekuensial, yang membuatnya berguna dalam berbagai aplikasi, termasuk analisis deret waktu, pemrosesan bahasa alami, dan pengenalan suara. Dengan kemampuannya untuk mengingat informasi dalam jangka waktu yang lebih lama, LSTM telah menjadi alat penting dalam bidang pembelajaran mesin \citep{hochreiter1997}.
\par LSTM telah terbukti menjadi alat yang sangat efektif dalam berbagai aplikasi pembelajaran mesin. Dengan kemampuannya untuk menangkap dependensi jangka panjang dalam data sekuensial, LSTM telah membuka banyak peluang baru dalam analisis data dan pengembangan model prediktif. Meskipun ada tantangan yang perlu diatasi, potensi LSTM dalam berbagai bidang menjadikannya salah satu teknik yang paling menarik dalam pembelajaran mesin \citep{graves2013}.

\par Bagi para praktisi yang ingin mempelajari lebih lanjut tentang LSTM, disarankan untuk membaca literatur tambahan dan melakukan eksperimen dengan model. Menggunakan pustaka seperti Keras dapat mempermudah proses pengembangan dan pelatihan model. Selain itu, mengikuti perkembangan terbaru dalam penelitian LSTM dapat memberikan wawasan baru dan teknik yang lebih baik.

\section{Struktur dan Mekanisme LSTM}
\par Struktur LSTM terdiri dari beberapa komponen kunci, termasuk sel memori dan tiga pintu: pintu input, pintu lupa, dan pintu output. Pintu input mengontrol informasi baru yang masuk ke dalam sel memori, pintu lupa menentukan informasi mana yang harus dihapus, dan pintu output mengatur informasi yang akan dikeluarkan dari sel memori. Mekanisme ini memungkinkan LSTM untuk mempertahankan informasi yang relevan dan mengabaikan informasi yang tidak penting, sehingga meningkatkan kinerja model dalam memprediksi urutan data \citep{graves2013}.

\par Dalam analisis deret waktu, LSTM digunakan untuk memprediksi nilai masa depan berdasarkan data historis. Misalnya, dalam prediksi harga saham, model LSTM dapat dilatih menggunakan data harga historis untuk memprediksi harga di masa depan. Keunggulan LSTM dalam menangkap pola temporal membuatnya lebih unggul dibandingkan model tradisional, seperti regresi linier, yang sering kali gagal menangkap hubungan jangka panjang dalam data.

\section{Aplikasi LSTM}
\par LSTM juga banyak digunakan dalam pemrosesan bahasa alami (NLP), di mana urutan kata dalam kalimat sangat penting. Dalam tugas-tugas seperti penerjemahan bahasa dan analisis sentimen, LSTM dapat digunakan untuk memahami konteks dan makna dari urutan kata. Dengan memanfaatkan kemampuan LSTM untuk mengingat informasi dari kata-kata sebelumnya, model dapat menghasilkan terjemahan yang lebih akurat dan analisis sentimen yang lebih tepat \citep{zhang2019}.

\par Dalam pengenalan suara, LSTM digunakan untuk mengubah sinyal audio menjadi teks. Model ini dilatih menggunakan data audio yang telah dilabeli untuk mengenali pola dalam suara dan menghasilkan transkripsi yang akurat. Dengan kemampuan LSTM untuk menangkap informasi temporal, model dapat mengenali kata-kata dalam konteks yang lebih luas, meningkatkan akurasi pengenalan suara.

\section{Pelatihan dan Evaluasi Model LSTM}
\par Pelatihan model LSTM melibatkan penggunaan algoritma optimasi untuk meminimalkan fungsi kerugian. Proses ini biasanya dilakukan dengan menggunakan teknik backpropagation melalui waktu (BPTT), yang memungkinkan model untuk memperbarui bobot berdasarkan kesalahan prediksi. Pemilihan hyperparameter, seperti jumlah unit LSTM dan tingkat dropout, juga sangat penting untuk mencapai kinerja optimal \citep{bengio2012}.

\par Evaluasi model LSTM dilakukan dengan menggunakan metrik seperti akurasi, presisi, dan recall, tergantung pada jenis masalah yang dihadapi. Untuk masalah regresi, metrik seperti Mean Squared Error (MSE) sering digunakan. Evaluasi yang tepat sangat penting untuk memastikan bahwa model dapat diandalkan dan memberikan hasil yang akurat pada data baru.



\subsection{Input Layer}
\par Input layer adalah komponen pertama dalam arsitektur LSTM yang bertanggung jawab untuk menerima data masukan. Dalam konteks analisis spektrum, data yang dimasukkan biasanya berupa serangkaian nilai intensitas yang diukur pada berbagai panjang gelombang. Data ini perlu diubah menjadi format tiga dimensi, yaitu (samples, timesteps, features), agar dapat diproses oleh model LSTM. Misalnya, jika kita memiliki 1000 sampel, 10 timestep, dan 5 fitur, maka bentuk inputnya adalah:
$$
\text{Input Shape} = (1000, 10, 5)
$$
Pengaturan yang tepat dari input layer sangat penting untuk memastikan bahwa model dapat belajar dari data dengan efektif \cite{hochreiter1997}.

\par Selain itu, input layer juga berfungsi untuk menstandarisasi dan menormalisasi data sebelum diproses lebih lanjut. Proses ini penting untuk menghindari masalah yang dapat muncul akibat skala data yang berbeda, yang dapat mempengaruhi kinerja model. Normalisasi data dapat dilakukan dengan berbagai metode, seperti Min-Max Scaling atau Z-score Normalization. Dengan memastikan bahwa data berada dalam rentang yang sesuai, model dapat belajar dengan lebih cepat dan efisien, serta mengurangi risiko konvergensi yang lambat selama pelatihan.

\par Konfigurasi untuk input layer dalam kode Python menggunakan Keras dapat dituliskan sebagai berikut:
\begin{minipage}{\textwidth}
\begin{verbatim}
model.add(InputLayer(input_shape=(timesteps, features)))
\end{verbatim}
\end{minipage}

\subsection{LSTM Layer}
\par LSTM layer adalah inti dari model ini, yang dirancang untuk menangkap dependensi temporal dalam data sekuensial. Layer ini terdiri dari unit-unit LSTM yang memiliki kemampuan untuk menyimpan informasi dalam jangka waktu yang lebih lama dibandingkan dengan jaringan saraf tradisional. Setiap unit LSTM memiliki tiga pintu: pintu input, pintu lupa, dan pintu output, yang mengatur aliran informasi ke dalam dan keluar dari sel memori. Dengan mekanisme ini, LSTM dapat mengatasi masalah vanishing gradient yang sering terjadi pada jaringan saraf konvensional \cite{graves2013}.

\par Mekanisme pintu dalam LSTM memungkinkan model untuk memutuskan informasi mana yang harus disimpan dan mana yang harus dilupakan. Pintu input mengontrol informasi baru yang masuk ke dalam sel memori, pintu lupa menentukan informasi mana yang harus dihapus dari sel memori, dan pintu output mengatur informasi yang akan dikeluarkan dari sel memori. Dengan cara ini, LSTM dapat mempertahankan informasi yang relevan untuk jangka waktu yang lebih lama, sehingga sangat efektif dalam aplikasi yang memerlukan pemahaman konteks temporal, seperti analisis deret waktu dan pemrosesan bahasa alami.

\par Konfigurasi untuk LSTM layer dalam kode Python menggunakan Keras dapat dituliskan sebagai berikut:
\begin{minipage}{\textwidth}
\begin{verbatim}
model.add(LSTM(64, return_sequences=True, input_shape=(timesteps, features)))
\end{verbatim}
\end{minipage}

\subsection{Dropout Layer}
\par Dropout layer ditambahkan setelah LSTM layer untuk mengurangi risiko overfitting, yang merupakan masalah umum dalam pelatihan model dengan banyak parameter. Layer ini bekerja dengan cara mengabaikan sejumlah neuron secara acak selama proses pelatihan, sehingga model tidak terlalu bergantung pada neuron tertentu. Dengan demikian, dropout membantu meningkatkan generalisasi model terhadap data yang belum pernah dilihat sebelumnya \cite{bengio2012}.

\par Penentuan tingkat dropout yang tepat sangat penting untuk mencapai keseimbangan antara pelatihan yang efektif dan generalisasi yang baik. Biasanya, tingkat dropout berkisar antara 0.2 hingga 0.5, tergantung pada kompleksitas model dan ukuran dataset. Penelitian menunjukkan bahwa penggunaan dropout dapat secara signifikan meningkatkan akurasi model dalam berbagai aplikasi, termasuk analisis spektrum. Dengan mengurangi overfitting, model dapat memberikan prediksi yang lebih akurat dan dapat diandalkan ketika diterapkan pada data baru.

\par Konfigurasi untuk dropout layer dalam kode Python menggunakan Keras dapat dituliskan sebagai berikut:
\begin{minipage}{\textwidth}
\begin{verbatim}
model.add(Dropout(0.3))
\end{verbatim}
\end{minipage}

\subsection{Dense Layer}
\par Dense layer, atau fully connected layer, berfungsi untuk mengolah informasi yang diekstrak oleh LSTM layer dan menghasilkan output akhir dari model. Dalam konteks LSTM, dense layer biasanya digunakan setelah LSTM layer dan dropout layer. Layer ini menghubungkan setiap neuron di layer sebelumnya dengan setiap neuron di layer berikutnya, memungkinkan model untuk belajar dari kombinasi fitur yang lebih kompleks. Dengan cara ini, dense layer dapat menangkap interaksi yang lebih dalam antara fitur-fitur yang ada dalam data \cite{zhang2019}.

\par Fungsi aktivasi yang umum digunakan dalam dense layer adalah \texttt{ReLU} (Rectified Linear Unit) atau \texttt{tanh}. Pemilihan fungsi aktivasi yang tepat dapat mempengaruhi kinerja model secara signifikan. Setelah dense layer, biasanya terdapat satu output layer yang menghasilkan prediksi akhir, baik untuk regresi maupun klasifikasi. Dengan mengoptimalkan arsitektur dense layer, model dapat mencapai kinerja yang lebih baik dalam memprediksi hasil yang diinginkan, sehingga meningkatkan akurasi dan efisiensi dalam analisis data.

\par Konfigurasi untuk dense layer dalam kode Python menggunakan Keras dapat dituliskan sebagai berikut:
\begin{minipage}{\textwidth}
\begin{verbatim}
model.add(Dense(1024, activation='tanh'))
\end{verbatim}
\end{minipage}

\par Simulasi spektrum adalah alat penting dalam analisis spektroskopi, yang memungkinkan peneliti untuk memprediksi intensitas cahaya yang dipancarkan atau diserap oleh suatu substansi pada berbagai panjang gelombang. Kelas \texttt{SpectrumSimulator} dirancang untuk melakukan simulasi ini dengan mempertimbangkan berbagai parameter fisik, termasuk tingkat energi, degenerasi, dan koefisien Einstein.

\section{Struktur Kelas}
\par Kelas \texttt{SpectrumSimulator} memiliki beberapa atribut penting, termasuk \texttt{nist\_data}, \texttt{temperature}, dan \texttt{resolution}. Atribut \texttt{nist\_data} berisi data dari National Institute of Standards and Technology (NIST), yang mencakup informasi tentang tingkat energi dan koefisien transisi. Atribut \texttt{temperature} menentukan suhu dalam Kelvin, dan \texttt{resolution} mengatur resolusi spektrum yang dihasilkan.

\section{Metode Utama}
\par Kelas ini memiliki beberapa metode statis dan non-statis yang berfungsi untuk menghitung berbagai aspek dari simulasi spektrum:

\subsection{Fungsi Partisi}
\par Metode \texttt{partition\_function} menghitung fungsi partisi $$Z$$ berdasarkan tingkat energi dan degenerasi. Fungsi partisi adalah konsep penting dalam statistik termal yang digunakan untuk menghitung probabilitas distribusi keadaan dalam sistem termal. Fungsi ini didefinisikan sebagai:
$$
Z = \sum_{i} g_i e^{-\frac{E_i}{k_B T}}
$$
di mana $$g_i$$ adalah degenerasi, $$E_i$$ adalah tingkat energi, $$k_B$$ adalah konstanta Boltzmann, dan $$T$$ adalah suhu \citep{pathria2011}.

\subsection{Menghitung Intensitas}
\par Metode \texttt{calculate\_intensity} digunakan untuk menghitung intensitas spektrum berdasarkan suhu, energi, degenerasi, dan koefisien Einstein. Intensitas dapat dihitung dengan rumus:
$$
I = \frac{g \cdot e^{-\frac{E}{k_B T}} \cdot A}{Z}
$$
di mana $$I$$ adalah intensitas, $$g$$ adalah degenerasi, $$E$$ adalah energi, $$A$$ adalah koefisien Einstein, dan $$Z$$ adalah fungsi partisi \citep{mason2015}.


\section{Profil Voigt dalam Analisis Spektrum}
\par Profil Voigt adalah gabungan dari dua fungsi profil spektral: profil Lorentzian dan Gaussian. Fungsi Voigt sering digunakan untuk mendeskripsikan bentuk garis spektrum yang diukur dalam berbagai teknik spektroskopi, termasuk Laser Induced Breakdown Spectroscopy (LIBS). Fungsi Voigt dapat didefinisikan sebagai konvolusi dari fungsi Lorentzian dan Gaussian.

\subsection{Profil Lorentzian}
\par Fungsi Lorentzian, \( L(\lambda) \), menggambarkan lebar garis spektrum yang disebabkan oleh efek tekanan atau damping. Dalam konteks panjang gelombang \( \lambda \), fungsi Lorentzian dapat dinyatakan sebagai:
\begin{equation}
L(\lambda) = \frac{\Gamma / 2\pi}{(\lambda - \lambda_0)^2 + (\Gamma / 2)^2}
\end{equation}
di mana \( \lambda \) adalah panjang gelombang yang diukur, \( \lambda_0 \) adalah panjang gelombang pusat dari garis spektrum, dan \( \Gamma \) adalah lebar garis Lorentzian yang berhubungan dengan lebar dari garis spektrum akibat efek tekanan. Dalam hal ini, \( \Gamma \) adalah Full Width at Half Maximum (FWHM) dari profil Lorentzian.

\subsection{Profil Gaussian}
\par Fungsi Gaussian, \( G(\lambda) \), menggambarkan lebar garis spektrum yang disebabkan oleh efek Doppler, yang terkait dengan pergerakan relatif antara sumber spektrum dan detektor. Fungsi Gaussian dalam konteks panjang gelombang \( \lambda \) dinyatakan sebagai:

\begin{equation}
G(\lambda) = \frac{1}{\sqrt{2\pi \sigma^2}} \exp\left(-\frac{(\lambda - \lambda_0)^2}{2\sigma^2}\right)
\end{equation}
di mana \( \sigma \) adalah deviasi standar dari distribusi Gaussian, yang berkaitan dengan lebar dari garis spektrum dalam konteks efek Doppler. Deviasi standar \( \sigma \) berhubungan dengan Half Width at Half Maximum (HWHM) Gaussian, yang dinyatakan sebagai \( b = \sigma \sqrt{2 \ln 2} \).

\subsection{Konvolusi Gaussian dan Lorentzian}
\par Fungsi Voigt, \( V(\lambda; \Gamma, \sigma) \), adalah hasil konvolusi dari fungsi Lorentzian dan Gaussian. Konvolusi ini menghasilkan profil spektrum yang menggabungkan kedua efek. Fungsi Voigt didefinisikan sebagai:
\begin{equation}
V(\lambda; \Gamma, \sigma) = \int_{-\infty}^{\infty} G(\lambda - \lambda') L(\lambda') \, d\lambda'
\end{equation}
di mana \( \Gamma \) adalah lebar garis Lorentzian (FWHM), dan \( \sigma \) adalah deviasi standar Gaussian (HWHM).

\par Fungsi Voigt dapat disederhanakan dengan menggunakan fungsi \textit{W} (fungsi Voigt) yang merupakan integral konvolusi dari fungsi Lorentzian dan Gaussian. Fungsi Voigt dapat dinyatakan sebagai: \citep{Godio2016}
\begin{equation}
V(\lambda; a, b) = \text{Re} \left[ W\left(\frac{\lambda - \lambda_0 + i a}{b}\right) \right]
\end{equation}
di mana \( a \) adalah lebar Lorentzian (FWHM) dan \( b \) adalah deviasi standar Gaussian (HWHM). Fungsi Voigt kompleks, \( W(z) \), didefinisikan sebagai:
\begin{equation}
W(z) = e^{-z^2} \left( 1 + \text{erfi}(z) \right)
\end{equation}
dengan \text{erfi}(z) sebagai fungsi kesalahan kompleks.

\section{Simulasi Spektrum}
\par Metode \texttt{simulate} menggabungkan semua komponen di atas untuk menghasilkan spektrum. Metode ini pertama-tama menginisialisasi panjang gelombang dan intensitas, kemudian mengumpulkan tingkat energi dan degenerasi dari data NIST. Setelah menghitung fungsi partisi, intensitas dihitung untuk setiap transisi dan digabungkan menggunakan profil Gaussian. Hasil akhirnya adalah panjang gelombang dan intensitas yang dinormalisasi.

\par Kelas \texttt{SpectrumSimulator} menyediakan alat yang kuat untuk simulasi spektrum berdasarkan data fisik yang relevan. Dengan menggunakan metode statistik dan fisika dasar, kelas ini memungkinkan peneliti untuk memodelkan perilaku spektrum dengan akurasi yang tinggi. Simulasi ini dapat diterapkan dalam berbagai bidang, termasuk kimia, fisika, dan ilmu material.

%-------------------------------------------------------------------------------
%                            BAB III
%               		METODE PENELITIAN
%-------------------------------------------------------------------------------


\chapter{METODE PENELITIAN}

\section{Waktu dan Lokasi Penelitian}
Penelitian ini dilaksanakan di Laboratorium Gelombang dan Optik, Departemen Fisika, Fakultas Matematika dan Ilmu Pengetahuan Alam, Universitas Syiah Kuala, dari Desember 2024 hingga Juni 2025.

\section{Jadwal Pelaksanaan Penelitian}
Penelitian ini direncanakan berlangsung selama 10 bulan, dari September 2024 hingga Juni 2025. Tabel~\ref{tab:jadwal_penelitian} menunjukkan jadwal pelaksanaan penelitian yang mencakup fase studi literatur, pengumpulan data, pengembangan model, validasi, dan penulisan laporan akhir.

Studi Literatur dan Pembentukan Persamaan
Perancangan Model Simulasi Spektral dan Dataset Pelatihan
Pengembangan Model \textit{Informer} sistem evaluasi dan sistem prediksi
Validasi Hasil Prediksi 
Penulisan  Laporan Akhir

\begin{table}[H]
  \centering
  \caption{Jadwal Penelitian}
  \label{tab:jadwal_penelitian}
  \small
  \begin{tabularx}{\textwidth}{X *{10}{c}}
    \toprule
    \textbf{Fase Penelitian} & \multicolumn{4}{c}{\textbf{2024}} & \multicolumn{6}{c}{\textbf{2025}} \\
    \cmidrule(lr){2-5} \cmidrule(lr){6-11}
    & \textbf{Sep} & \textbf{Okt} & \textbf{Nov} & \textbf{Des} & \textbf{Jan} & \textbf{Feb} & \textbf{Mar} & \textbf{Apr} & \textbf{Mei} & \textbf{Jun} \\
    \midrule
    Studi Literatur dan Pembentukan Persamaan & \cellcolor{gray!20} & \cellcolor{gray!20} & & & & & & & & \\
    Perancangan Model Simulasi Spektral dan Dataset Pelatihan & & & \cellcolor{gray!20} & \cellcolor{gray!20} & & & & & & \\
    Pengembangan Model \textit{Informer} sistem evaluasi dan sistem prediksi & & & & & \cellcolor{gray!20} & \cellcolor{gray!20} & \cellcolor{gray!20} & & & \\
    Validasi Hasil Prediksi & & & & & & & & \cellcolor{gray!20} & \cellcolor{gray!20} & \\
    Penulisan Laporan Akhir & & & & & & & & & \cellcolor{gray!20} & \cellcolor{gray!20} \\
    \bottomrule
  \end{tabularx}
  \vspace{0.2cm}
\end{table}

\section{Alat dan Bahan}
Penelitian ini memanfaatkan berbagai alat dan bahan yang mencakup perangkat keras, perangkat lunak, serta sumber data untuk mendukung proses pengumpulan, pengolahan, analisis, dan visualisasi data spektral atomik. Berikut adalah rincian alat dan bahan yang digunakan:

\subsection{Perangkat Keras}
\begin{enumerate}
  \item \textbf{Laptop \textit{Apple MacBook} Air M1 2020}: Dilengkapi dengan prosesor \textit{Apple M1}, memori (RAM) sebesar \SI{8}{\giga\byte}, dan penyimpanan internal berbasis \textit{SSD}. Perangkat ini digunakan untuk persiapan data, eksplorasi awal, pengembangan kode, serta penyusunan laporan penelitian.
  \item \textbf{\textit{Google Colaboratory}}: Lingkungan komputasi berbasis awan yang menyediakan akses ke unit pemrosesan grafis (\textit{GPU}) \textit{NVIDIA Tesla T4} dengan memori \SIrange{15}{20}{\giga\byte}. Platform ini digunakan untuk pelatihan model \textit{Informer}, evaluasi performa model, serta komputasi intensif lainnya.
\end{enumerate}

\subsection{Perangkat Lunak}
\begin{enumerate}
  \item \textbf{Sistem Operasi}:
  \begin{itemize}
    \item \textit{macOS Ventura} 13.6: Digunakan pada perangkat lokal untuk pengembangan dan pengujian awal.
    \item \textit{Ubuntu}: Digunakan dalam lingkungan virtual \textit{Google Colaboratory} untuk komputasi berbasis awan.
  \end{itemize}
  
  \item \textbf{Bahasa Pemrograman dan Lingkungan Kerja}:
  \begin{itemize}
    \item Python 3.10: Bahasa pemrograman utama untuk pengembangan algoritma dan analisis data.
    \item \textit{Jupyter Notebook}: Digunakan baik secara lokal maupun pada \textit{Google Colaboratory} untuk pengembangan kode interaktif dan dokumentasi analisis.
  \end{itemize}
  
  \item \textbf{Pustaka dan Modul Python}:
  \begin{itemize}
    \item \textit{NumPy} (versi 2.2.0) dan \textit{Pandas} (versi 2.2.3): Untuk manipulasi, eksplorasi, dan analisis data numerik serta tabular.
    \item \textit{h5py} (versi 3.13.0): Untuk pengelolaan dataset dalam format \textit{HDF5}, termasuk pembacaan dan penyimpanan data.
    \item \textit{scikit-learn} (versi 1.6.1): Untuk prapemrosesan data, evaluasi model \textit{machine learning}, dan visualisasi dimensi rendah menggunakan algoritma \textit{t-SNE} (\texttt{sklearn.manifold.TSNE}).
    \item \textit{Matplotlib} (versi 3.10.3): Untuk pembuatan visualisasi grafik dan representasi hasil analisis.
    \item \textit{Joblib} (versi 1.2.0): Untuk serialisasi model dan optimalisasi pemrosesan paralel.
    \item \textit{PyTorch} (versi 2.7.0): Kerangka kerja pembelajaran mendalam untuk pengembangan, pelatihan, dan evaluasi model.
    \item \textit{TensorBoard} (versi 2.19.0): Untuk pemantauan metrik pelatihan dan validasi model secara \textit{real-time}.
    \item \textit{itertools}: Modul standar Python untuk operasi iterasi kompleks.
  \end{itemize}
\end{enumerate}

\subsection{Sumber Data}
\begin{enumerate}
\item \textbf{\textit{NIST Atomic Spectra Database} (\textit{ASD})}: Basis data resmi dari \textit{National Institute of Standards and Technology} (\textit{NIST}) yang digunakan sebagai sumber utama parameter spektral atomik. Parameter yang diambil meliputi energi ionisasi (\(E_\text{i}\)), energi keadaan (\(E_\text{k}\)), bobot statistik (\(g_\text{i}\), \(g_\text{k}\)), dan koefisien probabilitas transisi (\(A_\text{ki}\)). Data diakses melalui antarmuka daring resmi \textit{NIST}.%\footnote{\url{https://physics.nist.gov/PhysRefData/ASD/}}
\end{enumerate}




\section{Prosedur Penelitian}

Penelitian ini mengadopsi prosedur sistematis yang mencakup studi literatur, perancangan model simulasi, perancangan model \textit{Informer} hingga membangun sistem prediksi spektrum LIBS. Seperti yang ditunjukkan pada Gambar~\ref{diagram}, prosedur penelitian ini terdiri dari beberapa langkah utama yang saling berkaitan. Prosedur ini diuraikan dalam langkah-langkah berikut:

\begin{enumerate}

    \begin{figure}[H]
        \centering
        \includegraphics[width=1\textwidth]{images/3-Diagram.drawio.pdf}
        \caption{Diagram alur penelitian.}
        \label{diagram}
    \end{figure}

    \item \textbf{Studi Literatur} \\
    Studi literatur dilakukan secara komprehensif untuk membentuk landasan teoretis yang kokoh bagi simulasi spektrum emisi atom dalam kondisi LTE. Proses ini melibatkan analisis mendalam terhadap prinsip-prinsip fisika plasma, yang mencakup:
    \begin{enumerate}
      \item \textbf{Kalkulasi Intensitas Spektral} \\
      Intensitas spektrum atom pada suhu tertentu dihitung menggunakan distribusi Boltzmann, yang menggambarkan distribusi populasi tingkat energi, yang kemudian dirumuskan dalam persamaan \eqref{eq:intensity_relative}.

      \item \textbf{Kalkulasi Rasio Ionisasi Plasma} \\
      Rasio antara atom terionisasi dan netral dalam plasma ditentukan melalui persamaan Saha, yang mempertimbangkan parameter termodinamika dan fungsi partisi, dan juga densitas elektron yang kemudian disebut pada persamaan \eqref{eq:saha_final_int}.

      \item \textbf{Profil Garis Spektral Voigt} \\
      Profil Voigt menggabungkan kontribusi pelebaran Doppler (berdistribusi Gaussian) dan pelebaran tekanan (berdistribusi Lorentzian) untuk memodelkan garis spektral secara akurat, dengan lebar HWHM Gaussian pada \eqref{eq:sigma_doppler}, lebar HWHM Lorentzian pada \eqref{eq:stark_broadening}, dan profil Voigt secara keseluruhan pada \eqref{eq:voigt}.
    \end{enumerate}

    Data transisi atom yang digunakan dalam simulasi divalidasi dengan merujuk pada basis data NIST \citep{Kramida2023}, yang menyediakan informasi spektral atom dan ion yang andal.

    \item \textbf{Pemodelan Simulasi Spektrum Emisi Atom} \\
    Simulasi spektrum emisi atom dilaksanakan dalam kondisi LTE, dengan asumsi bahwa plasma memiliki distribusi energi termal yang seragam, dikendalikan oleh suhu \(T\) dan densitas elektron \(n_e\). Pendekatan ini berfokus pada interaksi atom-elektron yang menghasilkan emisi foton pada panjang gelombang tertentu. Intensitas garis spektral ditentukan oleh probabilitas transisi dan populasi tingkat energi, sedangkan pelebaran garis dimodelkan menggunakan profil Voigt untuk memperhitungkan efek Doppler dan tekanan.

    Secara matematis, simulasi ini menghitung fungsi partisi, rasio ionisasi, intensitas relatif, dan profil garis dengan persamaan berikut:
    \begin{equation}
    I_{\text{rel}} = \frac{N g_k A_{ki} \exp\left(-\frac{E_k}{k_B T}\right)}{Z}
    \end{equation}
    \begin{equation}
    I(\lambda) = I_{\text{rel}} \cdot V(\lambda - \lambda_{ij}, \alpha_G, \alpha_L)
    \end{equation}
    di mana \(I_{\text{rel}}\) adalah intensitas relatif, \(A_{ki}\) adalah probabilitas transisi, \(Z\) adalah fungsi partisi, \(f\) adalah faktor koreksi, dan \(V\) adalah fungsi profil Voigt dengan parameter HWHM Gaussian \(\alpha_G\) dan Lorentzian \(\alpha_L\).
    \begin{table}[h]
      \centering
      \small
      \caption{Tabel Hyperparameter untuk Semua Algoritma Simulasi Spektral Atomik}
      \label{tab:hyperparameter}
      \begin{tabular}{p{2cm} p{5cm} p{3cm} p{3cm}}
        \toprule
        \textbf{Simbol} & \textbf{Deskripsi} & \textbf{Satuan} & \textbf{Nilai Default} \\
        \midrule
        $k$ & Jumlah pasangan elemen-ion & -- & 4 \\
        $N$ & Jumlah maksimum sampel spektral & -- & Tidak ditentukan ($\mathbb{N}$) \\
        $\Delta T$ & Langkah suhu & \si{\kelvin} & 1000 \\
        $\Delta n_e$ & Langkah densitas elektron & \si{\per\cubic\centi\metre} & $10^{0.5} \times 10^{12} \approx 3.162 \times 10^{12}$ \\
        $T_{\text{min}}$ & Batas bawah rentang suhu & \si{\kelvin} & 5000 \\
        $T_{\text{max}}$ & Batas atas rentang suhu & \si{\kelvin} table& 15000 \\
        $n_{e,\text{min}}$ & Batas bawah rentang densitas elektron & \si{\per\cubic\centi\metre} & $10^{12}$ \\
        $n_{e,\text{max}}$ & Batas atas rentang densitas elektron & \si{\per\cubic\centi\metre} & $10^{16}$ \\
        Modulus penyimpanan & Frekuensi penyimpanan $\mathcal{S}$ & -- & 1000 \\
        \bottomrule
      \end{tabular}
    \end{table}

    Untuk efisiensi komputasi, algoritma dirancang dengan memilih subset atom secara acak, menerapkan langkah diskrit untuk \(T\) dan \(n_e\), serta menyimpan spektrum secara periodik. Proses ini menghasilkan dataset spektral \(\mathcal{S}\) yang siap digunakan untuk pelatihan model \textit{Informer}.

    Pemodelan simulasi terdiri dari empat sub-langkah berikut:
    \begin{enumerate}
        \item \textbf{Inisialisasi Data Spektral Atom} \\
        Algoritma ini mengumpulkan dan memvalidasi data transisi atom dari basis data NIST, menginisialisasi set kandidat atom (\(\mathcal{C}\)), kamus subset atom (\(\mathcal{A}\)), dan set spektrum (\(\mathcal{S}\)). Subset atom dipilih secara acak untuk setiap kombinasi \(T\) dan \(n_e\) guna menjamin representasi yang beragam. Algoritma dapat dilihat pada  \textbf{Lampiran~\ref{app:algo1}}.

        \item \textbf{Kalkulasi Rasio Populasi Ionisasi} \\
        Algoritma ini menghitung rasio populasi ionisasi untuk setiap pasangan elemen-ion, menghasilkan kamus rasio populasi (\(\mathcal{R}\)). Rasio ini penting untuk menentukan fraksi atom netral dan terionisasi dalam plasma. Algoritma dapat dilihat pada \textbf{Lampiran~\ref{app:algo2}}.

        \item \textbf{Kalkulasi Intensitas Garis Spektral} \\
        Algoritma ini menghitung intensitas relatif garis spektral (\(I_{\text{rel}}\)) yang menghasilkan kamus intensitas sementara (\(\mathcal{I}\)). Intensitas ini mencerminkan probabilitas emisi foton. Algoritma dapat dilihat pada  \textbf{Lampiran~\ref{app:algo3}}.

        \item \textbf{Kalkulasi Spektrum Emisi dengan Profil Voigt} \\
        Algoritma ini menghasilkan nilai HWHM Gaussian (\(\alpha_G\)) dan Lorentzian (\(\alpha_L\)) yang kemudian mengaplikasikan profil Voigt untuk menghasilkan spektrum emisi akhir, yang disimpan dalam set spektrum (\(\mathcal{S}\)) dengan normalisasi intensitas.Algoritma dapat dilihat pada  \textbf{Lampiran~\ref{app:algo4}}.
    \end{enumerate}

    \item \textbf{Perancangan Model Informer} \\
    \begin{table}[h]
      \centering
      \small
      \caption{Tabel Hyperparameter untuk Model Informer}
      \label{tab:hyperparameter_informer}
      \begin{tabular}{p{3cm} p{5cm} p{3cm}}
        \toprule
        \textbf{Simbol} & \textbf{Deskripsi} & \textbf{Nilai Default} \\
        \midrule
        $d_{\text{model}}$ & Dimensi embedding & 64 \\
        $H$ & Jumlah kepala perhatian & 4 \\
        $L$ & Jumlah lapisan enkoder & 3 \\
        $d_{\text{ff}}$ & Dimensi feedforward & 128 \\
        $\beta$ & Faktor dropout & 0.1 \\
        $S$ & Panjang sekuens & 4096 \\
        $A$ & Faktor perhatian & 7 \\
        $B$ & Ukuran batch & 4 \\
        $E$ & Jumlah epoch & 30 \\
        $\eta$ & Laju pembelajaran & $10^{-3}$ \\
        \bottomrule
      \end{tabular}
    \end{table}
    Tahap ini melibatkan perancangan model \textit{Informer}, sebuah arsitektur Transformer yang dioptimalkan untuk data sekuensial panjang, untuk memprediksi atom berdasarkan dataset spektral \(\mathcal{S}\). Model ini memanfaatkan mekanisme perhatian probabilitas guna menangkap pola spektral yang kompleks. Pelatihan model bertujuan meminimalkan \textit{mean squared error} (MSE):
    \[
    \text{MSE} = \frac{1}{n} \sum_{i=1}^n (y_i - \hat{y}_i)^2,
    \]
    di mana \(y_i\) adalah nilai aktual dan \(\hat{y}_i\) adalah nilai prediksi model.

    Hyperparameter model disesuaikan untuk mencapai keseimbangan optimal antara akurasi dan efisiensi komputasi, sebagaimana dirinci dalam Tabel~\ref{tab:hyperparameter_informer}.
    Model Informer dirancang dengan beberapa komponen utama:
    \begin{enumerate}
      \item \textbf{Enkoder} \\
      Enkoder terdiri dari beberapa lapisan yang menerapkan mekanisme perhatian probabilitas untuk menangkap hubungan antar elemen spektral. Setiap lapisan enkoder mengolah input sekuensial dan menghasilkan representasi yang kaya.

      \item \textbf{ProbSparse Self-Attention} \\
      Mekanisme perhatian ini mengurangi kompleksitas komputasi dengan hanya mempertimbangkan subset elemen yang relevan, sehingga memungkinkan model menangani sekuens panjang dengan efisien.

      \item \textbf{Feed-Forward Network (FFN)} \\
      FFN mengaplikasikan transformasi non-linear pada representasi dari lapisan perhatian untuk meningkatkan kemampuan model dalam menangkap pola kompleks dalam data spektral.

      \item \textbf{Normalisasi Layer dan Dropout} \\
      Normalisasi layer diterapkan untuk stabilisasi pelatihan, sedangkan dropout digunakan untuk mencegah overfitting. Keduanya berkontribusi pada generalisasi model yang lebih baik.

      \item \textbf{Output Layer} \\
      Lapisan output menghasilkan prediksi spektral akhir, yang diukur dengan MSE terhadap nilai aktual. Output ini mencakup intensitas relatif dan profil garis spektral untuk setiap elemen yang diprediksi.
    \end{enumerate}   
    \begin{figure}
        \centering
        \includegraphics[width=0.8\textwidth]{images/informer.drawio.pdf}
        \caption{Arsitektur model Informer yang digunakan dalam penelitian ini.}
        \label{fig:informer_architecture}
    \end{figure}
\end{enumerate}






%-------------------------------------------------------------------------------
%                            BAB IV
%               		HASIL DAN PEMBAHASAN
%-------------------------------------------------------------------------------
% \fancyhf{} 
% \fancyfoot[R]{\thepage}
\chapter{HASIL DAN PEMBAHASAN}
%\thispagestyle{plain} % Halaman pertama bab menggunakan gaya plain


\begin{table}[H]
    \centering
    \caption{Hasil Pengujian Akurasi Menggunakan SVM Terhadap Data \textit{Training} dan \textit{Testing}}
    \label{tb_detail_akurasi_face}
    \begin{tabular}{lcccc}
    \toprule
    \textbf{Jenis Data} & \textbf{Jumlah Label} & \textbf{Jumlah Data} & {\color[HTML]{000000} \textbf{Akurasi}} \\ 
    \midrule
    {\color[HTML]{000000} Training} & {\color[HTML]{000000} 41} & {\color[HTML]{000000} 1640} & {\color[HTML]{000000} 99,51\%} \\ 
    {\color[HTML]{000000} Testing} & {\color[HTML]{000000} 41} & {\color[HTML]{000000} 410} & {\color[HTML]{000000} 96,34\%} \\ 
    \bottomrule
    \end{tabular}
    \end{table}

    \begin{table}[H]
        \centering
        \caption{Perbandingan Performa Model Performa Model Performa Model Performa Model Machine Learning pada Data LIBS}
        \label{tab:performa_ml}
        \centering
        \begin{tabular}{lcccc}
          \toprule
          Model & Akurasi (\%) & Presisi (\%) & Recall (\%) & RMSE \\
          \midrule
          Random Forest & 95.2 & 94.8 & 95.1 & 0.12 \\
          SVM & 89.7 & 88.5 & 90.2 & 0.21 \\
          Transformer & 97.1 & 96.9 & 97.0 & 0.08 \\
          CNN & 93.4 & 92.7 & 93.5 & 0.15 \\
          \bottomrule
        \end{tabular}
        
        \smallskip
        \footnotesize
        \textit{Keterangan:} Data diperoleh dari 100 sampel logam dengan 5 kelas komposisi.
      \end{table}
    

% Baris ini digunakan untuk membantu dalam melakukan sitasi
% Karena diapit dengan comment, maka baris ini akan diabaikan
% oleh compiler LaTeX.
\begin{comment}
\bibliography{daftar-pustaka}
\end{comment}
%-------------------------------------------------------------------------------
%                            BAB V
%               		KESIMPULAN DAN SARAN
%-------------------------------------------------------------------------------
% \fancyhf{} 
% \fancyfoot[R]{\thepage}
\chapter{KESIMPULAN DAN SARAN}
%\thispagestyle{plain} % Halaman pertama bab menggunakan gaya plain

\section{Kesimpulan}
[Isi Bab 5.1: Kontribusi dan batasan.]

\section{Saran}
\subsection{Eksperimen Validasi}
[Isi Bab 5.2.1]

\subsection{Model Non-LTE}
[Isi Bab 5.2.2]

\subsection{Efek Matriks}
[Isi Bab 5.2.3]

\subsection{Optimalisasi Transformer}
[Isi Bab 5.2.4]
%-----------------------------------------------------------------------------%



\fancypagestyle{daftarpustaka}{
  \fancyhf{}
  \fancyfoot[R]{\thepage}
  \renewcommand{\headrulewidth}{0pt}
  \renewcommand{\footrulewidth}{0pt}
}

\addcontentsline{toc}{chapter}{DAFTAR PUSTAKA}
\begin{onehalfspace}
  \begin{spacing}{1}
    \pagestyle{daftarpustaka}
  \end{spacing}
  \bibliography{daftar-pustaka}
\end{onehalfspace}

% File: include/lampiran.tex
% Lampiran untuk Tugas Akhir Jurusan Informatika Unsyiah
% Sesuai Panduan Tugas Akhir dan Tesis 2024 FMIPA Universitas Syiah Kuala
% Struktur: Hanya Lampiran 1, 2, dst., tanpa subbab atau section

% \begin{onehalfspace} % Spasi 1,5 sesuai dokumen utama

\lampiran{Algoritma Inisialisasi Data Spektral Atom}
\label{app:algo1}
\begin{algoritma}[H]
\small
\caption{Inisialisasi Data Spektral Atom}
\begin{algorithmic}[1]
  \REQUIRE Transition dataset $\mathcal{D} = \{ (\lambda_{ij}, E_i, E_k, g_i, g_k, A_{ki}) \mid \lambda_{ij} \in [200, 900], E_i, E_k, g_i, g_k, A_{ki} > 0 \}$ ; Number of element-ion pairs $k = 4$; Maximum spectral samples $N \in \mathbb{N}$; Temperature step $\Delta T > 0$; Electron density step $\Delta n_e > 0$
  \ENSURE Candidate atom set $\mathcal{C}$, atom subset dictionary $\mathcal{A}$, spectra set $\mathcal{S}$
  \STATE Validate inputs: Ensure $N > 0$, $\Delta T > 0$, $\Delta n_e > 0$
  \STATE Initialize: $\Delta T \gets 1000$, $\Delta n_e \gets 10^{0.5} \times 10^{12}$
  \STATE Define: $\mathcal{C} \gets \{\text{H}, \text{He}, \text{O}, \text{N}, \text{Si}, \text{Al}, \text{Fe}, \text{Ca}, \text{Mg}, \text{Na}, \text{Ti}, \text{Mn}, \text{S}, \text{Cl}, \text{Cr}, \text{Ni}, \text{Cu}\}$
  \STATE Initialize: $\mathcal{S} \gets \emptyset$, $\mathcal{A} \gets \emptyset$
  \FORALL{$T \in [5000, 15000]$ \textbf{step} $\Delta T$}
    \FORALL{$n_e \in [10^{12}, 10^{16}]$ \textbf{step} $\Delta n_e$}
      \STATE Randomly select $\mathcal{A}_{T,n_e} \subseteq \mathcal{C}$ with $|\mathcal{A}_{T,n_e}| = k$ without replacement
      \IF{no transitions exist in $\mathcal{D}$ for any species in $\mathcal{A}_{T,n_e}$}
        \STATE Log warning: ``No transitions for $\mathcal{A}_{T,n_e}$ at $T$, $n_e$'' \COMMENT{Skip}
        \STATE continue
      \ENDIF
      \STATE Store $(T, n_e, \mathcal{A}_{T,n_e})$ in $\mathcal{A}$
    \ENDFOR
  \ENDFOR
  \RETURN $\mathcal{C}$, $\mathcal{A}$, $\mathcal{S}$
\end{algorithmic}
\end{algoritma}
% Algoritma ini mengumpulkan dan memvalidasi data transisi atom dari basis data NIST, menginisialisasi set kandidat atom (\(\mathcal{C}\)), kamus subset atom (\(\mathcal{A}\)), dan set spektrum (\(\mathcal{S}\)). Subset atom dipilih secara acak untuk setiap kombinasi \(T\) dan \(n_e\), memastikan representasi yang beragam.

\lampiran{Algoritma Kalkulasi Rasio Populasi Ionisasi}
\label{app:algo2}
\begin{algoritma}[h]
\small
\caption{Kalkulasi Rasio Populasi Ionisasi}
\begin{algorithmic}[1]
  \REQUIRE Transition dataset $\mathcal{D} = \{ (\lambda_{ij}, E_i, E_k, g_i, g_k, A_{ki}) \mid \lambda_{ij} \in [200, 900], E_i, E_k, g_i, g_k, A_{ki} > 0 \}$; \\ Candidate atom set $\mathcal{C}$; \\ Atom subset dictionary $\mathcal{A} = \{ (T, n_e, \mathcal{A}_{T,n_e}) \}$; \\ Physical constants $m_e = 9.109 \times 10^{-31}$, $k_B = 8.617 \times 10^{-5}$, $h = 4.1357 \times 10^{-15}$
  \ENSURE Population ratio dictionary $\mathcal{R} = \{ (T, n_e, \mathbf{R}_{T,n_e}) \mid \mathbf{R}_{T,n_e} \in [0, 1]^2 \}$
  \STATE Initialize $\mathcal{R} \gets \emptyset$ \COMMENT{Ratio dictionary}
  \FORALL{$(T, n_e, \mathcal{A}_{T,n_e}) \in \mathcal{A}$}
    \STATE $\mathbf{R}_{T,n_e} \gets \emptyset$ \COMMENT{Temporary ratio}
    \FORALL{$S \in \mathcal{A}_{T,n_e}$}
      \STATE Define $(S_{\text{neutral}}, S_{\text{ion}}) \gets (S_{\text{neutral}}, S_{\text{ion}})$ \COMMENT{Species pair}
      \STATE Extract $\mathcal{T}_S \subseteq \mathcal{D}$ for $S_{\text{neutral}}$ or $S_{\text{ion}}$ \COMMENT{Transitions}
      \IF{$\mathcal{T}_S = \emptyset$}
        \STATE Log warning: ``No transitions for $S$'' \COMMENT{Skip}
        \STATE continue
      \ENDIF
      \STATE $Z_{\text{neutral}} \gets \sum_i g_i \exp\left(-\frac{E_i}{k_B T}\right)$ \COMMENT{Neutral partition}
      \IF{$Z_{\text{neutral}} \leq 0$}
        \STATE Log warning: ``Invalid partition function for $S_{\text{neutral}}$'' \COMMENT{Skip}
        \STATE continue
      \ENDIF
      \STATE $Z_{\text{ion}} \gets \sum_i g_i \exp\left(-\frac{E_i}{k_B T}\right)$ \COMMENT{Ion partition}
      \IF{$Z_{\text{ion}} \leq 0$}
        \STATE Log warning: ``Invalid partition function for $S_{\text{ion}}$'' \COMMENT{Skip}
        \STATE continue
      \ENDIF
      \STATE $\frac{N_{\text{ion}}}{N_{\text{neutral}}} \gets \frac{2 Z_{\text{ion}}}{n_e Z_{\text{neutral}}} \left( \frac{2\pi m_e k_B T}{h^2} \right)^{3/2} \exp\left(-\frac{E_{\text{ion}}}{k_B T}\right)$ \COMMENT{Ionization ratio}
      \STATE $f_{\text{neutral}} \gets \frac{1}{1 + \frac{N_{\text{ion}}}{N_{\text{neutral}}}}$ \COMMENT{Neutral fraction}
      \STATE $f_{\text{ion}} \gets \frac{\frac{N_{\text{ion}}}{N_{\text{neutral}}}}{1 + \frac{N_{\text{ion}}}{N_{\text{neutral}}}}$ \COMMENT{Ion fraction}
      \STATE Store $(f_{\text{neutral}}, f_{\text{ion}})$ in $\mathbf{R}_{T,n_e}[S_{\text{neutral}}, S_{\text{ion}}]$ \COMMENT{Store ratio}
    \ENDFOR
    \STATE Store $(T, n_e, \mathbf{R}_{T,n_e})$ in $\mathcal{R}$ \COMMENT{Add to dictionary}
  \ENDFOR
  \RETURN $\mathcal{R}$ \COMMENT{Result}
\end{algorithmic}
\end{algoritma}
% Algoritma ini menghitung rasio populasi ionisasi untuk setiap pasangan elemen-ion, menghasilkan kamus rasio populasi (\(\mathcal{R}\)). Rasio ini penting untuk menentukan fraksi atom netral dan terionisasi dalam plasma.

\lampiran{Algoritma Kalkulasi Intensitas Garis Spektral}
\label{app:algo3}
\begin{algoritma}[H]
\small
\caption{Kalkulasi Intensitas Garis Spektral}
\begin{algorithmic}[1]
  \REQUIRE Transition dataset $\mathcal{D} = \{ (\lambda_{ij}, E_i, E_k, g_i, g_k, A_{ki}, m_a, w) \mid \lambda_{ij} \in [200, 900], E_i, E_k, g_i, g_k, A_{ki}, m_a, w > 0 \}$; \\ Atom subset dictionary $\mathcal{A} = \{ (T, n_e, \mathcal{A}_{T,n_e}) \}$; \\ Population ratio dictionary $\mathcal{R} = \{ (T, n_e, \mathbf{R}_{T,n_e}) \mid \mathbf{R}_{T,n_e} \in [0, 1]^2 \}$; \\ Physical constants $c = 2.998 \times 10^8$, $k_B = 8.617 \times 10^{-5}$
  \ENSURE Temporary intensity dictionary $\mathcal{I} = \{ (T, n_e, \mathbf{I}_{\text{temp}}) \mid \mathbf{I}_{\text{temp}} = \{ (\lambda_{ij}, I_{\text{rel}}) \} \}$
  \STATE Initialize $\mathcal{I} \gets \emptyset$ \COMMENT{Temporary intensity dictionary}
  \FORALL{$(T, n_e, \mathcal{A}_{T,n_e}) \in \mathcal{A}$}
    \STATE Extract $\mathbf{R}_{T,n_e}$ from $\mathcal{R}$ \COMMENT{Population ratio}
    \STATE $\mathbf{I}_{\text{temp}} \gets \emptyset$ \COMMENT{Temporary intensity list}
    \FORALL{$S \in \mathcal{A}_{T,n_e}$}
      \STATE Extract $\mathcal{T}_S \subseteq \mathcal{D}$ for $S_{\text{neutral}}$ or $S_{\text{ion}}$ \COMMENT{Transitions}
      \IF{$\mathcal{T}_S = \emptyset$}
        \STATE Log warning: ``No transitions for $S$'' \COMMENT{Skip}
        \STATE continue
      \ENDIF
      \FORALL{$(\lambda_{ij}, E_i, E_k, g_i, g_k, A_{ki}, m_a, w) \in \mathcal{T}_S$}
        \STATE $\Delta E \gets E_k - E_i$ \COMMENT{Energy}
        \STATE $n_{e,\text{min}} \gets 1.6 \times 10^{12} T^{1/2} (\Delta E)^{3/2}$ \COMMENT{Minimum density}
        \IF{$n_e \geq n_{e,\text{min}}$}
          \IF{$S$ is $S_{\text{neutral}}$}
            \STATE $Z \gets Z_{\text{neutral}}$ (Algoritma 2) \COMMENT{Neutral partition}
            \STATE $f \gets f_{\text{neutral}}$ from $\mathbf{R}_{T,n_e}$ \COMMENT{Neutral fraction}
          \ELSE
            \STATE $Z \gets Z_{\text{ion}}$ (Algoritma 2) \COMMENT{Ion partition}
            \STATE $f \gets f_{\text{ion}}$ from $\mathbf{R}_{T,n_e}$ \COMMENT{Ion fraction}
          \ENDIF
          \STATE $I_{\text{rel}} \gets \frac{g_k A_{ki} \exp\left(-\frac{E_k}{k_B T}\right)}{Z} \cdot f$ \COMMENT{Relative intensity}
          \STATE Append $(\lambda_{ij}, I_{\text{rel}})$ to $\mathbf{I}_{\text{temp}}$ \COMMENT{Accumulate}
        \ELSE
          \STATE Log warning: ``LTE condition not satisfied'' \COMMENT{Skip}
        \ENDIF
      \ENDFOR
    \ENDFOR
    \STATE Store $(T, n_e, \mathbf{I}_{\text{temp}})$ in $\mathcal{I}$ \COMMENT{Add to dictionary}
  \ENDFOR
  \RETURN $\mathcal{I}$ \COMMENT{Result}
\end{algorithmic}
\end{algoritma}
% \par Algoritma ini menghitung intensitas relatif garis spektral (\(I_{\text{rel}}\)) untuk setiap transisi atom, menghasilkan kamus intensitas sementara (\(\mathcal{I}\)). Intensitas ini mencerminkan probabilitas emisi foton.

\lampiran{Algoritma Kalkulasi Spektrum Emisi dengan Profil Voigt}
\label{app:algo4}
\begin{algoritma}[H]
\small
\caption{Kalkulasi Spektrum Emisi Atom dengan Profil Garis \textit{Voigt}}
\begin{algorithmic}[1]
  \REQUIRE Temporary intensity dictionary $\mathcal{I} = \{ (T, n_e, \mathbf{I}_{\text{temp}}) \mid \mathbf{I}_{\text{temp}} = \{ (\lambda_{ij}, I_{\text{rel}}) \} \}$; \\ Transition dataset $\mathcal{D} = \{ (\lambda_{ij}, E_i, E_k, g_i, g_k, A_{ki}, m_a, w) \mid \lambda_{ij} \in [200, 900], E_i, E_k, g_i, g_k, A_{ki}, m_a, w > 0 \}$; \\ Atom subset dictionary $\mathcal{A} = \{ (T, n_e, \mathcal{A}_{T,n_e}) \}$; \\ Population ratio dictionary $\mathcal{R} = \{ (T, n_e, \mathbf{R}_{T,n_e}) \mid \mathbf{R}_{T,n_e} \in [0, 1]^2 \}$; \\ Physical constants $c = 2.998 \times 10^8$, $k_B = 8.617 \times 10^{-5}$; \\ Maximum spectral samples $N \in \mathbb{N}$; \\ Spectra set $\mathcal{S}$
  \ENSURE $\mathcal{S}$ with $(T, n_e, \mathbf{I}_{T,n_e}, \mathbf{R}_{T,n_e})$
  \STATE Initialize $c \gets 0$ \COMMENT{Counter}
  \FORALL{$(T, n_e, \mathbf{I}_{\text{temp}}) \in \mathcal{I}$}
    \STATE Extract $\mathbf{R}_{T,n_e}$ from $\mathcal{R}$ \COMMENT{Population ratio}
    \STATE $\mathbf{I}_{T,n_e} \gets \emptyset$ \COMMENT{Final intensity list}
    \FORALL{$(\lambda_{ij}, I_{\text{rel}}) \in \mathbf{I}_{\text{temp}}$}
      \STATE Extract $m_a, w$ from $\mathcal{D}$ for the transition at $\lambda_{ij}$ \COMMENT{Atom mass and Lorentzian width}
      \STATE $\alpha_G \gets \frac{\lambda_{ij}}{c} \sqrt{\frac{2 k_B T \ln 2}{m_a}}$ \COMMENT{Gaussian HWHM (Doppler)}
      \STATE $\alpha_L \gets w \cdot \frac{n_e}{10^{16} \sqrt{T}}$ \COMMENT{Lorentzian HWHM}
      \STATE $I(\lambda) \gets I_{\text{rel}} \cdot V(\lambda - \lambda_{ij}, \alpha_G, \alpha_L)$ \COMMENT{Voigt profile}
      \STATE Append $(\lambda_{ij}, I(\lambda))$ to $\mathbf{I}_{T,n_e}$ \COMMENT{Accumulate}
    \ENDFOR
    \IF{$\max(\{I \mid (\lambda_{ij}, I) \in \mathbf{I}_{T,n_e}\}) > 0$}
      \STATE Normalize $\mathbf{I}_{T,n_e}$: divide each $I$ by $\max(\{I \mid (\lambda_{ij}, I) \in \mathbf{I}_{T,n_e}\})$ \COMMENT{Normalization}
    \ENDIF
    \STATE Store $(T, n_e, \mathbf{I}_{T,n_e}, \mathbf{R}_{T,n_e})$ in $\mathcal{S}$ \COMMENT{Add spectrum}
    \STATE $c \gets c + 1$ \COMMENT{Increment}
    \IF{$c \mod 1000 = 0$} %\COMMENT{Check storage modulo}
      \STATE Save $\mathcal{S}$ \COMMENT{Save}
      \STATE Clear $\mathcal{S}$ \COMMENT{Clear}
    \ENDIF
    \IF{$c \geq N$}
      \STATE break \COMMENT{Terminate}
    \ENDIF
  \ENDFOR
  \RETURN $\mathcal{S}$ \COMMENT{Result}
\end{algorithmic}
\end{algoritma}
% Algoritma ini menghitung parameter pelebaran Gaussian (\(\alpha_G\)) dan Lorentzian (\(\alpha_L\)), lalu mengaplikasikan profil Voigt untuk menghasilkan spektrum emisi akhir, menyimpannya dalam set spektrum (\(\mathcal{S}\)) dengan normalisasi intensitas.

% % \end{onehalfspace}
\addcontentsline{toc}{chapter}{LAMPIRAN}

\end{onehalfspace}
\end{document}